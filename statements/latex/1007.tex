\documentclass[11pt,a4paper]{article}

\usepackage{../../templates/style}

\begin{document}

\begin{problem}{Bee}{standard input}{standard output}{1 second}{64 megabytes}

ผึ้งสายพันธุ์หนึ่งประกอบด้วย นางพญา ผึ้งงาน และผึ้งทหาร การเจริญพันธุ์เต็มวัยของผึ้งในสายพันธุ์นี้ มีระยะเวลาหนึ่งปี เมื่อผสมพันธุ์แล้วผึ้งทหารหนึ่งตัวสามารถให้กำเนิดลูกเป็นผึ้งงานได้เพียงหนึ่งตัว ส่วนผึ้งงานหนึ่งตัวสามารถให้กำเนิดลูกได้สองตัวเป็นผึ้งงานและผึ้งทหารอย่างละหนึ่งตัว เมื่อให้กำเนิดลูกผึ้งแล้วผึ้งงานและผึ้งทหารที่เป็นผู้ให้กำเนิดจะตายไป สำหรับนางพญาสามารถให้กำเนิดลูกเป็นผึ้งงานได้เพียงหนึ่งตัว และมีชีวิตอยู่ตลอดไปไม่มีวันตาย

ในทำนองเดียวกันกับผึ้งรุ่นก่อน ผึ้งที่เกิดใหม่เมื่อมีอายุได้หนึ่งปีจะเจริญพันธุ์เต็มวัย มีการผสมพันธุ์ และให้กำเนิดลูกผึ้งรุ่นต่อไปตามกฎในย่อหน้าแรก และสำหรับนางพญาเมื่อให้กำเนิดลูกผึ้งครบหนึ่งปีแล้ว สามารถผสมพันธุ์และให้กำเนิดลูกผึ้งได้เช่นเดียวกัน

\textbf{กำหนดให้ผึ้งรังหนึ่งเริ่มต้นด้วยนางพญาหนึ่งตัวและผึ้งงานอีกหนึ่งตัว} ดังนั้นเมื่อสิ้นปีแรก (นับเป็นปีที่หนึ่ง)ผึ้งรังนี้จะมีนางพญาจำนวนหนึ่งตัว, ผึ้งงาน(ที่เกิดจากนางพญา) จำนวนหนึ่งตัว, ผึ้งทหารและผึ้งงาน อย่างละหนึ่งตัว (ที่เกิดจากผึ้งงานในรุ่นก่อน ซึ่งเมื่อให้กำเนิดลูกผึ้งแล้วตายไป) รวมเป็นผึ้งในรังทั้งสิ้น 4 ตัว และโดยวิธีการเดียวกันในปีที่สองผึ้งรังนี้จะประกอบด้วยนางพญาจำนวนหนึ่งตัว ผึ้งทหารจำนวนสองตัว และผึ้งงานจำนวนสี่ตัว รวมเป็นผึ้งในรังทั้งสิ้น 7 ตัว


\underline{\textbf{โจทย์}} จงเขียนโปรแกรมเพื่อคำนวณจำนวนผึ้งงานและผึ้งทั้งหมดในรังของแต่ละปีที่กำหนด

\InputFile

\textbf{มีบรรทัดเดียว} รับจำนวนเต็มตั้งแต่สองจำนวนขึ้นไป โดยจำนวนแรกจนถึงจำนวนรองสุดท้าย เป็นจำนวนเต็มบวกแทนปีที่ต้องการคำนวณหาจำนวนผึ้งในรัง ค่าสุดท้ายเป็น $-1$ ซึ่งใช้เป็นรหัสปิดท้ายข้อมูล โดยมีรายละเอียดดังนี้
\begin{enumerate}

\item จำนวนปีที่ต้องการคำนวณมีค่าได้ตั้งแต่ $1$ ถึง $24$ จำนวน
\item ค่าตัวเลขของปีในข้อ 1. เป็นตัวเลขที่ไม่ซ้ำกัน และมีค่าได้ตั้งแต่ $1$ ถึง $24$
\item รหัสปิดท้ายข้อมูล (sentinel) มีค่าเป็น $-1$ เสมอใช้แสดงว่าข้อมูลที่ต้องทำการประมวลผลหมดแล้ว ให้เลิกทำงาน และไม่ต้องประมวลผลค่านี้
\item ข้อมูลแต่ละจำนวนแยกจากกันด้วยเครื่องหมายเว้นวรรคจำนวน $1$ วรรค
\end{enumerate}

\OutputFile

\textbf{มีหลายบรรทัด} จำนวนบรรทัดของผลลัพธ์มีจำนวนเท่ากับจำนวนปีที่เป็นข้อมูลนำเข้า โดยผลลัพธ์ในแต่ละบรรทัดมีสองค่า ได้แก่
\begin{enumerate}

\item ค่าแรกเป็นจำนวนของผึ้งงาน
\item ค่าที่สองเป็นจำนวนของผึ้งทั้งหมดในรัง
\end{enumerate}

โดยที่ระหว่างค่าแรกและค่าที่สองให้คั่นด้วยเว้นวรรคจำนวน 1 วรรค

\Examples

\begin{example}
\exmp{1  3  -1}{2  4
7  12}%
\end{example}

\Source

การแข่งขันคอมพิวเตอร์โอลิมปิก สอวน. ครั้งที่ 2 มหาวิทยาลัยบูรพา

\end{problem}

\end{document}