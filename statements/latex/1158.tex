\documentclass[11pt,a4paper]{article}

\usepackage{../../templates/style}

\begin{document}

\begin{problem}{หาทำเลตั้งศูนย์บริการลูกค้า}{standard input}{standard output}{1 second}{32 megabytes}

ร้านคอมพิวเตอร์ \textit{K.I.B.} ต้องการขยายฐานลูกค้าไปยังเมืองใหม่ โดยเมืองดังกล่าวมีการวางผังเมืองเป็นพื้นที่สี่เหลี่ยมย่อยจำนวน $M \times N$ พื้นที่ ($M$ แถว $N$ หลัก) และจากการสำรวจสำมะโนประชากรทำให้ทราบจำนวนประชากรในแต่ละพื้นที่ (ดูภาพประกอบด้านล่าง)

เนื่องจากร้าน \textit{K.I.B.} ต้องการเปิดศูนย์บริการลูกค้าเพียงร้านเดียวในเมืองนี้ ยิ่งไปกว่านั้นพื้นที่บริการที่ร้านให้บริการลูกค้าได้จะครอบคลุมบริเวณที่ประกอบด้วยสี่เหลี่ยมย่อยจำนวน $K \times K$ พื้นที่ ($K$ แถว $K$ หลัก) เท่านั้น ทางร้านจึงพยายามหาพื้นที่บริการที่ดีที่สุด ซึ่งในที่นี้หมายถึงพื้นที่บริการที่มีประชากรรวมกันมากที่สุด

\begin{center}
\begin{tabular}{|c|c|c|c|c|c|c|c|c|c|}
\hline
5 & 9 & 2 & 9 & 1 & 2 & 8 & 9 & 1 & 6\\
\hline
9 & 1 & 3 & 9 & 8 & 4 & 2 & 1 & 5 & 7\\
\hline
2 & 7 & 9 & 3 & 8 & 5 & 2 & 7 & 6 & 8\\
\hline
1 & 6 & 2 & 1 & \cellcolor{red!25}7 & \cellcolor{red!25}7 & 1 & 9 & 4 & 1\\
\hline
8 & 5 & 2 & 3 & \cellcolor{red!25}9 & \cellcolor{red!25}8 & 5 & 6 & 3 & 3\\
\hline
\end{tabular}

\textbf{ภาพประกอบตัวอย่างโจทย์} แสดงผลการหาทำเลตั้งศูนย์บริการลูกค้าในพื้นที่ขนาด $2\times 2$  $(K = 2)$\\ ของผังเมืองขนาด $5 \times 10$ ซึ่งในที่นี้ บริเวณที่ถูกเน้นคือพื้นที่บริการที่ดีที่สุด
\end{center}

\bigskip
\underline{\textbf{โจทย์}}  จงเขียนโปรแกรมที่มีประสิทธิภาพในการหาจำนวนประชากรรวมในทำเลพื้นที่บริการที่ดีที่สุด


\InputFile

\textbf{บรรทัดแรก} ระบุเลขจำนวนเต็มบวกสองตัวบอกจำนวนแถว $M$ และจำนวนหลัก $N$ ตามลำดับ โดยที่ $2 \leq M,N \leq 1\,000$

\textbf{บรรทัดที่สอง} ระบุขนาดพื้นที่บริการของร้าน $K$ โดยที่ $0 < K < M$ และ $0 < K < N$

\textbf{บรรทัดที่ $3$ ถึง $M+2$} ระบุจำนวนประชากรในแถวที่ $1$ ถึง $M$ ตามลำดับ ข้อมูลแต่ละบรรทัดประกอบด้วยตัวเลขจำนวนเต็มบวก $N$ จำนวน ซึ่งระบุจำนวนประชากรของพื้นที่สี่เหลี่ยมย่อย $N$ หลัก เรียงจากซ้ายไปขวาในแถวนั้น ๆ แต่ละจำนวนถูกคั่นด้วยช่องว่าง โดยประชากรในแต่ละพื้นที่สี่เหลี่ยมย่อยมีจำนวนไม่เกิน $2\,000$ คน


\OutputFile

\textbf{มีบรรทัดเดียว} ระบุจำนวนประชากรภายในพื้นที่บริการที่ดีที่สุด

\Examples

\begin{example}
\exmp{5 10
2
5 9 2 9 1 2 8 9 1 6
9 1 3 9 8 4 2 1 5 7
2 7 9 3 8 5 2 7 6 8
1 6 2 1 7 7 1 9 4 1
8 5 2 3 9 8 5 6 3 3}{31}%
\exmp{6 4
3
7 8 5 1
0 3 5 2
3 3 2 9
9 7 8 9
4 3 5 9
8 6 5 2}{55}%
\end{example}


\Source

การแข่งขันคอมพิวเตอร์โอลิมปิกระดับชาติครั้งที่ 8 (SUTOI8)

\end{problem}

\end{document}