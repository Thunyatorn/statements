\documentclass[11pt,a4paper]{article}

\usepackage{../../templates/style}

\begin{document}

\begin{problem}{รัฐบาลผสม (coalition)}{standard input}{standard output}{1 second}{32 megabytes}

ประเทศหนึ่งในแถบเอเชียตะวันออกเฉียงใต้ มีการปกครองในรูปแบบรัฐสภา โดย \textit{ส.ส.} ในสภาจะเป็นผู้ลงคะแนนเลือกนายกรัฐมนตรี ซึ่งผู้ที่จะได้รับเลือกเป็นนายกรัฐมนตรีนั้นจะต้องได้รับเสียงสนับสนุนจาก \textit{ส.ส.} \textbf{มากกว่าครึ่งหนึ่ง}ของจำนวน \textit{ส.ส.} ทั้งหมดในสภา จากนั้นนายกรัฐมนตรีก็จะเป็นผู้จัดตั้งคณะรัฐมนตรีขึ้นมาทำหน้าที่บริหารประเทศต่อไป

\textit{ส.ส.} แต่ละคนนั้นก็จะสังกัดพรรคการเมืองต่างๆ หากมีพรรคการเมืองใดที่มี \textit{ส.ส.} มากกว่าครึ่งหนึ่งของจำนวน \textit{ส.ส.} ทั้งหมดในสภา พรรคการเมืองนั้นก็จะเป็นผู้ครองเสียงข้างมากในสภา และจะสามารถจัดตั้งรัฐบาลพรรคเดียวขึ้นมาได้ แต่หากไม่มีพรรคการเมืองใดที่มีจำนวน \textit{ส.ส.} เกินครึ่ง ก็จะต้องมีการจับขั้วกันระหว่างพรรคการเมืองบางพรรค เพื่อให้จำนวน \textit{ส.ส.} ของพรรคการเมืองเหล่านั้นรวมกันแล้วมากกว่าครึ่งหนึ่งของจำนวน \textit{ส.ส.} ทั้งหมด ก็จะสามารถครองเสียงข้างมากในสภา และจัดตั้งรัฐบาลผสมขึ้นมาได้

ในการเลือกตั้ง \textit{ส.ส.} ครั้งล่าสุด ผลปรากฏว่าไม่มีพรรคการเมืองใดที่มีจำนวน \textit{ส.ส.} เกินครึ่ง ดังนั้น พรรคการเมืองแต่ละพรรคต่างก็พยายามที่จะไปจับขั้วกับพรรคการเมืองอื่น เพื่อจัดตั้งรัฐบาลผสมขึ้นมาให้ได้ หัวหน้าพรรคการเมืองแต่ละพรรคก็อยากจะทราบว่า พรรคการเมืองของตนจะต้องไปจับขั้วกับพรรคการเมืองอื่นอย่างน้อยกี่พรรค จึงจะทำให้จำนวน \textit{ส.ส.} ทั้งหมดของพรรคการเมืองของตนและของทุกพรรคที่ไปจับขั้วด้วย รวมกันแล้วมากกว่าครึ่งหนึ่งของจำนวน \textit{ส.ส.} ทั้งหมดในสภา

\bigskip
\underline{\textbf{โจทย์}}  จงเขียนโปรแกรมเพื่อรับจำนวน \textit{ส.ส.} ของพรรคการเมืองต่างๆ และคำนวณว่าพรรคการเมืองแต่ละพรรคจะต้องไปจับขั้วกับพรรคการเมืองอื่นอย่างน้อยกี่พรรค จึงจะสามารถครองเสียงข้างมากในสภาได้


\InputFile

\textbf{บรรทัดแรก} ระบุจำนวนเต็ม $N$ $(3 \leq N \leq 300\,000)$ แทนจำนวนพรรคการเมืองทั้งหมด

\textbf{บรรทัดที่ $2$ ถึง $N+1$} ในบรรทัดที่ $i+1$ $(1 \leq i \leq N)$ จะระบุจำนวนเต็ม $M_i$ $(1 \leq M_i \leq 1\,000\,000)$ แทนจำนวน \textit{ส.ส.} ของพรรคการเมืองที่ $i$

รับประกันว่าจะไม่มีพรรคการเมืองใดที่มีจำนวน \textit{ส.ส.} มากกว่าครึ่งหนึ่งของจำนวน \textit{ส.ส.} ทั้งหมดในสภา และจำนวน \textit{ส.ส.} ทั้งหมดจะมีไม่เกิน $2\,000\,000\,000 $ คน



\OutputFile

\textbf{มี $N$ บรรทัด} โดยในบรรทัดที่ $i$ $(1 \leq i \leq N)$ แสดงจำนวนพรรคการเมืองน้อยที่สุดที่พรรคการเมืองที่ $i$ ต้องไปจับขั้วด้วยเพื่อให้สามารถครองเสียงข้างมากในสภา


\Examples

\begin{example}
\exmp{5
7
1
1
2
3}{1
1
1
1
1}%
\exmp{7
5
5
6
7
5
5
8}{3
3
2
2
3
3
2}%
\end{example}

\Scoring

\textbf{$30$\% ของข้อมูลทดสอบ:} $N \leq 5\,000$
  
\Source

สุธี เรืองวิเศษ

การแข่งขัน IOI Thailand League เดือนมิถุนายน 2553


\end{problem}

\end{document}