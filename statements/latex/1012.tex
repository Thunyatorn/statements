\documentclass[11pt,a4paper]{article}

\usepackage{../../templates/style}

\begin{document}

\begin{problem}{Silly Bakery}{standard input}{standard output}{1 second}{64 megabytes}

ร้านสั่งทำเค้กแห่งหนึ่งแถวมหาวิทยาลัยขอนแก่น รับใบสั่งเค้กเฉพาะช่วงที่มีการแข่งขันคอมพิวเตอร์โอลิมปิกของสอวน.เท่านั้น เจ้าของร้านเค้กนี้จะผลิตแค่เค้กขนาด\textit{หนึ่งปอนด์ }แต่แบ่งขายเป็นห้าแบบคือ \textit{เต็มปอนด์} \textit{เศษสามส่วนสี่ปอนด์} \textit{ครึ่งปอนด์} \textit{หนึ่งส่วนสี่ปอนด์} และ \textit{หนึ่งส่วนแปดปอนด์} เผอิญว่าเจ้าของร้านเค้กเป็นนักคณิตศาสตร์ที่รักความสวยงามและความอร่อย ดังนั้นเวลาขายเค้กแต่ละครั้งเจ้าของร้านจะไม่ยอมเอาเค้กแบบที่เล็กกว่ามารวมให้ได้ขนาดของเค้กตามที่ลูกค้าต้องการ

สมมุติว่า ถ้าลูกค้าสั่งเค้กขนาด\textit{เศษสามส่วนสี่ปอนด์} เจ้าของร้านก็จะไม่นำเค้กขนาด\textit{หนึ่งส่วนสี่ปอนด์}ให้ลูกค้าไปสามก้อนแต่จะให้เค้กขนาด\textit{สามส่วนสี่ปอนด์}ที่มีอยู่แก่ลูกค้าเท่านั้น และถ้าไม่มีเค้กขนาด\textit{สามส่วนสี่ปอนด์}อยู่เลย เจ้าของร้านจะเอาเค้กขนาด\textit{เต็มปอนด์}มาแบ่งแล้วให้ลูกค้าไป โดยเก็บเศษที่เหลือไว้เผื่อให้กับลูกค้าคนอื่นที่อาจต้องการแบบ\textit{หนึ่งส่วนสี่ปอนด์} หรือไว้เพื่อแบ่งให้ลูกค้าที่ต้องการขนาด\textit{หนึ่งส่วนแปดปอนด์}

ด้วยความคุ้นเคยของลูกค้า ลูกค้าจะสั่งเค้กเป็นจำนวนเต็ม $a$ $b$ $c$ $d$ $e$ สำหรับก้อนของขนาดเค้ก\textit{เต็มปอนด์ เศษสามส่วนสี่ปอนด์ ครึ่งปอนด์ หนึ่งส่วนสี่ปอนด์} และ\textit{ หนึ่งส่วนแปดปอนด์} ตามลำดับ ในวันหนึ่งจะมีรายการสั่งของทั้งหมดจากลูกค้า $n$ ราย เจ้าของร้านจะรับรายการสั่งของวันนี้เพื่อคำนวณว่าจะต้องทำเค้กทั้งหมดกี่ปอนด์เพื่อให้เพียงพอในการส่งของให้ลูกค้าในวันถัดไป

\underline{\textbf{โจทย์}}  จงเขียนโปรแกรมเพื่อรับข้อมูลรายการสั่งเค้กของลูกค้าและคำนวณว่าจะต้องทำเค้กอย่างน้อยที่สุดกี่ปอนด์

\InputFile

\textbf{บรรทัดแรก} รับค่าจำนวนเต็ม $n$ $( 1\leq  n \leq 10)$

\textbf{บรรทัดที่ $2$ ถึง $n+1$} บรรทัดที่ $i+1$ จะรับข้อมูลของลูกค้าคนที่ $i$ แต่ละบรรทัดจะประกอบด้วยจำนวนเต็ม $a$ $b$ $c$ $d$ $e$  $(0 \leq a,b,c,d,e \leq 10\,000)$ โดยแต่ละค่าจะคั่นด้วยช่องว่างหนึ่งช่อง

\OutputFile

\textbf{มีบรรทัดเดียว} ประกอบด้วยจำนวนเต็มหนึ่งค่า ซึ่งเป็นจำนวนของเค้กเต็มปอนด์ที่น้อยที่สุดที่เจ้าของร้านต้องเตรียมให้เพียงพอตามรายการที่ลูกค้าสั่ง

\Examples

\begin{example}
\exmp{3
1 0 1 0 1
0 1 0 1 0
0 1 0 0 0}{4}%
\exmp{4
0 0 1 0 0
0 0 0 0 1
0 0 1 0 0
0 1 0 0 0}{2}%
\end{example}

\Source

การแข่งขันคอมพิวเตอร์โอลิมปิก สอวน. ครั้งที่ 3 มหาวิทยาลัยขอนแก่น

\end{problem}

\end{document}