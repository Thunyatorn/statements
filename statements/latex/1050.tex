\documentclass[11pt,a4paper]{article}

\usepackage{../../templates/style}

\begin{document}

\begin{problem}{Chemical Reactant}{standard input}{standard output}{1 second}{16 megabytes}

ศูนย์วิจัยแห่งหนึ่งสร้างสารเคมีด้วยการผสมสารตั้งต้นสองชนิดเข้าด้วยกัน  สารตั้งต้นแต่ละตัวมีคุณสมบัติอยู่ $P$ ชนิด โดยเรียกเป็นคุณสมบัติที่ $1$ ถึงคุณสมบัติที่ $P$ ในการผสมสารตั้งต้นสองชนิดเข้าด้วยกันนั้นจะให้สารผลลัพธ์ที่มีคุณสมบัติที่ $i$ เท่ากับผลรวมของคุณสมบัติที่ $i$ ของสารตั้งต้นสองชนิด ตัวอย่างเช่น ถ้าสารตั้งต้นที่หนึ่งมีคุณสมบัติที่ $3$ เท่ากับ $5$ และสารตั้งต้นที่สองมีคุณสมบัติที่ $3$ เท่ากับ $7$ สารผลลัพธ์ที่ได้จะมีคุณสมบัติที่ $3$ เท่ากับ $12$

นักวิจัยคนหนึ่งต้องการสารที่มีคุณสมบัติตามต้องการ  แต่ในโกดังเก็บสารตั้งต้นมีสารอยู่ถึง $N$ ตัว ให้คุณเขียนโปรแกรมเพื่อหาว่าในโกดังนั้นมีสารที่เขาต้องการหรือไม่ หรือถ้าไม่มี สามารถสร้างสารนั้นได้จากการผสมสารสองชนิด (ที่เป็นคนละสารกัน) ในโกดังได้หรือไม่

\bigskip
\underline{\textbf{โจทย์}}  จงเขียนโปรแกรมรับข้อมูลสารเคมีในโกดังทั้งหมดและสารเคมีที่ต้องการ แล้วแสดงว่าสามารถสร้างสารเคมีชนิดนั้นตามเงื่อนไขได้หรือไม่

\InputFile

\textbf{บรรทัดแรก} มีจำนวนเต็มสองจำนวน $N$  $P$  $(1 \leq N \leq 100\,000; 1 \leq P \leq 10)$  

\textbf{บรรทัดที่ $2$ ถึง $N+1$ }จะเป็นข้อมูลของสารตั้งต้นที่มีในโกดัง  กล่าวคือในบรรทัดที่ $i+1$ จะเป็นข้อมูลของสารที่ $i$ ซึ่งประกอบไปด้วยจำนวนเต็มบวก $P$ จำนวน, $a_1$ $a_2$ $a_3$ $...$ $a_P$ โดย $a_j$ ระบุคุณสมบัติที่ $j$ ของสารที่ $i$ ข้อมูลรับประกันว่าไม่มีสารตั้งต้นสองสารใด ๆที่มีคุณสมบัติทั้ง $P$ ชนิดเท่ากันทั้งหมด   

\textbf{บรรทัดที่ $N+2$ } ระบุคุณสมบัติของสารที่ต้องการ โดยระบุเป็นจำนวนเต็ม $P$ จำนวน

\OutputFile

\textbf{มีบรรทัดเดียว} ให้แสดงผลตามเงื่อนไขต่อไปนี้
\begin{itemize}

\item ถ้ามีสารที่มีคุณสมบัติตรงตามที่ต้องการ (โดยไม่ต้องผสม) ให้พิมพ์หมายเลขของสารนั้นเลย
\item ถ้าไม่มีสารที่มีคุณสมบัติตามต้องการ แต่สามารถสร้างได้จากการผสมสารตั้งต้นสองชนิด (ที่ไม่ซ้ำกัน) ให้พิมพ์หมายเลขของสารทั้งสองออกมา โดยพิมพ์หมายเลขของสารที่มีหมายเลขน้อยกว่าก่อน  
\item ถ้าไม่สามารถสร้างได้ให้ตอบ NO
\end{itemize}
'
\Examples

\begin{example}
\exmp{3 2
1 2
3 4
5 6
1 2}{1}%
\exmp{3 2
1 2
3 4
5 6
6 8}{1 3}%
\exmp{3 2
1 2
3 4
5 6
16 18}{NO}%
\end{example}

\Scoring

\textbf{20\% ของข้อมูลชุดทดสอบ}: $N \leq 1\,000$

\Source

สอบปฏิบัติครั้งที่ 1 ค่ายคัดเลือกผู้แทนประเทศไทยไปแข่งขันคอมพิวเตอร์โอลิมปิกระหว่างประเทศปี 2550 ค่ายที่ 1

\end{problem}

\end{document}