\documentclass[11pt,a4paper]{article}

\usepackage{../../templates/style}

\begin{document}

\begin{problem}{รวมอนุภาค MAX (atom\_max)}{standard input}{standard output}{0.4 second}{64 megabytes}

อนุภาคแบบสั่งทำพิเศษจำนวน $N$ อนุภาควางเรียงกัน (โดยที่ $N$ เป็นจำนวนเต็มคู่) เราจะเรียกอนุภาคดังกล่าวว่าอนุภาคที่ $1, 2, ...,$ ถึง อนุภาคที่ $N$ ตามลำดับ อนุภาคแต่ละอนุภาคจะมีค่าพลังงานสะสมอยู่ กล่าวคืออนุภาคที่ $i$ จะมีพลังงานสะสมเท่ากับ $X_i$ หน่วย

            อนุภาคสองอนุภาคใด ๆ \textbf{ที่อยู่ติดกัน}เมื่อนำมาชนกันจะสลายตัวและปล่อยพลังงานออกมา โดย\textbf{พลังงานที่ปล่อยออกมานั้น มีค่าเท่ากับผลต่างของพลังงานสะสมของอนุภาคทั้งสอง} สังเกตว่าเมื่ออนุภาคชนกันแล้วจะสลายไปทั้งคู่ ทำให้อนุภาคคู่อื่น ๆ ที่เมื่อเริ่มต้นไม่ได้มีตำแหน่งติดกัน มีลำดับอยู่ติดกันได้

ตัวอย่างการดำเนินการเป็นดังนี้ สมมติมีอนุภาค $6$ อนุภาคที่มีพลังงานสะสมดังนี้: $1$ $2$ $4$ $3$ $1$ $2$

คุณเลือกชนอนุภาคที่ $2$ กับ $3$ ได้พลังงาน $2$ หน่วย หลังจากนั้นเราจะเหลืออนุภาค $4$ อนุภาค: $1$ $3$ $1$ $2$

เลือกคู่อนุภาค $1$ กับอนุภาค $4$ ได้พลังงาน $2$ หน่วย หลังจากนั้นเราจะเหลืออนุภาค $2$ อนุภาค: $1$ $2$

เลือกคู่อนุภาค $5$ กับอนุภาค $6$ ได้พลังงาน $1$ หน่วย รวมแล้วได้พลังงานทั้งหมด $5$ หน่วย

หัวหน้าห้องปฏิบัติการวานให้คุณหาวิธีนำอนุภาคทั้ง $N$ อันมาชนกัน โดยให้คุณหาวิธีการชนที่ทำให้พลังงานรวมสุดท้ายมากที่สุด

\bigskip
\underline{\textbf{โจทย์}}  รับข้อมูลพลังงานสะสมของอนุภาค จากนั้นคำนวณหาพลังงานรวมสูงสุดที่สามารถทำได้จากการชนอนุภาค

\InputFile

\textbf{บรรทัดแรก} ระบุจำนวนเต็ม $N$ $(1 \leq N \leq 1\,000\,000)$ แทนจำนวนอนุภาค

\textbf{บรรทัดที่ $2$ ถึง $N+1$} ระบุพลังงานสะสมของแต่ละอนุภาค กล่าวคือ บรรทัดที่ $i+1$ จะระบุจำนวนเต็ม $X_i$ $(1 \leq X_i ≤ 1\,000\,000\,000)$ แทนพลังงานสะสมของอนุภาคที่ $i$


\OutputFile

\textbf{มีบรรทัดเดียว} คือพลังงานรวมทั้งหมดที่ได้รับ

\Examples

\begin{example}
\exmp{6
1
2
4
3
1
2}{5}%
\exmp{4
20
17
15
12}{10}%
\end{example}

\Scoring

\textbf{50\% ของชุดข้อมูลทดสอบ}: $1 \leq N \leq 300$

\textbf{80\% ของชุดข้อมูลทดสอบ}: $1 \leq N \leq 100\,000$

\textbf{100\% ของชุดข้อมูลทดสอบ}: $1 \leq N \leq 1\,000\,000$

\Source

โจทย์ข้อสอบสสวท. ค่ายที่ 2 ระยะที่ 1 ปี 2552 โดย ดร.จิตร์ทัศน์ ฝักเจริญผล และแนวคิดของพศิน มนูรังษี

\end{problem}

\end{document}