\documentclass[11pt,a4paper]{article}

\usepackage{../../templates/style}

\begin{document}

\begin{problem}{Slikar}{standard input}{standard output}{1 second}{32  megabytes}

จักรพรรดิ์แคกตัสผู้ชั่วร้ายครอบครองเข่งวิเศษอยู่ และได้ใช้เข่งวิเศษเทน้ำอย่างไม่หมดสิ้นให้ท่วมป่าอาถรรพณ์ ช่างทาสีและสามสหายเม่นน้อยซึ่งเดินทางอยู่ในป่า จะต้องรีบเดินทางไปยังรังของตัวบีเวอร์เพื่อให้ปลอดภัยจากน้ำที่กำลังจะท่วม

เราแทนแผนผังของป่าอาถรรพณ์ด้วยเมตริกซ์ขนาด $R$ แถวและ $C$ คอลัมน์ โดยช่องที่ว่างแทนด้วยตัวอักษร \textbf{‘.’} ช่องที่ถูกน้ำท่วมแทนด้วยตัวอักษร \textbf{‘*’} และช่องที่เป็นหินแทนด้วยตัวอักษร ‘\textbf{X}’ สำหรับช่องที่เป็นรังของบีเวอร์แทนด้วยตัวอักษร ‘\textbf{D}’ และ ช่องที่ช่างทาสีและ hedgehogs อยู่จะแทนด้วยตัวอักษร ‘\textbf{S}’

ในแต่ละนาทีที่ผ่านไป ช่างทาสีและสามสหายเม่นน้อยสามารถเดินทางไปได้เพียง $1$ ช่อง โดยเลือกจากช่องที่อยู่ถัดไปทางด้านบน ล่าง ซ้าย หรือขวาเท่านั้น และทุก ๆนาทีน้ำที่ถูกเทออกมาจะไหลไปท่วมช่องที่อยู่ใกล้เคียงได้เพิ่มขึ้น $1$ ช่องในทุก ๆด้าน สำหรับช่องที่เป็นหินนั้น ไม่ว่าน้ำหรือช่างทาสีและสามสหายเม่นน้อยก็จะไม่สามารถผ่านไปได้ และถ้าน้ำท่วมช่องใดแล้ว ช่างทาสีและสามสหายแม่นน้อยก็จะไม่สามารถข้ามหรือเข้าไปในช่องนั้นได้เช่นกัน  อย่างไรก็ตามน้ำจะไม่ท่วมรังของบีเวอร์

ในการเลือกช่องเดินของช่างทาสีและสามสหายเม่นน้อยมีข้อจำกัดอยู่ว่า จะต้องไม่เลือกเดินไปในช่องที่น้ำกำลังจะท่วมเข้ามาในนาทีเดียวกันพอดี

\bigskip
\underline{\textbf{โจทย์}}  จงเขียนโปรแกรมเพื่อรับอินพุตเป็นแผนผังของป่าอาถรรพณ์ เพื่อคำนวณหาเวลาเป็นนาทีที่น้อยที่สุดที่ช่างทาสีและสามสหายเมนน้อยจะเดินทางไปถึงรังของบีเวอร์อย่างปลอดภัย

\InputFile

\textbf{บรรทัดแรก} รับตัวเลขจำนวนเต็มสองจำนวน $R$ $C$ แทนจำนวนแถวและจำนวนคอลัมน์ของป่าอาถรรพณ์ โดยที่ $ 1 \leq R,C \leq 50$

\textbf{บรรทัดที่ $2$ ถึง $R+1$ }บรรทัดที่ $i+1$ ให้รับข้อมูลป่าอาถรรพ์ในแถวที่ $i$ โดยแต่ละบรรทัดมี $C$ ตัวอักษร (รับตัวอักษร ‘.’, ‘*’, ‘X’, ‘D’ หรือ ‘S’ ติดกันไปโดยไม่มีช่องว่างคั่น และทั้งป่าอาถรรพณ์จะมี ‘D’ และ ‘S’ ได้เพียงอย่างละหนึ่งตัวอักษรเท่านั้น


\OutputFile

\textbf{มีบรรทัดเดียว} แสดงตัวเลขเวลาที่สั้นที่สุดที่เป็นไปได้สำหรับช่างทาสีและสามสหายเม่นน้อยใช้ในการเดินทางไปจนถึงรังของบีเวอร์ ถ้าไม่สามารถเดินทางไปถึงได้ ให้พิมพ์คำว่า ‘KAKTUS’


\Examples

\begin{example}
\exmp{3 3
D.*
...
.S.}{3}%
\exmp{3 3
D.*
...
..S}{KAKTUS}%
\exmp{3 6
D...*.
.X.X..
....S.}{6}%
\end{example}


\Source

Croatian Open Competition in Informatics
Contest 1 – October 28, 2006

\end{problem}

\end{document}