\documentclass[11pt,a4paper]{article}

\usepackage{../../templates/style}

\begin{document}

\begin{problem}{Sudoku}{standard input}{standard output}{1 second}{64 megabytes}

\textbf{สุโดกุ \textit{(sudoku)}} เป็นเกมคณิตศาสตร์เกมหนึ่ง โดยกำหนดให้มีตารางจัตุรัสขนาด $3 \times 3$ ตารางหน่วยอยู่ $9$ ตาราง และตาราง $3 \times 3$ หน่วยทั้ง $9$ ตารางนี้ ได้ถูกนำมาเรียงต่อกันเป็นตารางจัตุรัสขนาด $9 \times 9$ ตารางหน่วย เป้าหมายของการเล่นสุโดกุคือการหาจำนวนเต็ม \textbf{1-9 }มาใส่ในตาราง $9 \times 9$ โดยที่ในแต่ละคอลัมน์ แต่ละแถว และแต่ละตารางจัตุรัสย่อย $3 \times 3$ นั้นไม่มีตัวเลขซ้ำกันเลย

ความยากง่ายของเกมสุโดกุจะอยู่ที่จำนวนเต็มในบางช่องของตารางที่บังคับไว้ก่อนการเริ่มเล่น เพื่อจำกัดอิสระในการวางตัวเลขที่เหลือ ซึ่งเป็นส่วนสำคัญในการสร้างทักษะทางคณิตศาสตร์ให้กับผู้เล่น จากตัวอย่างต่อไปนี้ ตารางซ้ายคือโจทย์สุโดกุข้อหนึ่งพร้อมช่องบังคับเริ่มต้น และตารางขวาคือเป็นตัวอย่างของคำตอบที่ถูกต้องแบบหนึ่ง ทั้งนี้คำตอบที่ถูกอาจมีรูปแบบมากกว่าหนึ่งแบบก็ได้
\bigskip

\textbf{ตัวอย่าง}
\begin{table}[!htb]
\begin{minipage}{.5\linewidth}
\caption{ตารางเริ่มต้น}
      \centering
\begin{tabular}{ |c|c|c| } 
\multicolumn{3}{c}{}\\
\hline
\rule{0pt}{3ex}  
5  3 \_ & \_  7 \_ & \_  \_  \_\\[0.5ex]
 6  \_  \_ & 1 9 5 & \_  \_  \_ \\[0.5ex]
 \_  9 8 & \_  \_  \_ & \_  6  \_\\[0.75ex]
\hline
\rule{0pt}{3ex}
 8  \_  \_ & \_  6  \_ & \_  \_  3\\[0.5ex]
 4  \_  \_ & 8  \_  3 & \_  \_  1\\[0.5ex]
 7  \_  \_ & \_  2  \_ & \_  \_  6\\[0.75ex]
\hline
\rule{0pt}{3ex} 
 \_  6  \_ & \_  \_  \_ & 2  8  \_\\[0.5ex]  
 \_  \_  \_ & 4  1  9 & \_  \_  5\\[0.5ex]
 \_  \_  \_ & \_  8  \_ & \_  7 9\\[0.75ex]
 \hline
\end{tabular}
\end{minipage}%
\begin{minipage}{.3\linewidth}
\caption{ตารางที่ถูกต้อง}
      \centering
\begin{tabular}{ |c|c|c| } 
\multicolumn{3}{c}{}\\
\hline
\rule{0pt}{3ex}  
 5 3 4 & 6 7 8 & 9 1 2\\[0.5ex]
  6 7 2 & 1 9 5 & 3 4 8\\[0.5ex]
  1 9 8 & 3 4 2 & 5 6 7\\[0.75ex]
\hline
\rule{0pt}{3ex}  
  8 5 9 & 7 6 1 & 4 2 3\\[0.5ex]
  4 2 6 & 8 5 3 & 7 9 1\\[0.5ex]
  7 1 3 & 9 2 4 & 8 5 6\\[0.75ex]
\hline
\rule{0pt}{3ex}  
  9 6 1 & 5 3 7 & 2 8 4\\[0.5ex]
  2 8 7 & 4 1 9 & 6 3 5\\[0.5ex]
  3 4 5 & 2 8 6 & 1 7 9\\[0.75ex]
  \hline
\end{tabular}
\end{minipage}
\end{table}

การเล่นสุโดกุนี้ได้รับความนิยมเพิ่มขึ้นเรื่อยๆจนกระทั่งมีการจัดการแข่งขันสุโดกุขึ้นมา เนื่องจากในการแข่งขันมีผู้เข้าแข่งขันจำนวนมาก การตรวจสอบว่าตารางนั้นถูกต้องหรือไม่นั้นเป็นงานที่ใช้เวลามาก ทำให้มีความจำเป็นที่จะต้องอาศัยโปรแกรมคอมพิวเตอร์ช่วยตรวจ



\bigskip
\underline{\textbf{โจทย์}}  จงเขียนโปรแกรมเพื่อตรวจสอบว่าตารางสุโดกุที่ส่งเข้ามาตรวจในระหว่างการแข่งขันนั้น มีผู้เข้าแข่งขันคนใดทำถูกต้องบ้าง

\InputFile

\textbf{บรรทัดแรก} รับจำนวนเต็ม $n$ $(0 < n \leq 100)$ ซึ่งเป็นจำนวนผู้เข้าแข่งขัน

\textbf{บรรทัดที่สองถึงสิบ} ในบรรทัดที่ $i+1$ ให้รับข้อมูลแถวที่ $i$ ของตารางโจทย์สุโดกุเริ่มต้นที่ผู้เข้าแข่งขันทุกคนจะต้องทำเหมือนกัน ซึ่งแต่ละแถวประกอบด้วยจำนวนเต็ม $x$ $(0 \leq x \leq 9)$ อยู่ $9$ ตัวคั่นด้วยช่องว่าง จำนวนเต็มแต่ละตัวคือค่าที่อยู่ในตารางสุโดกุ โดยค่าเป็น $0$ หมายถึงค่าเริ่มต้นนั้นเป็นช่องว่าง

\textbf{บรรทัดที่ $11$ ถึง $9n+10$} จะเป็นตารางสุโดกุของผู้เข้าแข่งขันคนที่ $1$ ไปถึงคนที่ $n$ ดังนั้นข้อมูลของผู้เข้าแข่งขันคนที่ $i$ จึงเริ่มต้นที่บรรทัด $9i+2$ ไปจนถึงบรรทัดที่ $9i+10$ ซึ่งแต่ละแถวประกอบด้วยจำนวนเต็ม $y$ $(1 \leq y \leq 9)$ อยู่ $9$ ตัวคั่นด้วยช่องว่าง จำนวนเต็มแต่ละตัวคือค่าที่อยู่ในตารางสุโดกุของผู้เข้าแข่งขันคนที่ $i$

\OutputFile

\textbf{มีหลายบรรทัด} ข้อมูลส่งออกจะมีหมายเลขของผู้เข้าแข่งขันที่ตอบได้ถูกต้องบรรทัดละหนึ่งหมายเลข บรรทัดหนึ่ง ๆจะแสดงจำนวนเต็ม $i$ หากว่าตารางสุโดกุของผู้เข้าแข่งขันคนที่ $i$ เป็นคำตอบที่ถูกต้องตามโจทย์และเงื่อนไข บรรทัดสุดท้ายให้แสดงคำว่า 'END'

\Examples

\begin{example}
\exmp{2
5 3 0 0 7 0 0 0 0
6 0 0 1 9 5 0 0 0
0 9 8 0 0 0 0 6 0
8 0 0 0 6 0 0 0 3
4 0 0 8 0 3 0 0 1
7 0 0 0 2 0 0 0 6
0 6 0 0 0 0 2 8 0
0 0 0 4 1 9 0 0 5
0 0 0 0 8 0 0 7 9
1 1 1 6 7 8 9 1 2
1 1 1 1 9 5 3 4 8
1 1 1 1 4 2 5 6 7
8 5 9 7 6 1 4 2 3
4 2 6 8 5 3 7 9 1
7 1 3 9 2 4 8 5 6
9 6 1 5 3 7 2 8 4
2 8 7 4 1 9 6 3 5
3 4 5 2 8 6 1 7 9
5 3 4 6 7 8 9 1 2
6 7 2 1 9 5 3 4 8
1 9 8 3 4 2 5 6 7
8 5 9 7 6 1 4 2 3
4 2 6 8 5 3 7 9 1
7 1 3 9 2 4 8 5 6
9 6 1 5 3 7 2 8 4
2 8 7 4 1 9 6 3 5
3 4 5 2 8 6 1 7 9}{2
END}%
\end{example}


\Source

การสอบแข่งขันคณิตศาสตร์และวิทยาศาสตร์โอลิมปิกแห่งประเทศไทย
ประจำปี พ.ศ.2549 (สอบแข่งขันรอบที่ 2 ภาคปฏิบัติวันที่ 1)

\end{problem}

\end{document}