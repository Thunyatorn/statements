\documentclass[11pt,a4paper]{article}

\usepackage{../../templates/style}

\begin{document}

\begin{problem}{ซื้อบ้าน (house)}{standard input}{standard output}{1 second}{16 megabytes}

บริษัททีโอไอจำกัด เป็นบริษัทที่ทำการขายบ้านในที่ต่าง ๆ และในที่แห่งหนึ่งซึ่งคุณเป็นผู้ดูแลนั้น มีบ้านที่ถูกสร้างวางเป็นช่องตารางจำนวน $R \times C$ หลัง บ้านแต่ละหลังจะมีดัชนีค่าความสุขอยู่ค่าหนึ่งเป็นจำนวนเต็มที่ไม่เป็นลบ

    เมื่อมีคนมาอาศัยอยู่ในบ้านหลังหนึ่ง ค่าความสุขแท้จริงที่คนนั้นได้รับ คือผลรวมของ ดัชนีความสุข$\times ((-1)$ ยกกำลังด้วย ระยะทางที่ห่างออกไป$)$ ของบ้านแต่ละหลังที่อยู่ในแถว (\textit{Row}) เดียวกันหรือหลัก (\textit{Column}) เดียวกันและอยู่ห่างจากบ้านนั้นไม่เกิน $K$ หลัง โดยในการคำนวณให้พิจารณารวมบ้านหลังนั้นเองด้วยกล่าวคือนั่นคือบ้านที่อยู่ห่างจากหลังนั้นเป็นระยะทางจำนวนคู่ จะส่งผลต่อค่าความสุขแท้จริงในผลบวก แต่บ้านที่อยู่ห่างจากหลังนั้นเป็นระยะทางจำนวนคี่ จะส่งผลต่อค่าความสุขแท้จริงในผลลบ

    ในวันหนึ่ง มีผู้มีพระคุณมาขอซื้อบ้านหนึ่งหลัง และให้คุณเป็นคนเลือกให้ท่าน คุณอยากให้ท่านผู้นั้นได้อาศัยอยู่อย่างมีความสุขมากที่สุด ถามว่าบ้านหลังที่มีค่าความสุขรวมมากที่สุด มีค่าความสุขรวมเท่าไร

\bigskip
\underline{\textbf{โจทย์}}  เขียนโปรแกรมขนาดของพื้นที่แห่งนั้น และค่าความสุขของบ้านแต่ละหลัง แล้วตอบค่าความสุขรวมที่มากที่สุดของบ้านหลังหนึ่งในพื้นที่แห่งนั้น

\InputFile

\textbf{บรรทัดแรก} มีจำนวนเต็มบวกสามจำนวน $R, C$ และ $K$ $(1 \leq R,C,K \leq 300)$

\textbf{บรรทัดที่ $2$ ถึง $R+1$} จะมีข้อมูลชองดัชนีความสุขของบ้านในแต่ละหลัง บรรทัดละ $C$ จำนวน โดยจำนวนลำดับที่ $j$ ในบรรทัดที่ $i+1$ สำหรับ $1 \leq i \leq R$  จะมีจำนวน $V_ij$ ซึ่งเป็นค่าความสุขของบ้านแถวที่ $i$ หลักที่ $j$ โดยที่ $0 \leq V_ij \leq 10\,000$


\OutputFile

\textbf{มีบรรทัดเดียว} มีจำนวนเต็มหนึ่งจำนวน บอกค่าความสุขรวมที่มากที่สุดที่เป็นไปได้

\Examples

\begin{example}
\exmp{5 4 2
2 2 3 5
3 2 4 0
5 3 1 2
0 2 0 1
3 2 1 5}{13}%
\end{example}

\Note 

\textbf{อธิบายข้อมูลน้ำเข้าและส่งออก}

      มีบ้าน $5$ แถวๆละ $4$ หลัง ระยะทางที่ใช้คำนวณความสุขรวมคือ $2$ หลัง และในบ้านในแถวที่ $3$ หลักที่ $4$ จะทำได้ค่าความสุขรวมเท่ากับ $2 + 5 + 3 + 5 - 0 - 1 - 1 = 13$ ดังตัวอย่างด้านล่าง

\begin{center}
\begin{tabular}{cccc}
0   & 0  &  0 &  +5\\
0   & 0 &   0  & -0\\
0  & +3 &  -1 & [+2]\\
0  & 0 &   0 &  -1\\
0  &  0 &   0 &  +5\\
\end{tabular}
\end{center}
\Source

วิสิฐ ภัทรนุธาพร

\underline{\href{http://www.thailandoi.org/toi.c/04-2009}{TOI.C:04-2009}}

\end{problem}

\end{document}