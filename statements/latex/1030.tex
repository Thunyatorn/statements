\documentclass[11pt,a4paper]{article}

\usepackage{../../templates/style}

\begin{document}

\begin{problem}{Rice}{standard input}{standard output}{1 second}{128 megabytes}

โรงสีแห่งหนึ่งมีข้าวจำหน่ายอยู่หลายชนิด แต่ละชนิดในปริมาณและราคาแตกต่างกันไป โดยโรงสีจำหน่ายข้าวแยกเป็นกิโลกรัม วันหนึ่งมีพ่อค้าข้าวหลายรายมาเข้าคิวซื้อข้าว โดยพ่อค้าแต่ละคนจะซื้อข้าวในปริมาณเป็นกิโลกรัมต่างกันไป พ่อค้าทุกคนจะพยายามซื้อข้าวให้ได้ราคาถูกที่สุดเท่าที่จะทำได้

\bigskip
\underline{\textbf{โจทย์}}  กำหนดจำนวนชนิดข้าว ราคาเป็นบาทและปริมาณเป็นกิโลกรัมของข้าวแต่ละชนิด จำนวนพ่อค้าในคิว และปริมาณข้าวที่พ่อค้าแต่ละคนจะซื้อ จงเขียนโปรแกรมคำนวณเงินที่ต่ำที่สุดที่พ่อค้าแต่ละคนจะต้องจ่าย

\InputFile

\textbf{บรรทัดแรก} รับค่าจำนวนเต็ม $K$ $(1 \leq K \leq 100\,000)$ แทนจำนวนชนิดของข้าว

\textbf{บรรทัดที่ $2$ ถึง $K+1$} บรรทัดที่ $i+1$ รับค่าจำนวนเต็ม $P$ $(1 \leq P \leq 1\,000\,000)$ และ $Q$ $(1 \leq Q \leq  1\,000\,000)$ หมายความว่าข้าวชนิดที่ $i$ มีปริมาณ $Q$ กิโลกรัม และข้าว $Q$ กิโลกริมนี้มีราคา $P$ บาท ดังนั้นข้าวแต่ละกิโลกรัมของข้าวชนิดนี้มีราคา $\frac{P}{Q}$ บาท

\textbf{บรรทัดที่ $K+2$} รับจำนวนเต็ม $M$ $(1 \leq M \leq 100\,000)$ แทนจำนวนพ่อค้าในคิว

\textbf{บรรทัดที่ $K+3$ ถึง $K+M+2$} บรรทัดที่ $K+i+2$ ให้รับจำนวนเต็ม $B$ $(1 \leq B \leq 1\,000\,000)$ แสดงจำนวนข้าวเป็นกิโลกรัมที่พ่อค้าคนที่ $i$ จะซื้อ จำนวนกิโลกรัมเหล่านี้จะให้มาตามลำดับของพ่อค้าที่อยู่ในคิว กล่าวคือจำนวนกิโลกรัมแรกเป็นของพ่อค้าคนแรกที่จะได้ซื้อข้าว จำนวนกิโลกรัมที่สองเป็นของพ่อค้าคนที่สองที่จะได้ซื้อข้าว เช่นนี้ไปเรื่อยๆ

\OutputFile

\textbf{มี M บรรทัด} แต่ละบรรทัดแสดงจำนวนจริง $X$ ซึ่งมีความละเอียดถึงทศนิยมตำแหน่งที่สาม แสดงเงินที่ต่ำที่สุดที่พ่อค้าคนหนึ่งจะต้องจ่าย โดยจำนวน $X$ ในบรรทัดที่ $i$ มีค่าเท่ากับเงินที่พ่อค้าในคิวคนที่ $i$ จะต้องจ่าย เรารับประกันว่าโรงสีมีข้าวมากพอให้พ่อค้าทุกคนซื้อได้

\Examples

\begin{example}
\exmp{5
10 10
5 10
2000 100
5 5
2 5
5
3
5
5
20
10}{1.200
2.300
2.500
76.000
200.000}%
\end{example}


\Source

Young Thai Online Programming Competition 2008

\end{problem}

\end{document}