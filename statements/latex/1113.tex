\documentclass[11pt,a4paper]{article}

\usepackage{../../templates/style}

\begin{document}

\begin{problem}{อ้าโอ้ (RO)}{standard input}{standard output}{2 second}{128 megabytes}

คุณเล่นเกม กับเพื่อนของคุณ โดยเกมของคุณนั้นมีกฏอันแสน ปญอ (ปัญญาอ่อน) คือเพื่อนของคุณจะร้องคำว่า \textit{“อ้า”} และ \textit{“โอ้”} ไปเรื่อยๆ แล้วคุณจะต้องตอบให้ได้ว่า ช่วงติดกันที่ยาวที่สุดที่มีจำนวนคำว่า \textit{“โอ้”} มีจำนวนเป็น $k$ เท่าของจำนวนคำว่า \textit{“อ้า”} มีความยาวเท่าใด
\bigskip

            \textbf{ตัวอย่างประกอบ}

สมมติว่าเพื่อนของคุณร้องว่า\\
\begin{center}
\textit{“อ้า”} \textit{“โอ้”} \textit{“อ้า”} \textit{“อ้า”} \textit{“โอ้”} \textit{“โอ้”} \textit{“อ้า”} \textit{“โอ้”} \textit{“โอ้”} \textit{“อ้า”} \textit{“โอ้”}  \textit{“โอ้”}  \textit{“โอ้”} \textit{“อ้า”} \textit{“โอ้”} 
\end{center}
\\และ $k = 3$

            คุณจะพบว่าช่วง \textit{“โอ้”} \textit{“โอ้”} \textit{“อ้า”} \textit{“โอ้”} \textit{“โอ้”} \textit{“อ้า”} \textit{“โอ้”}  \textit{“โอ้”} มีจำนวนคำว่า \textit{“โอ้”} $6$ ครั้งและ \textit{“อ้า”} 2 ครั้ง ซึ่งถูกต้องตามหลักเกณฑ์และเป็นช่วงที่ยาวที่สุด ดังนั้นคำตอบของกรณีนี้จึงเป็น $8$ เพราะช่วง \textit{“โอ้”} \textit{“โอ้”} \textit{“อ้า”} \textit{“โอ้”} \textit{“โอ้”} \textit{“อ้า”} \textit{“โอ้”}  \textit{“โอ้”}  มีความยาว $8$ คำ 

\bigskip
\underline{\textbf{โจทย์}}  จงเขียนโปรแกรมหาช่วงที่ยาวที่สุดที่คำว่า \textit{"โอ้"} มีจำนวนวเป็น $k$ เท่าของคำว่า \textit{"อ้า"}


\InputFile

\textbf{บรรทัดแรก} ประกอบด้วยจำนวนนับ $n$ และ $k$ $( 1 \leq n \leq 1\,000\,000; 2 \leq k \leq n )$

\textbf{บรรทัดที่สอง} ประกอบด้วยสตริงยาว $n$ โดยตัวอักษรแต่ละตัวจะแสดงคำที่เพื่อนของคุณพูดตามลำดับ ซึ่ง ‘R’ จะแทนคำว่า “อ้า” และ ‘O’ จะแทนคำว่า “โอ้”



\OutputFile

\textbf{มีบรรทัดเดียว} แสดงความยาวของช่วงที่ติดกันที่ยาวที่สุดที่มีจำนวนคำว่า “โอ้” มีจำนวนเป็น $k$ เท่าของจำนวนคำว่า “อ้า”

\Examples

\begin{example}
\exmp{15 3
RORROOROOROOORO}{8}%
\exmp{17 3
OROOOOOROOOOORRRR}{12}%
\end{example}

\Scoring

\textbf{มีชุดทดสอบทั้งสิ้น $26$ ชุด}โดย
\begin{itemize}

\item $10$ ชุดทดสอบ: $n \leq 2\,000$
\item $17$ ชุดทดสอบ: $n \leq 200\,000$
\item $26$ ชุดทดสอบ: $n \leq 1\,000\,000$
\end{itemize}
\Source

สรวิทย์  สุริยกาญจน์ ( PS.int ) และแนวคิดจากโจทย์ข้อหนึ่งในดินแดนโปแลนด์

ศูนย์ สอวน. โรงเรียนมหิดลวิทยานุสรณ์



\end{problem}

\end{document}