\documentclass[11pt,a4paper]{article}

\usepackage{../../templates/style}

\begin{document}

\begin{problem}{สวนดอกไม้}{standard input}{standard output}{1 second}{32 megabytes}

คุณได้รับแผนที่ของที่ดินเปล่ากว้าง $N$ หน่วย ยาว $M$ หน่วย ที่ถูกแบ่งเป็นส่วนย่อย ๆ จำนวน $N \times M$ ส่วน แต่ละส่วนมีขนาด $1 \times 1$ หน่วย

                แผนที่ดังกล่าวเขียนระบุลักษณะของดินในแต่ละส่วนย่อย โดยใช้สัญลักษณ์สองแบบ คือ ‘.’ ดินดีเหมาะสำหรับปลูกดอกไม้ และ ‘\#’ ดินที่เต็มไปด้วยหิน  ตัวอย่างของแผนที่กรณีที่ $N = 4, M = 6$ แสดงด้านล่าง

\begin{tabular}{l}
\ttfamily ..\#...\\
\ttfamily ...\#\#.\\
\ttfamily ..\#..\#\\
\ttfamily .\#...\#\\
\end{tabular}

เราจะกล่าวว่าส่วนย่อยสองส่วนติดกัน ถ้าในแผนที่ส่วนย่อยทั้งสองอยู่ในแถวเดียวกันและอยู่ติดกัน หรืออยู่ในคอลัมน์เดียวกันของแถวที่ติดกัน  (นั่นคือ เป็นส่วนย่อยที่ติดกันในทิศทาง บน ล่าง ซ้ายและขวา เท่านั้น)

คุณต้องการเลือกพื้นที่เพื่อสร้างสวนดอกไม้ โดยมีเงื่อนไขดังนี้  พื้นที่ดินที่จะสร้างเป็นสวนดอกไม้ได้จะต้องเป็นดินดี และไม่ติดกับดินส่วนที่เต็มไปด้วยหิน   จากตัวอย่างข้างต้น พื้นที่ดินที่สร้างเป็นสวนดอกไม้ได้แสดงด้วยส่วนที่มีเครื่องหมาย "@" ในรูปด้านล่าง

\begin{tabular}{l}
\ttfamily @.\#..@\\
\ttfamily @@.\#\#.\\
\ttfamily @.\#..\#\\
\ttfamily .\#.@.\#\\
\end{tabular}

คุณต้องการหาพื้นที่ที่เหมาะกับการสร้างสวนดอกไม้ที่อยู่ติดกันที่มีขนาดใหญ่ที่สุด  ในตัวอย่างข้างต้น พื้นที่ดังกล่าวคือส่วนบนซ้าย ซึ่งมีขนาด $4$ หน่วย

\bigskip
\underline{\textbf{โจทย์}}  เขียนโปรแกรมรับแผนที่ จากนั้นคำนวณขนาดของพื้นที่ที่เหมาะสำหรับการสร้างสวนดอกไม้ที่มีขนาดใหญ่ที่สุด


\InputFile

\textbf{บรรทัดแรก} ระบุจำนวนเต็มสองจำนวน $N$ และ $M$ $(1 \leq N \leq 30; 1 \leq M \leq 30)$  แทนขนาดของที่ดิน 

\textbf{บรรทัดที่ $2$ ถึง $N+1$} ระบุแผนที่ของที่ดินดังกล่าว กล่าวคือในบรรทัดที่ $i+1$ จะระบุสตริงความยาว $M$ ตัวอักษรแทนพื้นที่ดินในแถวที่ $i$ สตริงดังกล่าวประกอบด้วยตัวอักษร ‘.’ และ ‘\#’ เท่านั้น


\OutputFile

\textbf{มีบรรทัดเดียว} เป็นจำนวนเต็มหนึ่งจำนวน แทนขนาดของพื้นที่ที่เหมาะสำหรับปลูกดอกไม้ที่มีขนาดใหญ่ที่สุด

\Examples

\begin{example}
\exmp{4 6
..\#...
...\#\#.
..\#..\#
.\#...\#}{4}%
\end{example}


\Source

IOI Thailand League 2010 เดือนมีนาคม

\end{problem}

\end{document}