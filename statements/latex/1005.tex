\documentclass[11pt,a4paper]{article}

\usepackage{../../templates/style}

\begin{document}

\begin{problem}{Max Sequence}{standard input}{standard output}{1 second}{64 megabytes}

กำหนดให้ $\{a_1 , a_2 , … , a_n\}$ เป็นลำดับของจำนวนเต็ม และกำหนดให้ $\{a_i , a_{i+1} , … , a_j\}$ เป็นลำดับย่อยของลำดับดังกล่าวนี้ โดยที่ $i$ และ $j$ เป็นจำนวนเต็มบวก และ $1 \leq i \leq j \leq n$ หรือกล่าวอีกนัยหนึ่งคือสมาชิกทุกตัวของลำดับย่อยต้องมีตำแหน่งต่อเนื่องกัน ลำดับย่อยอาจมีได้หลายชุด เมื่อหาค่าผลบวกของสมาชิกทุกตัวในลำดับย่อยแต่ละชุด ผลบวกที่ได้อาจมีค่าแตกต่างกัน

ลำดับย่อยที่มีผลบวกของสมาชิกสูงสุดเรียกว่า \textit{ลำดับย่อยที่มีค่าสูงสุด} ซึ่งอาจมีเพียงชุดเดียวหรืออาจมีหลายชุดก็ได้

ในกรณีที่ \textit{ลำดับย่อยที่มีค่าสูงสุด} มีค่าน้อยกว่าหรือเท่ากับศูนย์ เรียกว่า\textit{ลำดับย่อยว่าง (Empty sequence)}

\bigskip

\textbf{ตัวอย่าง}

ลำดับ $\{4, -6, 3, -2, 6, -4, -6, 6\}$ มี \textit{ลำดับย่อยที่มีค่าสูงที่สุด} เพียงชุดเดียว คือลำดับย่อย $\{3, -2, 6\}$ โดยผลบวกของลำดับย่อยมีค่าเท่ากับ $7$

ลำดับ $\{-2, -3, -1\}$ ไม่มีลำดับย่อยใดที่มีผลบวกมากกว่าศูนย์ ถือว่ามี \textit{ลำดับย่อยว่าง}


\underline{\textbf{โจทย์}} จงเขียนโปรแกรมเพื่อรับจำนวนของสมาชิกในลำดับและรับค่าสมาชิกทุกตัวของลำดับนั้น จากนั้นทำการคำนวณและแสดงผลเป็น \textit{ลำดับย่อยที่มีค่าสูงสุด} และผลบวกของสมาชิกของลำดับย่อยนั้นตามรูปแบบที่โจทย์กำหนด


\InputFile
\textbf{บรรทัดแรก} รับจำนวนเต็มบวก $N$ ซึ่งเป็นจำนวนของสมาชิกในลำดับ โดยที่ $1 \leq N \leq 2\,500$ 

\textbf{บรรทัดที่สอง} รับค่าจำนวนเต็ม $N$ ตัว, $a_1,a_2,...,a_N$ โดยที่ค่า $a_i$ คือค่าของสมาชิกลำดับที่ $i$ ของลำดับนี้ ค่าของสมาชิกแต่ละตัวคั่นด้วยเครื่องหมายเว้นวรรคจำนวน $1$ วรรค รับประกันว่า $-127 \leq a_i \leq +127$ สำหรับค่า $a_i$ ใดๆในลำดับ


\OutputFile

\textbf{ให้แสดงผลตามเงื่อนไขดังต่อไปนี้:}
\begin{enumerate}

\item ในกรณีที่หา \textit{ลำดับย่อยที่มีค่าสูงสุด} ได้เพียงชุดเดียว ให้แสดงลำดับย่อยนั้น
\item ในกรณีที่หา \textit{ลำดับย่อยที่มีค่าสูงสุด} ได้หลายชุด ให้แสดงเฉพาะชุดแรกที่พบเมื่อนับจากต้นลำดับ เช่นลำดับ\\ $\{4, -6, 3, -2, 6, -4, -6, 6, -6, 4, -2, 5\}$ มีลำดับย่อยที่มีค่าสูงสุด $2$ ชุด คือ $\{3, -2, 6\}$ และ $\{4, -2, 5\}$ ซึ่งมีค่าผลบวกของลำดับย่อยเป็น $7$ เท่ากัน ในกรณีนี้ให้แสดงคำตอบเพียงคำตอบเดียว คือลำดับย่อยชุดแรกที่พบคือ $\{3, -2, 6\}$ \newpage
\item การแสดง \textit{ลำดับย่อยที่มีค่าสูงสุด} ให้แสดงสมาชิกของลำดับย่อยทั้งหมดใน\textbf{บรรทัดแรก} โดยใช้เครื่องหมายเว้นวรรคคั่นระหว่างสมาชิกแต่ละตัวจำนวน $1$ วรรค
\item \textbf{บรรทัดที่สอง} ให้แสดงผลเป็นผลบวกของ \textit{ลำดับย่อยที่มีค่าสูงสุด} นั้น
\item \textbf{ในกรณีที่ \textit{ลำดับย่อยที่มีค่าสูงสุด} เป็น \textit{ลำดับย่อยว่าง} ให้แสดงข้อความ "Empty sequence" โดยไม่ต้องแสดงลำดับย่อยและผลบวกของลำดับย่อยนั้น}
\end{enumerate}

\Examples

\begin{example}
\exmp{8
4  -6  3  -2  6  -4  -6   6}{3  -2   6
7}%
\exmp{3
-2  -3  -1}{Empty sequence}%
\end{example}

\Source

การแข่งขันคอมพิวเตอร์โอลิมปิก สอวน. ครั้งที่ 2 มหาวิทยาลัยบูรพา

\end{problem}

\end{document}