\documentclass[11pt,a4paper]{article}

\usepackage{../../templates/style}

\begin{document}

\begin{problem}{Short Message Service}{standard input}{standard output}{1 second}{64 megabytes}

รายการทีวีปัจจุบันมีธุรกิจแนวใหม่โดยหารายได้จากการให้ผู้ชมร่วมโหวตผ่านบริการส่งข้อความ sms ผ่านระบบโทรศัพท์มือถือ โดยปัจจุบันมีหลายรายการที่จัดการแสดงเป็นแบบนำเสนอชีวิตจริง (reality show)

การโหวตคือทุกข้อความที่โหวตให้ผู้เข้าแข่งขันคนใด คนนั้นจะได้หนึ่งคะแนน คะแนนจะสะสมไปเรื่อยๆตั้งแต่สัปดาห์แรกจนถึงสัปดาห์สุดท้าย ทุกสัปดาห์จะมีการประเมินผลโหวตและมีการดำเนินการดังต่อไปนี้

\begin{itemize}
  \item ผู้เข้าแข่งขันที่มีผลโหวตต่ำสุดแต่เพียงผู้เดียวจะถูกคัดชื่อออก นั่นคือจะไม่มีการคัดผู้ใดออกในสัปดาห์ที่มีผู้เข้าแข่งขันมากกว่าหนึ่งรายที่ได้คะแนนรวมต่ำสุด ดังนั้นในแต่ละสัปดาห์จะมีผู้ถูกคัดออกเพียงคนเดียวหรือไม่มีเลย
  \item คะแนนที่ถูกโหวตให้ผู้เข้าแข่งขันที่ถูกคัดออกไปแล้วจะไม่มีผลใดๆ
  \item ถ้าเหลือผู้แข่งขันเพียงคนเดียวผู้นั้นจะเป็นผู้ชนะและจะไม่ถูกคัดออกอีก
\end{itemize}

ทั้งนี้ ทางผู้จัดรายการต้องการให้เราช่วยประมวลผลว่าสถานะปัจจุบันผู้เข้าแข่งขันคนใดยังอยู่บ้างและคะแนนโหวตเป็นเท่าใด ทั้งนี้ผู้เข้าแข่งขันมีทั้งหมด $7$ คน คือ $A, B, C, D, E, F, G$

\underline{\textbf{โจทย์}} จงเขียนโปรแกรมเพื่อแสดงผู้เข้าแข่งขันที่ยังไม่ตกรอบทั้งหมดพร้อมทั้งแสดงผลโหวตสะสมโดยเรียงลำดับจากมากไปน้อย โดยคะแนนรวมของผู้เข้าแข่งขันแต่ละคนจะมีได้ไม่เกิน $30,000$ คะแนน

\InputFile

แต่ละบรรทัดแสดงข้อมูลการโหวตในแต่ละสัปดาห์ ซึ่งอยู่ในรูปสายอักขระที่ประกอบด้วยตัวอักษร $A…G$ (ไม่จำเป็นต้องมีครบทุกตัว) ตัวอักษรแต่ละตัวแสดงถึงการโหวตให้ผู้เข้าแข่งขันคนนั้น ๆ ในสัปดาห์หนึ่ง ๆ จะมีผู้โหวตสูงสุดไม่เกิน $10,000$ คน ข้อมูลจะสิ้นสุดเมื่อบรรทัดนั้นมีตัวอักขระ $'!'$

\OutputFile

\textbf{บรรทัดแรก} แสดงจำนวนเต็ม $n$ ของผู้เข้าแข่งขันที่ยังเหลืออยู่ \\ \textbf{จากนั้น $n$ บรรทัด} แสดงรายชื่อผู้เข้าแข่งขันพร้อมทั้งเรียงคะแนนจากมากไปน้อย โดยแต่ละบรรทัดจะแสดง ตัวอักษร แทนชื่อผู้เข้าแข่งขันที่เหลืออยู่ และความถี่สะสมของการโหวต โดยคั่นด้วยช่องว่างหนึ่งช่อง และถ้าหากว่าความถี่สะสมของผู้เข้าแข่งขันมีจำนวนเท่ากัน ให้เรียงลำดับของผู้เข้าแข่งขันตามพจนานุกรม กล่าวคือต้องพิมพ์ $A$ มาก่อน $B$ , $B$ มาก่อน $C \dotsc F$ มาก่อน $G$ ตามลำดับ

\Examples

\begin{example}
\exmp{ABCDEF
AAABBBBCDE
CCCCCADDCA
!
}{4
C 8
A 6
B 5
D 4
}%
\end{example}

\Source

การแข่งขันคณิตศาสตร์ วิทยาศาสตร์ โอลิมปิกแห่งประเทศไทย สาขาวิชาคอมพิวเตอร์ ประจำปี 2548

\end{problem}

\end{document}