\documentclass[11pt,a4paper]{article}

\usepackage{../../templates/style}

\begin{document}

\begin{problem}{บูชาด้วยสามเหลี่ยม (triangle)}{standard input}{standard output}{1 second}{8 megabytes}

  ในที่สุด พวกคุณก็เข้ามาถึงห้องรูปสามเหลี่ยมด้านใน เพราะคุณสามารถแก้ปริศนาที่ทางเข้าของโบราณสถานมาซู่ได้
  
    นักโบราณคดีมองไปรอบๆ ด้วยความตื่นเต้น บนผนังมีอักขระโบราณมากมาย ด้วยพจนานุกรมภาษาโบราณของเขา เขาก็เริ่มทำความเข้าใจกับอักขระนั้น หลังจากการพยายามแปลอยู่นานก็แปลความได้ว่า


\begin{center}\textit{“จงเผากิ่งไม้เพื่อบูชาเทพเจ้าแห่งตัวเลข  แต่ระวังอย่าให้ท่านเทพโกรธนะจ๊ะ  \^{} - \^{}”}\end{center}


    ด้วยความงุนงงนักโบราณคดีรีบอ่านต่อว่าท่านเทพจะโกรธได้อย่างไร  ที่ผนังอีกด้านระบุเงื่อนไขไว้ดังนี้

\begin{center}
\textit{“เทพเจ้าตัวเลขทรงโปรดรูปสามเหลี่ยมมาก เมื่อได้รับกิ่งไม้จากการบูชาท่านจะนำมาต่อเล่น}\\  \textit{เทพเจ้าจะทรงพระกริ้วถึงขีดสุดเมื่อใดก็ตามที่หยิบกิ่งไม้สามกิ่งมา แล้วไม่สามารถต่อกันเป็นรูปสามเหลี่ยมได้”}
\end{center}

    แม้คุณจะพยายามอธิบายอย่างไรก็ตาม นักโบราณคดีก็ไม่เข้าใจว่าเพราะเหตุใดจึงไม่มีทางต่อกิ่งไม้ความยาว $1,2$ และ $5$ หน่วย หรือกระทั่ง $1, 2$ และ $3$ หน่วยให้เป็นสามเหลี่ยมได้  และยังดื้อรั้นที่จะหากิ่งไม้มาเผาให้ได้
    
    นักโบราณคดีได้ไปรวบรวมกิ่งไม้มาทั้งหมด $N$ กิ่ง โดยที่แต่ละกิ่งมีความยาวเป็นจำนวนเต็ม และรีบเผากิ่งไม้เหล่านั้นโดยไม่สนใจว่าจะเกิดอะไรขึ้นทั้งนั้น ด้วยความตระหนก คุณจึงรีบเขียนโปรแกรมเพื่อคำนวณว่าเทพเจ้าจะทรงกริ้วถึงขีดสุดหรือไม่
    
\bigskip
\underline{\textbf{โจทย์}} เขียนโปรแกรมรับความยาวของกิ่งไม้ทุกกิ่งที่เผาไป แล้วคำนวณว่าในเซตของกิ่งไม้ที่เผาไปนั้น มีกิ่งไม้สามกิ่งที่ไม่สามารถนำมาต่อเป็นสามเหลี่ยมได้หรือไม่

\InputFile

\textbf{บรรทัดแรก} รับจำนวนเต็มบวก $N$ $(1 \leq N \leq 100\,000)$

\textbf{บรรทัดที่ $2$ ถึง $N+1$} มีจำนวนเต็มบรรทัดละหนึ่งจำนวน ระบุความยาวของกิ่งไม้แต่ละกิ่ง เป็นจำนวนเต็มที่มีค่าอยู่ระหว่าง $1$ ถึง $100\,000$


\OutputFile

\textbf{มีบรรทัดเดียว}  ถ้ามีกิ่งไม้สามกิ่งจากเซตของกิ่งไม้ที่ไม่สามารถสร้างสามเหลี่ยมได้ ให้พิมพ์ yes ถ้าไม่มีให้พิมพ์ no


\Examples

\begin{example}
\exmp{5
2
3
2
5
2}{yes}%
\exmp{5
3
5
4
4
3}{no}%
\end{example}


\Source

การแข่งขัน YTOPC Challenge เมษายน 2552


\end{problem}

\end{document}