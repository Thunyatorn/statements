\documentclass[11pt,a4paper]{article}

\usepackage{../../templates/style}

\begin{document}

\begin{problem}{Jump}{standard input}{standard output}{1 second}{64 megabytes}

ในโรงเรียนแห่งหนึ่ง นักเรียนที่นี่รักการโดดเรียนเป็นชีวิตจิตใจ อย่างไรก็ตาม โรงเรียนแห่งนี้มีกฎเหล็กคือ \textbf{“นักเรียนคนหนึ่งสามารถโดดเรียนได้เพียงวันละ $k$ ครั้งเท่านั้น โดยแต่ละครั้งจะต้องโดดเรียนไม่มากกว่า $p$ คาบติดกัน แต่จะต้องเข้าเรียนอย่างน้อย $1$ คาบในแต่ละวัน”} กล่าวอีกนัยหนึ่งได้ว่า นักเรียนสามารถเริ่มต้นโดดเรียนได้ $k$ ครั้งโดยแต่ละครั้งจะโดดเรียนได้อย่างมาก $p$ คาบติดกันแต่ไม่สามารถโดดเรียนทุกคาบเรียนได้

            \textbf{Note:} การโดดเรียนแต่ละครั้งอาจกระทำต่อเนื่องกันได้ (พูดง่ายๆคือสามารถโดดเรียนติดกัน $p \times k$ คาบได้หากต้องการ)

            เป็นที่รู้กันในโรงเรียนว่าแต่ละคาบเรียนนั้น คาบเรียนมีความ “น่าเบื่อ” มากเพียงใด โดยคาบเรียนหนึ่งๆจะมีค่าความน่าเบื่อเฉพาะตัวแต่ละคาบ

            คุณก็เป็นนักเรียนคนหนึ่งในโรงเรียนแห่งนี้ซึ่งต้องการโดดเรียนอย่างคุ้มค่าที่สุด โดยในวันหนึ่งๆ จะมีคาบเรียนทั้งสิ้น $n$ คาบ และคุณต้องการโดดเรียนโดยที่ ค่าของความน่าเบื่อของคาบที่เข้าเรียนที่น่าเบื่อมากที่สุดมีค่าน้อยที่สุด

\bigskip
\underline{\textbf{โจทย์}}  จงเขียนโปรแกรมเพื่อหาว่าคุณจะโดดเรียนเพื่อให้ความน่าเบื่อของคาบเรียนที่คุณต้องเรียนน้อยที่สุดเป็นเท่าไหร่


\InputFile

\textbf{บรรทัดแรก} ประกอบด้วยจำนวนนับ $n, k$ และ $p$ แทนจำนวนคาบ, จำนวนครั้งที่สามารถโดดได้ และความยาวนานของการโดดแต่ละครั้ง $( 1\leq n \leq 1\,000\,000 ; 1 \leq k \leq 1\,000\,000 ; 1 \leq p \leq 1\,000\,000 )$

\textbf{บรรทัดที่ $2$ ถึง $n+1$} เป็นเลขจำนวนนับบรรทัดละ $1$ จำนวน โดยบรรทัดที่ $i+1$ จะแสดงค่าความน่าเบื่อของคาบเรียนที่ $i$ $(1\leq \text{ค่าความน่าเบื่อคาบที่ }i \leq 1\,000\,000\,000)$


\OutputFile

\textbf{มีบรรทัดเดียว} แสดงค่าของความน่าเบื่อที่มากที่สุด เมื่อคุณจัดการโดดเรียนแบบ ค่าของความน่าเบื่อของคาบที่เข้าเรียนที่น่าเบื่อมากที่สุดมีค่าน้อยที่สุด

\Examples

\begin{example}
\exmp{10 2 2
51
42
54
31
12
57
11
51
85
36}{54}%
\exmp{10 6 1
876035016
1354748
882042225
78564538
668028639
586686861
621124669
510077782
824111889
260600125}{510077782}%
\end{example}

  
\Source

สรวิทย์  สุริยกาญจน์ ( PS.int )

ศูนย์ สอวน. โรงเรียนมหิดลวิทยานุสรณ์



\end{problem}

\end{document}