\documentclass[11pt,a4paper]{article}

\usepackage{../../templates/style}

\begin{document}

\begin{problem}{หางไก่ (cocktail)}{standard input}{standard output}{1 second}{32 megabytes}

คุณเพิ่งได้รับงานใหม่ที่ฟาร์มเลี้ยงไก่แห่งหนึ่ง ซึ่งที่ฟาร์มนี้มีไก่พันธุ์พิเศษอยู่หลายตัว ไก่เหล่านี้มีความแปลกอยู่หลายอย่าง อย่างแรกคือไก่เหล่านี้จะร่วมกันฟักไข่เป็นคู่ ๆ อย่างที่สองคือแต่ละตัวจะมีจำนวนหางได้หลายหาง ตั้งแต่หนึ่งหาง สองหาง หนึ่งแสนหาง หรือแม้แต่ไม่มีหาง

    นักวิทยาศาสตร์บอกว่า ถ้าคุณอยากได้ลูกไก่ที่สุขภาพดี แข็งแรง ไก่สองตัวที่ร่วมกันฟักไข่จะต้องมีจำนวนหางรวมกันเท่ากับ $A$ พอดี ถามว่า คุณจะจับคู่ไก่สองตัวให้จำนวนหางเท่ากับ $A$ ได้กี่วิธี

\bigskip
\underline{\textbf{โจทย์}}    รับค่าจำนวนหางของไก่แต่ละตัวในฟาร์ม และหาว่าจะจับคู่ไก่สองตัวให้จำนวนหางรวม $A$ หางได้กี่วิธี

\InputFile
\textbf{บรรทัดแรก} มีจำนวนเต็มหนึ่งจำนวน $N$ แทนจำนวนไก่ที่คุณมี $(1 \leq N \leq 1\,000\,000)$

\textbf{บรรทัดที่ $2$ ถึง $N+1$} มีจำนวนเต็มบรรทัดละหนึ่งตัว บอกจำนวนหางของไก่แต่ละตัว โดยไก่แต่ละตัวจะมีหางไม่เกิน $100\,000$ หาง

\textbf{บรรทัดที่ $N+2$ }จำนวนเต็มหนึ่งตัว แทนค่า $A$ ที่นักวิทยาศาสตร์บอกคุณ


\OutputFile

\textbf{มีบรรทัดเดียว} มีจำนวนเต็มหนึ่งตัว คือ จำนวนวิธีที่สามารถจับคู่ไก่ให้จำนวนหางรวมเท่ากับ $A$


\Examples

\begin{example}
\exmp{5
1
3
4
3
0
4}{3}%
\end{example}

\Scoring

\textbf{ชุดข้อมูลทดสอบมูลค่าไม่เกิน $40$ คะแนน:} $N\leq 100$ 

\textbf{ในทุกชุดข้อมูลทดสอบ:} $N\leq 1\,000\,000$ 

\Source

ทักษพร กิตติอัครเสถียร

\underline{\href{http://www.thailandoi.org/toi.c/05-2009}{TOI.C:05-2009}}


\end{problem}

\end{document}