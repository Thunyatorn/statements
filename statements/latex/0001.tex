\documentclass[11pt,a4paper]{article}

\usepackage{../../templates/style}

\begin{document}

\begin{problem}{Grading}{standard input}{standard output}{1 second}{64 megabytes}

ในการเรียนวิชาคอมพิวเตอร์ ณ โรงเรียนแห่งหนึ่ง ใช้วิธีการเก็บคะแนนในรูปแบบของ

\begin{itemize}
	\item คะแนนเก็บ $30$ คะแนน
    \item คะแนนสอบกลางภาค $30$ คะแนน
    \item คะแนนสอบปลายภาค $40$ คะแนน
\end{itemize} 
   
รวมทั้งสิ้น $100$ คะแนน หลังจากที่จบภาคการศึกษา ฝ่ายทะเบียนวัดผลของโรงเรียนต้องการให้อาจารย์ที่สอนวิชา\\คอมพิวเตอร์มาคีย์คะแนนลงในระบบคอมพิวเตอร์ เพื่อจะได้ทราบถึงเกรดที่นักเรียนแต่ละคนควรจะได้ โดยใช้\\โปรแกรมเข้าช่วย แต่เนื่องจากว่าทางงานทะเบียนวัดผลนี้ยังไม่มีโปรแกรมใช้ (อาจเป็นเพราะเหตุเกิดเมื่อนานมาแล้ว) อาจารย์ฝ่ายทะเบียนวัดผลจึงมาขอให้คุณช่วยเขียนโปรแกรมให้หน่อย

\underline{\textbf{โจทย์}} ให้นักเรียนเขียนโปรแกรมตัดเกรดเพื่อช่วยงานทะเบียนวัดผลของโรงเรียนแห่งนี้

\InputFile

\textbf{บรรทัดแรก} จำนวนเต็มบวก $a$ $(0 \leq a \leq 30)$ เป็นคะแนนเก็บของนักเรียน

\textbf{บรรทัดที่สอง} จำนวนเต็มบวก $b$ $(0 \leq b \leq 30)$ เป็นคะแนนสอบกลางภาคของนักเรียน

\textbf{บรรทัดที่สาม} จำนวนเต็มบวก $c$ $(0 \leq c \leq 40)$ เป็นคะแนนสอบปลายภาคของนักเรียน

\OutputFile

\textbf{บรรทัดแรก} เป็นอักขระใช้แทนเกรดของนักเรียน โดยที่ใช้อักขระตามเงื่อนไขดังต่อไปนี้
\begin{itemize}
    \item A ถ้าคะแนนรวมของนักเรียนอยู่ในช่วง $80 - 100$
    \item B+ ถ้าคะแนนรวมของนักเรียนอยู่ในช่วง $75 - 79$
    \item B ถ้าคะแนนรวมของนักเรียนอยู่ในช่วง $70 - 74$
    \item C+ ถ้าคะแนนรวมของนักเรียนอยู่ในช่วง $65 - 69$
    \item C ถ้าคะแนนรวมของนักเรียนอยู่ในช่วง $60 - 64$
    \item D+ ถ้าคะแนนรวมของนักเรียนอยู่ในช่วง $55 - 59$
    \item D ถ้าคะแนนรวมของนักเรียนอยู่ในช่วง $50 - 54$
    \item F ถ้าคะแนนรวมของนักเรียนอยู่ในช่วง $0 - 49$
\end{itemize}
\Examples

\begin{example}
\exmp{25
25
30
}{A
}%
\end{example}

\Source

 Programming.in.th (Northern series)

\end{problem}

\end{document}