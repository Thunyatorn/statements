\documentclass[11pt,a4paper]{article}

\usepackage{../../templates/style}

\begin{document}

\begin{problem}{Perket}{standard input}{standard output}{1 second}{32 megabytes}

“เปอร์เกต์” เป็นอาหารแสนอร่อยที่ใครๆก็รู้จักกัน และแน่นอนว่าส่วนผสมย่อมเป็นสิ่งที่ต้องพิถีพิถันอย่างยิ่ง

คุณมีส่วนผสมทั้งหมด $N$ ชนิด แต่ละชนิดจะมีความเปรี้ยว $S$ และความขม $B$ เมื่อนำส่วนผสมมารวมกัน ความเปรี้ยวลัพธ์ จะได้จากผลคูณของค่าความเปรี้ยวของทุกชนิดที่ใช้ ในขณะที่ความขมลัพธ์ จะได้จากผลบวกของความขมของทุกชนิดที่ใช้ ส่วนผสมที่ใช้นั้น

เปอร์เกต์ที่อร่อยที่สุดนั้น จะมีผลต่างค่าความเปรี้ยวลัพธ์และค่าความขมลัพธ์ของส่วนผสมทั้งหมดน้อยที่สุด และเราจำเป็นต้องใช้ส่วนผสมอย่างน้อย $1$ ชนิด

\underline{\textbf{โจทย์}} จงเขียนโปรแกรมเพื่อหาค่าผลต่างของความเปรี้ยวลัพธ์และความขมลัพธ์ของส่วนผสม ที่น้อยที่สุด

\InputFile

\textbf{บรรทัดแรก} เป็นจำนวนเต็ม $N$ โดยที่ $1$ $\leq$ $N$ $\leq$ $10$ คือจำนวนชนิดของส่วนผสม

\textbf{บรรทัดที่ $2$ ถึง $N+1$} แต่ละบรรทัด จะมีจำนวนเต็มสองจำนวน $S$ และ $B$ คือค่าความเปรี้ยวและค่าความขมของส่วนผสมชนิดนั้น

รับประกันว่าสำหรับทุกข้อมูลนำเข้า เมื่อนำส่วนผสมทุกชนิดแล้ว จะได้ค่าความเปรี้ยวลัพธ์และความขมลัพธ์ ไม่เกิน $1\,000\,000\,000$

\OutputFile

\textbf{มีบรรทัดเดียว} แสดงค่าผลต่างที่น้อยที่สุด

\Examples

\begin{example}
\exmp{1
3 10}{7}%
\exmp{2
3 8
5 8}{1}%
\end{example}
\begin{example}
\exmp{4
1 7
2 6
3 8
4 9}{1}%
\end{example}

\Note

\underline{\textbf{อธิบายตัวอย่างที่สาม}}

เราเลือกส่วนผสม $3$ ชนิดยกเว้นชนิดแรก

จะได้ค่าความเปรี้ยวลัพธ์เท่ากับ $2 \times 3 \times 4$  $=$ $24$

และค่าความขมลัพธ์เท่ากับ $6+8+9$ $=$ $23$

ซึ่งมีผลต่างเท่ากับ $1$

\Source

COCI 2008/2009, Contest \#2 – November 15, 2008



\end{problem}

\end{document}