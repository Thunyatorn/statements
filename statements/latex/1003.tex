\documentclass[11pt,a4paper]{article}

\usepackage{../../templates/style}

\begin{document}

\begin{problem}{Nugget Number}{standard input}{standard output}{1 second}{64 megabytes}

ร้านฟาสต์ฟู้ดแห่งหนึ่งขายนักเก็ตเป็นกล่อง มีกล่องนักเก็ตอยู่ $3$ ขนาด คือ เล็ก, กลาง, และใหญ่ ใส่นักเก็ตจำนวน $6$, $9$, และ $20$ ชิ้นตามลำดับ

\textbf{เลขนักเก็ต} คือจำนวนเต็มบวกที่เกิดจากผลรวมของจำนวนนักเก็ตในกล่องขนาดต่างๆ เช่น เลข $6$ เป็นเลขนักเก็ตเพราะเป็นจำนวนนักเก็ตในกล่องเล็ก, เลข $12$ เป็นเลขนักเก็ตเพราะเกิดจากการรวมกันของจำนวนนักเก็ตในกล่องเล็กสองกล่อง, เลข $15$ เป็นเลขนักเก็ตเพราะเกิดจากการรวมกันของจำนวนนักเก็ตในกล่องเล็กหนึ่งกล่องและกล่องกลางหนึ่งกล่อง เป็นต้น เลข $4$ และ $10$ ไม่เป็นเลขนักเก็ตเพราะเลขดังกล่าวไม่สามารถเกิดจากการรวมกันของจำนวนนักเก็ตในกล่องขนาดใดๆ ได้


\underline{\textbf{โจทย์}} จงเขียนโปรแกรมเพื่อรับเลขจำนวนเต็ม $n$ แฃ้วแสดงเลขนักเก็ต\textbf{ทั้งหมดที่มีค่าน้อยกว่าหรือเท่ากับค่า $n$}


\InputFile
\textbf{มีบรรทัดเดียว} รับค่า $n$ ที่เป็นจำนวนเต็ม จาก standard input โดยที่ $1 \leq n \leq 100$

\OutputFile

\textbf{มีหลายบรรทัด} พิมพ์เลขนักเก็ตที่น้อยกว่าหรือเท่ากับ $n$ โดยเรียงค่าจากน้อยไปหามาก พิมพ์บรรทัดละหนึ่งตัวเลข ถ้าไม่มีเลขนักเก็ตที่น้อยกว่าหรือเท่ากับ $n$ ให้พิมพ์คำว่า "no"

\Examples

\begin{example}
\exmp{15}{6
9
12
15}%
\exmp{4}{no}%
\end{example}

\Source

การแข่งขันคอมพิวเตอร์โอลิมปิก สอวน. ครั้งที่ 1 มหาวิทยาลัยเกษตรศาสตร์

\end{problem}

\end{document}