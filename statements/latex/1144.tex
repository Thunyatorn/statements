\documentclass[11pt,a4paper]{article}

\usepackage{../../templates/style}

\begin{document}

\begin{problem}{Hypercube}{standard input}{standard output}{1 second}{32 megabytes}

          มีกล่องลูกบาศก์บรรจุพลังงานว่างเปล่า $N$ กล่อง อยู่ในสเปซ $K$ มิติ เดิมทีกล่องแต่ละกล่องไร้ซึ่งพลังงาน เราได้มีเชื้อพลังงานเป็นจุด $M$ จุดเติมเข้าไปในสเปซ โดยเมื่อจุดพลังงานเข้าไปอยู่ภายในหรืออยู่บนผิวกล่องใด จะทำให้พลังงานในกล่องนั้น $1$ หน่วย และจุดพลังงาน $1$ จุดอาจเพิ่มพลังงานให้กล่องมากกว่า $1$ กล่อง

          พิกัดของกล่องจะบอกจุดมุม $2$ จุดที่อยู่ห่างกันมากที่สุดในแต่ละกล่อง ส่วนพิกัดจุดพลังงานนั้น จะบอกในรูปของจุด $1$ จุดเท่านั้น โดยที่พิกัดจะบอกในรูปของ $(x_1,x_2,x_3,..,x_k)$ โดยมีพิกัดบนระบบจำนวนเต็มที่อยู่ระหว่าง $–10^9$ กับ $10^9$

          เราต้องการกล่องที่แข็งแกร่ง กล่องที่แข็งแกร่งคือกล่องที่มีพลังงานรวมมากที่สุด หากมีคำตอบหลายกล่องให้ตอบหมายเลขของทุกกล่อง เรียงจากน้อยไปหามาก

\bigskip
\underline{\textbf{โจทย์}}  จงเขียนโปรแกรมรับพิกัดของกล่องแต่ละกล่องและพิกัดของจุดพลังงาน แล้วแสดงหมายเลขกล่องที่แข็งแกร่ง หรือก็คือกล่องที่มีพลังงานมากที่สุด


\InputFile

\textbf{บรรทัดที่หนึ่ง} รับจำนวนเต็มบวก $N, M, K$ คั่นด้วยช่องว่าง $( 1 \leq N, M \leq 1\,000 )$

\textbf{บรรทัดที่ $2$ ถึง $N+1$} รับจำนวนเต็ม $x_1,x_2,x_3,…,x_k,y_1,y_2,y_3,…,y_k$ แต่ละจำนวนคั่นด้วยช่องว่าง บอกพิกัดจุดมุม $x$ และ $y$ ของแต่ละกล่อง

\textbf{บรรทัดที่ $N+2$ ถึง $N+M+1$} รับจำนวนเต็ม $x_1,x_2,x_3,…,x_k$ แต่ละจำนวนคั่นด้วยช่องว่าง บอกพิกัดของจุดแต่ละจุด


\OutputFile

\textbf{บรรทัดที่หนึ่ง} แสดงจำนวนเต็ม $C$ บอกจำนวนของกล่องที่แข็งแกร่ง

\textbf{บรรทัดที่สอง} จำนวนเต็ม $a_1,a_2,…,a_C$ คั่นด้วยช่องว่าง โดยจะหมายถึงกล่องที่แข็งแกร่งแต่ละกล่อง ซึ่ง $a_1 \leq a_2 \leq a_3 \leq … \leq a_C$


\Examples

\begin{example}
\exmp{3 2 2
2 4 5 6
3 5 1 9
1 3 7 2
2 5
3 6}{2
1 2}%
\end{example}


\Source

พี่โมโม่ (Nautilus Processor)

ศูนย์ สอวน. โรงเรียนมหิดลวิทยานุสรณ์

\end{problem}

\end{document}