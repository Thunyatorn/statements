\documentclass[11pt,a4paper]{article}

\usepackage{../../templates/style}

\begin{document}

\begin{problem}{กบ (frog)}{standard input}{standard output}{1 second}{32 megabytes}

มีเจ้ากบน้อยอยู่ตัวหนึ่ง สามารถกระโดดได้ในทุกทิศทางบนระนาบ และจะกระโดดเป็นระยะทางครั้งละ $X$ หน่วยพอดี อยู่มาวันหนึ่ง เจ้ากบน้อยต้องการกระโดดจากจุด $A$ ไปยังจุด $B$ ซึ่งเป็นจุดบนระนาบ ที่ตั้งอยู่ห่างกัน $Y$ หน่วย เจ้ากบน้อยอยากให้คุณช่วยหาว่า มันจะต้องกระโดดอย่างน้อยกี่ครั้ง จึงจะไปหยุดที่จุด $B$ พอดี

\underline{\textbf{โจทย์}} จงเขียนโปรแกรมเพื่อรับจำนวนเต็ม $X$ และ $Y$ แล้วคำนวณหาจำนวนครั้งที่น้อยที่สุดที่เจ้ากบน้อยต้องใช้ในการกระโดดจากจุด $A$ ไปยังจุด $B$

\InputFile

\textbf{มีบรรทัดเดียว} ระบุจำนวนเต็ม $X$ และ $Y$ $(1 \leq X,Y \leq 1\,000)$ แทนระยะทางในการกระโดดแต่ละครั้งของเจ้ากบน้อย และระยะห่างระหว่างจุด $A$ และจุด $B$

\OutputFile

\textbf{มีบรรทัดเดียว} แสดงจำนวนครั้งที่น้อยที่สุดที่เจ้ากบน้อยต้องใช้ในการกระโดดจากจุด $A$ ไปยังจุด $B$

\Examples 

\begin{example}
\exmp{3 12}{4}%
\exmp{5 23}{5}%
\end{example}

\Source

โจทย์โดย: สุธี เรืองวิเศษ

การแข่งขัน IOI Thailand League เดือนสิงหาคม 2553

\end{problem}

\end{document}