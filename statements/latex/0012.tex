\documentclass[11pt,a4paper]{article}

\usepackage{../../templates/style}

\begin{document}

\begin{problem}{Okviri}{standard input}{standard output}{1 second}{32 megabytes}

\textit{Peter Pan frames} เป็นวิธีการตกแต่งตัวอักษร ด้วยการนำตัวอักษรแต่ละตัวมาล้อมกรอบด้วยกรอบรูปข้าวหลามตัด (Diamond shaped frame) แต่ละกรอบที่อยู่ข้างๆกัน จะมีส่วนที่ซ้อนทับกันอยู่

ตัวอย่างของ \textit{Peter Pan frame} สำหรับ $1$ ตัวอักษรเป็นดังนี้ (\textit{Peter Pan frame} ล้อมตัวอักษร X)

..\#..

.\#.\#.

\#.X.\#

.\#.\#.

..\#..

แต่อย่างไรก็ตาม เราต้องการให้การตกแต่งออกมาดูอลังการมากที่สุด เราจะตกแต่ง\textbf{กรอบที่เป็นพหุคูณของ $3$} โดยใช้ \textit{Wendy frames} แทนการตกแต่งแบบ \textit{Peter Pan frames}

ตัวอย่างของ \textit{Wendy frame} สำหรับ $1$ ตัวอักษร เป็นดังนี้ (\textit{Wendy frame }ล้อมตัวอักษร X)

..*..

.*.*.

*.X.*

.*.*.

..*..


เมื่อไรก็ตามที่ \textit{Wendy frame} เกิดการซ้อนทับกับ \textit{Peter Pan frame} , \textit{Wendy frame} ที่ดูสวยกว่าจะถูกนำมาไว้ด้านหน้า สำหรับตัวอย่างของการซ้อนทับให้คุณดูได้จากตัวอย่างข้อมูลส่งออก


\underline{\textbf{โจทย์}} จงเขียนโปรแกรมรับอักษรภาษาอังกฤษตัวพิมพ์ใหญ่ แล้วแสดงผลเป็นคำที่ได้รับการตกแต่งแล้วตามเงื่อนไขที่ได้กล่าวมา

\InputFile

\textbf{มีบรรทัดเดียว} บรรทัดแรกของข้อมูลนำเข้า จะประกอบไปด้วยตัวอักษรภาษาอังกฤษตัวพิมพ์ใหญ่ ไม่เกิน $15$ ตัวอักษร

\OutputFile

\textbf{มีทั้งหมดห้าบรรทัด} แสดงคำที่ถูกตกแต่งด้วย \textit{Peter Pan frame} และ \textit{Wendy frame}

\Examples

\begin{example}
\exmp{A}{..\#..
.\#.\#.
\#.A.\#
.\#.\#.
..\#..}%
\exmp{DOG}{	..\#...\#...*..
.\#.\#.\#.\#.*.*.
\#.D.\#.O.*.G.*
.\#.\#.\#.\#.*.*.
..\#...\#...*..}%
\exmp{ABCD}{..\#...\#...*...\#..
.\#.\#.\#.\#.*.*.\#.\#.
\#.A.\#.B.*.C.*.D.\#
.\#.\#.\#.\#.*.*.\#.\#.
..\#...\#...*...\#..}%
\end{example}

\Source

Croatian Open Competition in Informatics

Contest 1 – October 28, 2006

\end{problem}

\end{document}