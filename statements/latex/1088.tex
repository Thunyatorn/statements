\documentclass[11pt,a4paper]{article}

\usepackage{../../templates/style}

\begin{document}

\begin{problem}{กิ้งก่า (iguana)}{standard input}{standard output}{0.5 second}{32 megabytes}

คุณเป็นเจ้าของสวนสัตว์ ที่มีกิ้งก่าชนิดประหลาดหายากนำเข้าอยู่ $N$ ตัว อยู่ใน $N$ กรงที่วางเรียงกัน ถ้าเกิดว่ามีคนเอานิ้วไปจิ้มกิ้งก่าชนิดนี้ มันจะเปลี่ยนสีทันที และอาหารที่มันอยากกินก็จะเปลี่ยนไปตามสีของมันด้วย

    กิ้งก่ามีสีที่เป็นไปได้สามสี คือ \textbf{แดง เขียว} และ \textbf{น้ำเงิน} กิ้งก่าสีแดงจะเปลี่ยนสีเป็นสีเขียวเมื่อถูกจิ้ม สีเขียวจะเปลี่ยนเป็นสีน้ำเงิน และสีน้ำเงินจะเปลี่ยนเป็นสีแดง เริ่มต้นกิ้งก่าทุกตัวเป็นสีแดง
    
    เนื่องจากมีกิ้งก่าหิวโซตัวหนึ่งคาบกุญแจของคุณไปกิน ทำให้คุณไม่ได้ล็อกกรง เมื่อวานนี้ มีเด็กมือบอน $M$ คนเข้ามา คนที่ $i$ เดินจิ้มกิ้งก่าตั้งแต่ตัวละ $A_i$ ถึงตัวที่ $B_i$ ตัวละหนึ่งครั้ง จนกิ้งก่าเปลี่ยนสีมั่วไปหมด
    และเนื่องจากมีกิ้งก่าตัวหนึ่งป่วนคุณตอนกำลังสั่งอาหาร ทำให้อาหารทั้งหมดที่สั่งมานั้นกลายเป็นอาหารสำหรับกิ้งก่าสีเดียว ซึ่งคุณจะเปลี่ยนก็ไม่ทันแล้ว ถามว่า คุณต้องไปจิ้มกิ้งก่าอย่างน้อยกี่ครั้ง เพื่อให้กิ้งก่าทุกตัวสามารถกินอาหารที่สั่งมาได้

\bigskip
\underline{\textbf{โจทย์}}  จงเขียนโปรแกรมรับจำนวนกิ้งก่า การจิ้มกิ้งก่าของเด็ก และอาหารสีที่คุณสั่งมา แล้วตอบว่า คุณต้องจิ้มกิ้งก่าอย่างน้อยกี่ครั้ง เพื่อให้ทุกตัวสีเดียวกับอาหารนั้น

\InputFile

\textbf{บรรทัดแรก} มีจำนวนเต็มบวกสองจำนวน $N, M$ $(1 \leq N \leq 100\,000\,000; 0 \leq M \leq 100\,000)$

\textbf{บรรทัดที่ $2$ ถึง $M+1$} มีจำนวนเต็มบวกสองจำนวน $A_i, B_i$ $(1 \leq Ai \leq Bi \leq N)$ เป็นการบอกว่า เด็กมือบอนแต่ละคนจิ้มกิ้งก่าตั้งแต่ตัวไหนถึงตัวไหน

\textbf{บรรทัดที่ $M+2$} มีตัวหนังสือภาษาอังกฤษหนึ่งตัว R (แดง), G (เขียว) หรือ B (น้ำเงิน) เป็นการบอกว่า อาหารที่สั่งมาสำหรับกิ้งก่าสีอะไร


\OutputFile

\textbf{มีบรรทัดเดียว} มีจำนวนเต็ม บอกจำนวนครั้งที่น้อยที่สุดที่ต้องจิ้มกิ้งก่า ที่จะทำให้กิ้งก่าทุกตัวเป็นสีเดียวกับอาหาร

\Examples

\begin{example}
\exmp{3 2
1 2
2 3
R}{5}%
\end{example}

\Note 

\textbf{อธิบายข้อมูลน้ำเข้าและส่งออก}

          หลังจากการจิ้มทั้งหมด กิ้งก่าตัวแรกจะมีสีเขียว ตัวที่สองสีน้ำเงิน และตัวที่สามสีเขียว อาหารที่สั่งมาเป็นสีแดง จึงต้องจิ้มกิ้งก่าตัวแรกสองครั้ง ตัวที่สองหนึ่งครั้ง และตัวที่สามสองครั้ง เพื่อให้ทุกตัวเปลี่ยนเป็นสีแดง

\Scoring

\textbf{50\% ของชุดข้อมูลทดสอบทั้งหมด} $N\leq 10\,000;M \leq 10\,000$

\textbf{70\% ของชุดข้อมูลทดสอบทั้งหมด} $N \leq 100\,000\,000; M\leq 10\,000$

\textbf{100\%ของชุดข้อมูลทดสอบทั้งหมด} $N \leq 100\,000\,000; M\leq 100\,000$
\Source

ทักษพร กิตติอัครเสถียร

\underline{\href{http://www.thailandoi.org/toi.c/04-2009}{TOI.C:04-2009}}

\end{problem}

\end{document}