\documentclass[11pt,a4paper]{article}

\usepackage{../../templates/style}

\begin{document}

\begin{problem}{รหัสลับ (cipher)}{standard input}{standard output}{1 second}{32 megabytes}

สำนักงานถอดรหัสแห่งชาติได้ค้นพบรหัสลับจำนวน $N$ ตัว ที่แก๊งมาเฟียใช้ในการติดต่อสื่อสารกัน ทางสำนักงานต้องการที่จะถอดรหัสนี้ให้ได้เพื่อให้ทราบถึงแผนการของแก๊งมาเฟียดังกล่าว และได้คิดค้นโปรแกรมสำหรับถอดรหัสขึ้น โปรแกรมนี้จะรับข้อมูลนำเข้าเป็นจำนวนเต็ม $N$ แทนจำนวนของรหัส และรหัสอีก $N$ ตัวที่เก็บในอาร์เรย์ $a[0], a[1], a[2], …, a[N-1]$ จากนั้นโปรแกรมจะทำการคำนวณเพื่อถอดรหัสจนได้ผลลัพธ์ออกมาเป็นค่า $val$ ซึ่งการถอดรหัสมีซูโดโค้ดดังนี้

\quad \quad \textit{val = 0}

\quad \quad \textit{for i = 0 to N-1}

\quad \quad \quad \textit{for j = 0 to i}

\quad \quad \quad \quad \textit{for k = 0 to j}

\quad \quad \quad \quad \quad \textit{val += a[k]}

 \quad \quad \textit{val \%= 2553}

\quad \quad  \textit{return val}

สังเกตว่าในโปรแกรมมีส่วนที่เป็นลูป \textit{for} ซ้อนกันถึง $3$ ชั้น ทำให้โปรแกรมใช้เวลาในการทำงานนานมาก โดยเฉพาะเมื่อ $N$ มีค่ามากๆ เช่น ถ้า $N = 1\,000\,000$ โปรแกรมจะต้องใช้เวลาในการทำงานนานถึง $11$ ปี ทางสำนักงานจึงต้องการให้คุณช่วยเขียนโปรแกรมใหม่ที่ให้ผลลัพธ์เหมือนกับโปรแกรมเดิมทุกประการ แต่ต้องใช้เวลาในการทำงานไม่เกิน $1$ วินาที

\bigskip
\underline{\textbf{โจทย์}}  จงเขียนโปรแกรมสำหรับถอดรหัสที่ให้ผลลัพธ์เหมือนกับโปรแกรมเดิมทุกประการ



\InputFile

\textbf{บรรทัดแรก} ระบุจำนวนเต็ม $N$ $(1 \leq N \leq 1\,000\,000)$ แทนจำนวนของรหัส

\textbf{บรรทัดที่สอง} ระบุจำนวนเต็มทั้งหมด $N$ จำนวน แทนรหัสแต่ละตัว จำนวนเต็มแต่ละจำนวนจะมีค่าอยู่ในช่วงตั้งแต่ $1$ ถึง $1\,000$


\OutputFile

\textbf{มีบรรทัดเดียว} แสดงผลลัพธ์ที่ได้จากการถอดรหัส


\Examples

\begin{example}
\exmp{4
1 4 2 3}{43}%
\exmp{5
6 12 7 6 10}{280}%
\end{example}

\Scoring

\textbf{$30$\% ของข้อมูลทดสอบ:} $N \leq 100$
  
\Source

สุธี เรืองวิเศษ

การแข่งขัน IOI Thailand League เดือนสิงหาคม 2553

\end{problem}

\end{document}