\documentclass[11pt,a4paper]{article}

\usepackage{../../templates/style}

\begin{document}

\begin{problem}{รหัสนักเรียน (Codes)}{standard input}{standard output}{1 second}{32 megabytes}

ในโรงเรียนแห่งหนึ่ง นักเรียนแต่ละคนที่มีเลขประจำตัวนักเรียนคนละเลข ซึ่งเลขประจำตัวนักเรียนจะมีคุณสมบัติดังนี้

\begin{itemize}

\item ประกอบด้วยเลขโดดตั้งแต่ $0$ ถึง $9$ จำนวน $3$ หลักพอดี (สามารถมี $0$ นำหน้าได้)

       \item นักเรียนบางคู่อาจมีเลขประจำตัวซ้ำกันได้

       \item เราจะเรียกคู่ของเลขประจำตัวบางคู่ว่า \textit{“คู่พิเศษ”} ก็ต่อเมื่อ มีบางหลักของเลขประจำตัว (หลักหน่วย หลักสิบ หรือหลักร้อย) มีเลขโดดที่เหมือนกัน (กล่าวสั้นๆได้ว่า หลักตรงกันเลขตรงกันอย่างน้อย 1 จุด)
\end{itemize}

\textbf{ตัวอย่างของคู่เลขประจำตัวคู่พิเศษ}

($123$, $145$)\\ ($674$, $374$)  \\                ($071$, $981$)      \\            ($007$, $007$)      \\            ($000$, $190$)
    

\bigskip
\underline{\textbf{โจทย์}}  กำหนดนักเรียนจำนวน $n$ คน จงหาจำนวนของคู่นักเรียนที่มีเลขประจำตัวเป็น \textit{คู่พิเศษ}


\InputFile

\textbf{บรรทัดแรก} ประกอบด้วยจำนวนนับ $n$ แทนจำนวนของนักเรียน $( 1 \leq n \leq 100 \,000)$

\textbf{บรรทัดที่ $2$ ถึง $n+1$} เป็นข้อมูลเลขประจำตัวของนักเรียนแต่ละคนซึ่งแต่ละบรรทัดจะประกอบด้วยเลขโดด $3$ หลัก


\OutputFile

\textbf{มีบรรทัดเดียว} ระบุจำนวนคู่ของนักเรียนที่มีเลขประจำตัวเป็น\textit{ คู่พิเศษ}

\Examples

\begin{example}
\exmp{5
235
236
136
004
174}{5}%
\exmp{5
123
123
123
625
175}{10}%
\end{example}

\Note 

คำตอบอาจมีค่ามากเกินว่าที่ int จะรองรับได้ คุณควรใช้ long long ในการเก็บค่าคำตอบ
  
\Scoring
  
\textbf{$10$\% ของชุดทดสอบทั้งหมด:} $n \leq 10$

\textbf{$40$\% ของชุดทดสอบทั้งหมด:} $n \leq 5\,000$

\textbf{$100$\% ของชุดทดสอบทั้งหมด:} $n \leq 100\,000$

            
\Source

สรวิทย์  สุริยกาญจน์ ( PS.int )

ศูนย์ สอวน. โรงเรียนมหิดลวิทยานุสรณ์

\end{problem}

\end{document}