\documentclass[11pt,a4paper]{article}

\usepackage{../../templates/style}

\begin{document}

\begin{problem}{สามเหลี่ยมใหญ่ (bigtriangle)}{standard input}{standard output}{1 second}{32 megabytes}

ในขณะที่คุณกำลังสำรวจโบราณสถานของอาณาจักรสุโขทัยโบราณอยู่ คุณได้พบเสาหินโบราณที่หลงเหลือจากการทำลายจำนวนมากตั้งเรียงรายอยู่โดยรอบ เพื่อนของคุณนึกสนุกจึงท้าคุณให้หาพื้นที่ทั้งหมดที่ล้อมรอบด้วยเสาหินเหล่านี้ โดยมีรางวัลเป็นขนมแสนอร่อยที่เพื่อนของคุณซื้อมา

ด้วยความที่คุณมีความเป็นนักคณิตศาสตร์อยู่ในตัว คุณทราบดีว่ามันยากเกินไปที่จะหาพื้นที่ดังกล่าวในระยะเวลาอันสั้น คุณจึงขอเพื่อนเปลี่ยนเป็นหาพื้นที่สามเหลี่ยมที่ใหญ่ที่สุดที่มีเสาหินเป็นจุดมุมของสามเหลี่ยมแทน

\textbf{หมายเหตุ:} สำหรับสามเหลี่ยมที่จุดยอดทั้งสามมีพิกัด $(x_1, y_1), (x_2, y_2)$ และ $(x_3, y_3)$  พื้นที่ของสามเหลี่ยมจะมีค่าเท่ากับ: $$\frac{|x_1y_2 + x_2y_3 + x_3y_1 - y_1x_2 - y_2x_3 - y_3x_1|}{2}$$

\underline{\textbf{โจทย์}} จงเขียนโปรแกรมรับพิกัดของเสาหินทั้งหมด แล้วคำนวณหาพื้นที่สามเหลี่ยมที่ใหญ่ที่สุดที่มีเสาหินเป็นจุดมุม

\InputFile

\textbf{บรรทัดแรก} ระบุจำนวนเต็มบวก $N$ $(1 \leq N \leq 100)$ แทนจำนวนเสาหินทั้งหมด

\textbf{$N$ บรรทัดถัดมา} ระบุพิกัดของเสาหิน กล่าวคือ สำหรับบรรทัดที่ $i+1$ $(1 \leq i \leq N)$ ระบุจำนวนเต็มสองจำนวน $X_i$ และ $Y_i$ คั่นด้วยช่องว่าง โดย $(X_i, Y_i)$ คือพิกัดของเสาหินต้นที่ $i$ $(-1\,000 \leq X_i,Y_i \leq 1\,000)$ รับประกันว่าเสาหินสองต้นจะไม่มีพิกัดเดียวกัน

\OutputFile

\textbf{มีบรรทัดเดียว} มีหนึ่งจำนวน เป็นจำนวนจริงแทนพื้นที่ของสามเหลี่ยมที่ใหญ่ที่สุดที่มีเสาหินเป็นจุดมุม แสดงเป็นทศนิยม $3$ ตำแหน่ง

\Examples

\begin{example}
\exmp{5
1 2
2 1
0 0
3 4
-1 -2}{3.000}%
\exmp{5
1 1
2 2
3 3
1 4
4 1}{4.500}%
\end{example}

\Source

โจทย์โดย: ธงชัย วิโรจน์ศักดิ์เสรี

การแข่งขัน IOI Thailand League เดือนสิงหาคม 2553

\end{problem}

\end{document}6