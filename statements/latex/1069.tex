\documentclass[11pt,a4paper]{article}

\usepackage{../../templates/style}

\begin{document}

\begin{problem}{มาซู่ (mazu)}{standard input}{standard output}{1 second}{16 megabytes}

   “ถึงแล้ว” นักโบราณคดีบอกคุณ ส่องไฟฉายไปข้างหน้าท่ามกลางหมอก อักษรสลักบนหินเก่าคล้ายอักษรขอมโบราณผสมกับอักษรซิริลลิก อ่านได้ว่า “มาซู่”
   
   
    คุณอ้าปากค้างด้วยความประหลาดใจ ในหมอกขาวมีลูกแก้วกลมลอยเรียงเป็นแถวแนวนอนสุดสายตา ด้วยปัญญาอันปราดเปรื่องคุณนับได้ว่ามีลูกแก้วทั้งหมด $N$ ลูก แต่ละลูกจะส่องแสงเป็นสีสัน คุณบอกได้ว่ามีไม่เกิน $26$ สี แต่ละสีแทนด้วยตัวอักษรภาษาอังกฤษพิมพ์ใหญ่ \textbf{A-Z}
    
    
    นักโบราณคดีรู้ว่าลูกแก้วเหล่านี้เป็นด่านป้องกันมิให้คนเข้ามาในโบราณสถาน เมื่อกลไกทำงาน ลูกแก้วที่มีสีเดียวกันที่อยู่ติดกันจะระเบิดและสลายไปทั้งคู่  การระเบิดจะเริ่มจากทางด้านซ้ายสุดไปจนขวาสุด และการระเบิดจะเกิดทีละสองลูกเท่านั้น  เมื่อลูกแก้วสองลูกระเบิด ลูกแก้วที่เหลือทางขวาทั้งหมดจะเลื่อนไปติดลูกแก้วที่อยู่ทางด้านซ้าย  และจะระเบิดต่อไปเรื่อย ๆจนไม่อาจระเบิดต่อได้
    
    
    นักโบราณคดีวิเคราะห์จนรู้ว่า ก่อนที่จะผ่านด่านป้องกันเข้าไปได้ ต้องป้อนคำตอบลงไปในจอศักดิ์สิทธิ์ของเทพเจ้าแห่งตัวเลขว่า หลังจากลูกแก้วเหล่านี้ระเบิดเสร็จสิ้นแล้ว ลำดับของลูกแก้วที่เหลือจะเป็นอย่างไร

\bigskip
\underline{\textbf{โจทย์}}   เขียนโปรแกรมที่รับข้อมูลลำดับสีของลูกแก้วทั้งหมดเรียงจาก\textbf{ซ้ายไปขวา}  แล้วคำนวณว่าเมื่อการระเบิดสิ้นสุดลงแล้ว ลำดับของลูกแก้วที่เหลือจาก\textbf{ขวาไปซ้าย}จะเป็นอย่างไร

\InputFile

\textbf{บรรทัดแรก} รับจำนวนเต็ม $N$ $(1 \leq N \leq 100\,000)$ แทนจำนวนลูกแก้วตอนเริ่มต้น

\textbf{บรรทัดที่ $2$ ถึง $N+1$} จะระบุสีของลูกแก้วตั้งต้นตามลำดับจากซ้ายไปขวา กล่าวคือในบรรทัดที่ $i + 1$ สำหรับ $1 \leq i \leq N$ จะมีตัวอักษรพิมพ์ใหญ่ A-Z อยู่หนึ่งตัวอักษร ระบุสีของลูกแก้วที่อยู่ในตำแหน่งที่ $i$ จากทางซ้ายสุด


\OutputFile

\textbf{บรรทัดแรก} พิมพ์จำนวนลูกแก้วที่เหลือตอนสุดท้าย ถ้าไม่มีเหลือเลยให้พิมพ์ $0$
\textbf{บรรทัดที่สอง} พิมพ์สตริงแทนลำดับลูกแก้วที่เหลือ จากขวาไปซ้าย ถ้าไม่มีลูกแก้วเหลือเลยให้พิมพ์ empty

\Examples

\begin{example}
\exmp{5
A
D
D
D
C}{3
CDA}%
\exmp{4
A
B
B
A}{0
empty}%
\end{example}


\Source

การแข่งขัน YTOPC Challenge เมษายน 2552


\end{problem}

\end{document}