\documentclass[11pt,a4paper]{article}

\usepackage{../../templates/style}

\begin{document}

\begin{problem}{สงครามของนายพล (general)}{standard input}{standard output}{1 second}{64 megabytes}

    เกมออนไลน์ใหม่เพิ่งเปิดตัวขึ้น ในเกมนี้ผู้เล่นแต่ละคนจะเล่นเป็นนายพลซึ่งมีหน้าที่คุมทหารจำนวนหนึ่ง
    
    เมื่อเกิดการท้ารบระหว่างผู้เล่นสองคน  ผู้เล่นที่ชนะการสู้รบคือผู้เล่นที่มีทหารจำนวนมากกว่า แต่ถ้าหากทั้งสองฝ่ายมีจำนวนทหารเท่ากัน ผู้เล่นที่ชนะคือผู้เล่นที่มีหมายเลขประจำตัวนายพลที่น้อยกว่า
    ผู้เล่นที่ชนะ จะได้กำลังพลเพิ่มขึ้น ซึ่งเท่ากับทหารจำนวนครึ่งหนึ่งของฝ่ายที่แพ้ (กรณีที่จำนวนทหารหารด้วยสองไม่ลงตัว ให้ปัดเศษทิ้ง)
    
    ผู้เล่นที่แพ้จะถูกเปลี่ยนจากสถานะ “นายพล” เป็นสถานะ “เชลย” ของผู้เล่นที่ชนะ  นอกจากนี้ผู้เล่นที่เคยตกเป็นเชลยของฝ่ายแพ้ จะกลายเป็นเชลยของฝั่งผู้ชนะในการแข่งขันด้วย
    
    บางครั้งนายพลบางคนก็ขี้ขลาด ไม่ยอมท้ารบกับนายพลด้วยกันเอง แต่กลับไปท้ารบกับเชลยของนายพลคนอื่น ในกรณีเหล่านี้ นายพลของเชลยที่ถูกท้ารบนั้นก็มีหน้าที่ต้องปกป้องเชลยของตน และจะต้องต่อสู้แทนเชลยคนนั้น หรือบางครั้งเชลยก็ทะเลาะกันเอง จนทำให้นายพลของเชลยเหล่านี้ต้องมารบกัน ก็เป็นไปได้เช่นเดียวกัน
    
    คุณเป็นผู้ดูแลระบบเกมออนไลน์นี้ คุณได้รับข้อมูลการปะทะกันระหว่างผู้เล่นแต่ละคู่ หน้าที่ของคุณคือบอกว่าในแต่ละครั้ง ผู้เล่นฝั่งใดเป็นฝ่ายชนะ
    

\bigskip
\underline{\textbf{โจทย์}}  คุณมีไฟล์ประวัติว่าในช่วงหนึ่งอาทิตย์ที่ผ่านมามีใครท้ารบกับใครบ้าง หน้าที่ของคุณคือคำนวณว่าในการสู้รบแต่ละครั้ง นายพลคนไหนเป็นผู้ชนะ  เนื่องจากอาจมีการท้ารบระหว่างเชลยหลายคนที่อยู่ใต้การควบคุมของนายพลคนเดียวกันได้ ในกรณีนี้ให้ตอบ $-1$


\InputFile

\textbf{บรรทัดแรก} มีจำนวนเต็มสองจำนวน $N, M$ แทนจำนวนนายพลและจำนวนครั้งในการรบ $(1 \leq N,M \leq 100\,000)$

\textbf{บรรทัดที่ $2$ ถึง $N+1$} บอกข้อมูลของจำนวนทหารของผู้เล่นแต่ละคนในตอนเริ่มต้น โดยในบรรทัดที่ $i+1$ มีจำนวนเต็มหนึ่งตัว แสดงจำนวนทหารที่นายพลหมายเลข $i$ มี ผู้เล่นแต่ละคนมีทหารจำนวนไม่เกิน $10\,000$ นายในตอนเริ่มต้น

\textbf{บรรทัดที่ $N+2$ ถึง $N+M+1$} มีจำนวนเต็มบรรทัดละสองตัวคือ $a,b$ แสดงว่า $a$ และ $b$ ท้ารบกัน\\ $(1 \leq a,b \leq N; a \neq b)$



\OutputFile

\textbf{มี $M$ บรรทัด} แต่ละบรรทัดบอกหมายเลขประจำตัวนายพลของฝั่งผู้ชนะของการรบแต่ละครั้ง ถ้าไม่มีการรบเกิดขึ้น (คนที่ท้ารบกันเป็นเชลยของนายพลคนเดียวกัน) ให้พิมพ์ $-1$

\Examples

\begin{example}
\exmp{5 4
3
4
5
6
7
1 5
1 2
1 2
3 4}{5
5
-1
4}%
\end{example}

\Scoring

\textbf{ชุดข้อมูลทดสอบมูลค่าไม่เกิน $40$ คะแนน: }มีค่า $N,M \leq 1\,000$ 

\textbf{ในทุกชุดข้อมูลทดสอบ:} มีค่า $N,M \leq 100\,000$
  
\Source

ทักษพร กิตติอัครเสถียร

\underline{\href{http://www.thailandoi.org/toi.c/05-2009}{TOI.C:05-2009}}

\end{problem}

\end{document}