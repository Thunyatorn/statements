\documentclass[11pt,a4paper]{article}

\usepackage{../../templates/style}

\begin{document}

\begin{problem}{อุลตร้าแมน (Ultraman)}{standard input}{standard output}{2 second}{128 megabytes}

 อุลตร้าแมนเป็นสัตว์สังคม (สิ่งมีชีวิตที่อยู่กันเป็นกลุ่ม) แน่นอน ในกลุ่มของอุลตร้าแมนย่อมต้องมี \textit{“อุลตร้าแมนจอมโจรยอดนักฆ่า”} ซึ่งสร้างความเดือดร้อนให้กับสังคมอุลตร้าแมนอย่างมาก ดังนั้นในกลุ่มของอุลตร้าแมนจึงได้มีการแต่งตั้ง \textit{“อุลตร้าแมนลาดตระเวน”} เพื่อรักษาความปลอดภัยในชุมชนอุลตร้าแมน

            ลักษณะของหมู่บ้านอุลตร้าแมนจะประกอบด้วยบ้านครอบครัวอุลตร้าแมนทั้งสิ้น $n$ หลัง และจะมี \textit{“อุลตร้าแมนลาดตระเวน”} ทั้งสิ้น $k$ คน ทำให้เราสามารถแบ่งบ้านครอบครัวอุลตร้าแมนทั้ง $n$ หลัง ออกเป็น $k$ กลุ่ม เพื่อให้อยู่ในความรับผิดชอบของ \textit{“อุลตร้าแมนลาดตระเวน”} แต่ละบุคคล

            บ้านครอบครัวอุลตร้าแมนแต่ละหลังจะอยู่ในตำแหน่งพิกัด $(x,y)$ เมื่อ $x$ และ $y$ เป็นจำนวนนับ บนกริดขนาด $1\,000 \times 1\,000$ ซึ่งจะทำให้บ้านแต่ละหลังจะมีทางเดินหากัน (เอดจ์) เชื่อมถึงกันโดยบ้านที่อยู่พิกัด $(x_1,y_1)$ และ $(x_2,y_2)$ จะมีระยะห่างกัน $\sqrt{(x_1-x_2)^2 + (y_1-y_2)^2 } $

            สมมติ ให้ \textit{“อุลตร้าแมนลาดตระเวน”} คนหนึ่งดูแลบ้านครอบครัวอุลตร้าแมนจำนวนหนึ่ง เราจะนิยาม \textit{“ค่าความเหนื่อยของอุลตร้าแมนลาดตระเวน”} คนนี้ ให้มีค่าเท่ากับระยะห่างที่มากที่สุดเมื่อเราสร้างทางเชื่อมระหว่างบ้านครอบครัวอุลตร้าแมนทั้งสิ้น $p-1$ เส้นทาง (เมื่อมีบ้านครอบครัวอุลตร้าแมนทั้งสิ้น $p$ หลัง) เพื่อเชื่อมบ้านครอบครัวอุลตร้าแมนทั้ง $p$ หลังเข้าด้วยกัน (พูดง่ายๆคือระยะทางของเอดจ์ที่มากที่สุดเมื่อคุณสร้าง\textit{ minimum spanning tree} บนโหนดทั้ง $p$ โหนดจากเอดจ์ทั้งสิ้น $p \times (p-1)/2$ เอดจ์)

            เรากำหนดให้ค่าความเหนื่อยทั้งมวล ของเหล่า \textit{“อุลตร้าแมนลาดตระเวน”} คือค่าความเหนื่อยที่มากที่สุดของ \textit{“อุลตร้าแมนลาดตระเวน”} คนใดคนหนึ่ง

          

\bigskip
\underline{\textbf{โจทย์}}    กำหนดพิกัดของบ้านครอบครัวอุลตร้าแมนทั้ง $n$ หลัง และจำนวนของ \textit{“อุลตร้าแมนลาดตระเวน”} $k$ จงหาค่าความเหนี่อยทั้งมวลที่น้อยที่สุด


\InputFile

\textbf{บรรทัดแรก} ประกอบด้วยจำนวนนับ $n$ และ $k$ แทนจำนวนของบ้านครอบครัวอุลตร้าแมน และจำนวนของ \textit{“อุลตร้าแมนลาดตระเวน”} $( 1 \leq k \leq n \leq 1\,000)$

\textbf{บรรทัดที่ $2$ ถึง $n+1$} ประกอบด้วยจำนวนนับ $2$ จำนวน $x$ และ $y$ แทนพิกัดของบ้านครอบครัวอุลตร้าแมนแต่ละหลัง $( 1 \leq x, y \leq 1\,000 )$


\OutputFile

\textbf{มีบรรทัดเดียว} แสดงค่า \textbf{กำลังสอง} ของค่าความเหนื่อยทั้งมวลของเหล่าอุลตร้าแมน เมื่อคุณจัดสรรความรับผิดชอบของเหล่าอุลตร้าแมนอย่างดีที่สุด

\Examples

\begin{example}
\exmp{10 2
366 409
248 374
485 127
745 944
313 261
362 115
370 55
546 876
341 474
748 293}{96725}%
\exmp{10 5
712 420
413 638
854 53
152 430
430 703
248 450
758 388
653 578
2 302
637 198}{38884}%
\end{example}

\newpage
\Scoring 

\textbf{$30$\% ของชุดทดสอบทั้งหมด:} $n \leq 10$

\textbf{$50$\% ของชุดทดสอบทั้งหมด:} $n \leq 100$
           
\textbf{$100$\% ของชุดทดสอบทั้งหมด:} $n \leq 1\,000$
            
\Source

สรวิทย์  สุริยกาญจน์ ( PS.int ) และแนวคิดจากค่ายสสวท. ค่ายที่ 2 ระยะ 2 ประจำปี 2553

ศูนย์ สอวน. โรงเรียนมหิดลวิทยานุสรณ์

\end{problem}

\end{document}