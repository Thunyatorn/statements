\documentclass[11pt,a4paper]{article}

\usepackage{../../templates/style}

\begin{document}

\begin{problem}{Reseto}{standard input}{standard output}{1 second}{32 megabytes}

ตระแกรงร่อนของอีราโทสเธเนส เป็นอัลกอริทึมที่ให้หาจำนวนเฉพาะจนถึงจำนวนนับ $N$ \textbf{วิธีการมีดังนี้}:

\begin{enumerate}
\item เขียนจำนวนนับตั้งแต่ \textbf{$2$ จนถึง $N$ ทั้งหมด}
\item หาจำนวนที่\textbf{น้อยที่สุด}ที่ยังไม่ถูกขีดฆ่า และเราให้จำนวนนั้นคือ $P$\textbf{ ($P$ คือจำนวนเฉพาะ)}
\item ขีดฆ่าจำนวนที่เป็นพหุคูณของ $P$ \textbf{ทุกตัวที่ยังไม่ถูกขีดค่า}
\item ถ้ายังคงมีจำนวนนับที่ยังไม่ถูกขีดฆ่า ให้กลับไปทำขั้นตอนที่ 2
\end{enumerate}

\underline{\textbf{โจทย์}} จงเขียนโปรแกรมที่รับจำนวนเต็ม $N, K$ ให้หาว่าจำนวนตัวที่ $K$ ที่ถูกขีดฆ่าคือจำนวนใด

\InputFile

\textbf{มีบรรทัดเดียว} รับจำนวนเต็ม $N$ และ $K$ โดยที่ $(2 \leq K < N \leq 1\,000)$

\OutputFile

\textbf{มีบรรทัดเดียว} จำนวนเต็มลำดับที่ $K$ ที่ถูกขีดฆ่านับแต่เริ่มต้นอัลกอริทึม

\Examples

\begin{example}
\exmp{7 3}{6}%
\exmp{15 12}{7}%
\exmp{10 7}{9}%
\end{example}

\Note

\underline{\textbf{อธิบายตัวอย่างที่สาม}}

ลำดับจำนวนตัวเลขที่ถูกขีดฆ่าคือ $2, 4, 6, 8, 10, 3, 9, 5$ และ $7$ \underline{และจำนวนที่ $7$ \textbf{ที่}}\textbf{ถู}\underline{\textbf{กขีดฆ่าคือ $9$}}

\Source

COCI 2008/2009, Contest \#2 – November 15, 2008



\end{problem}

\end{document}