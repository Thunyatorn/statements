\documentclass[11pt,a4paper]{article}

\usepackage{../../templates/style}

\begin{document}

\begin{problem}{รถโรงเรียน (schoolbus)}{standard input}{standard output}{0.5 second}{32 megabytes}

หลังโรงเรียนเลิก นักเรียนได้ขึ้นรถโรงเรียนเพื่อที่จะเดินทางกลับบ้าน บ้านของนักเรียนทุกคนจะตั้งอยู่บนถนนสายเดียวกันหมด โดยมีโรงเรียนตั้งอยู่ที่หัวถนนซึ่งแทนด้วยพิกัด $0$ และบ้านของนักเรียนแต่ละคนจะตั้งอยู่ที่พิกัดซึ่งเป็นจำนวนเต็มบวก โดยพิกัดดังกล่าวแทนระยะห่างจากโรงเรียน

โรงเรียนนี้เป็นโรงเรียนที่มีกฎระเบียบเข้มงวดมาก นักเรียนแต่ละคนจะมีหมายเลขประจำตัวตั้งแต่ $1, 2, 3$ เรียงไปเรื่อยๆ และลำดับการลงจากรถของนักเรียนจะต้องเรียงไปตามหมายเลขจากน้อยไปหามาก กล่าวคือ นักเรียนหมายเลข $i$ จะต้องลงจากรถก่อนนักเรียนหมายเลข $i+1$ เสมอ

รถโรงเรียนจะวิ่งออกจากโรงเรียนไปตามถนนไปเรื่อยๆ และจะจอดเพื่อส่งนักเรียนในบางจุด ซึ่งจุดเหล่านั้นไม่จำเป็นต้องมีพิกัดเป็นจำนวนเต็ม แต่รถจะวิ่งไปในทิศทางเดียว ไม่มีการวิ่งย้อนกลับเด็ดขาด เนื่องจากกฎระเบียบอันเข้มงวดของโรงเรียนทำให้คุณครูไม่สามารถส่งนักเรียนบางคนลงที่บ้านพอดีได้ อย่างไรก็ตาม คุณครูได้พยายามส่งนักเรียนแต่ละคนให้ดีที่สุด โดยคุณครูมีหลักการว่า ให้พิจารณานักเรียนที่จุดลงจากรถอยู่ห่างจากบ้านของตนมากที่สุด คุณครูต้องการให้ระยะห่างนั้นมีค่าน้อยที่สุดเท่าที่จะทำได้

\bigskip
\underline{\textbf{โจทย์}}  จงเขียนโปรแกรมเพื่อรับจำนวนนักเรียน และพิกัดของบ้านของนักเรียนแต่ละคน แล้วคำนวณหาระยะห่างที่น้อยที่สุดที่เป็นไปได้สำหรับนักเรียนที่จุดลงจากรถอยู่ห่างจากบ้านของตนมากที่สุด


\InputFile

\textbf{บรรทัดแรก} ระบุจำนวนเต็ม $N$ $(1 \leq N \leq 1\,000\,000)$ แทนจำนวนนักเรียนทั้งหมด

\textbf{บรรทัดที่ $2$ ถึง $N+1$ }ในบรรทัดที่ $i+1$ ระบุจำนวนเต็ม $D_i$ $(1 \leq D_i \leq 1\,000\,000\,000)$ แทนพิกัดของบ้านของนักเรียนหมายเลข $i$


\OutputFile

\textbf{มีบรรทัดเดียว} ระบุระยะห่างที่น้อยที่สุดที่เป็นไปได้สำหรับนักเรียนที่จุดลงจากรถอยู่ห่างจากบ้านของตนมากที่สุด โดยตอบเป็นทศนิยม $6$ ตำแหน่ง

\Examples

\begin{example}
\exmp{3
3
2
1}{1.000000}%
\exmp{5
5
2
7
8
3}{2.500000}%
\end{example}

\Scoring

$30$\% ของข้อมูลทดสอบ: $N \leq 1\,000$

$50$\% ของข้อมูลทดสอบ: $N \leq 100\,000$

\Source

สุธี เรืองวิเศษ

ค่ายอบรมเตรียมความพร้อมผู้แทนประเทศไทย สำหรับการแข่งขันคอมพิวเตอร์โอลิมปิก พ.ศ. 2554

\end{problem}

\end{document}