\documentclass[11pt,a4paper]{article}

\usepackage{../../templates/style}

\begin{document}

\begin{problem}{Power}{standard input}{standard output}{1 second}{64 megabytes}

ตัวเลขจำนวนเต็มบวกที่จัดเป็นกำลัง $k$ สมบูรณ์ ($k$ เป็นจำนวนเต็มและ $k > 1$) เป็นตัวเลขที่สามารถเขียนให้อยู่ในรูป $x^k$ ได้โดยที่ $x$ เป็นจำนวนเต็มใด ๆ เช่น $8$ จัดเป็นกำลังสามสมบูรณ์ เพราะ $2^3 = 8$ ส่วน $2\,401$ เป็นกำลังสี่สมบูรณ์เนื่องจาก $7^4 = 2\,401$ นอกจากนั้นยังถือเป็นกำลังสองสมบูรณ์ด้วย เนื่องจาก $49^2 = 2\,401$ เช่นกัน


\bigskip
\underline{\textbf{โจทย์}}  จงเขียนโปรแกรมเพื่อรับข้อมูลชุดตัวเลขเข้ามา และตรวจสอบว่าตัวเลขแต่ละตัวจัดเป็นกำลัง $k$ สมบูรณ์หรือไม่ และ $k$ คือค่าใด หากค่า $k$ เป็นไปได้มากกว่าหนึ่งค่าให้รายงานค่าที่มากที่สุด


\InputFile

\textbf{บรรทัดแรก} รับค่าจำนวนเต็ม $n$ $(1 \leq n \leq 1\,000)$ ซึ่งเป็นจำนวนตัวเลขที่ต้องการตรวจสอบ

\textbf{บรรทัดที่ $2$ ถึง $n+1$} ในบรรทัดที่ $i+1$ ให้รับค่าที่สงสัยในลำดับที่ $i$ แต่ละบรรทัดเป็นตัวเลข $y_i$ ที่ต้องการตรวจสอบ โดย $(2 \leq y_i \leq 100\,000\,000)$

\OutputFile

\textbf{มี $n$ บรรทัด} ในบรรทัด $i$ ระบุว่าตัวเลขที่สงสัยในลำดับที่ $i$ เป็นกำลัง $k$ สมบูรณ์สำหรับจำนวนเต็มบวก $k > 1$ บางตัวหรือไม่ ถ้าใช่ให้แสดงค่า $k$ ที่มากที่สุด ถ้าไม่ใช่ให้แสดงคำว่า "NO"

\Examples

\begin{example}
\exmp{5
1000000
994009
20
59050
524288}{6
2
NO
NO
19}%
\end{example}


\Source

การแข่งขันคณิตศาสตร์ วิทยาศาสตร์ โอลิมปิกแห่งประเทศไทย สาขาวิชาคอมพิวเตอร์ ประจำปี 2550

\end{problem}

\end{document}