\documentclass[11pt,a4paper]{article}

\usepackage{../../templates/style}

\begin{document}

\begin{problem}{โกงเงินค่าอาหาร (Thief)}{standard input}{standard output}{1 second}{64 megabytes}

คุณเปิดร้านอาหารร่วมกันเพื่อนสนิทชั่วนิรันดร์ของคุณ ทำให้ทั้งร้านอาหารมีพนักงานอยู่เพียงสองคนคือคุณและเขา

                ในแต่ละวัน จะมีลูกค้าเข้ามาใช้บริการร้านอาหารของคุณทั้งสิ้น $N$ คน โชคร้ายที่คนเหล่านั้นมักจะโกงเงินค่าอาหารของร้านคุณอยู่เสมอ ทำให้คุณต้องตัดสินใจว่าเมื่อลูกค้าออกจากร้านอาหารไปแล้วคุณ (หรือเพื่อนสนิทชั่วนิรันดร์ของคุณ) จะวิ่งไล่ตามเก็บค่าอาหารที่พวกเขาโกงหรือไม่ ซึ่งลูกค้าคนที่ $i$ จะโกงเงินค่าอาหาร $V_i$ บาท ออกจากร้านอาหารที่เวลา $P_i$ และหากคุณ (หรือเพื่อนสนิทชั่วนิรันดร์ของคุณ) ออกไปตามเก็บค่าอาหารที่เขาโกงคุณจะกลับมาที่ร้านที่เวลา $K_i$  (และจะพร้อมจับลูกค้าที่ออกจากร้านอาหารที่เวลานั้น)

                
\textbf{Note:} ร้านของคุณมีพนักงานสองคนทำให้พวกคุณสามารถออกไปตามจับลูกค้าได้สองคนในเวลาเดียวกัน

\bigskip
\underline{\textbf{โจทย์}}  จงเขียนโปรแกรมเพื่อรับรายการของคนที่จะมาร้านอาหารของคุณในวันนี้ จงหาจำนวนเงินมากที่สุดที่คุณ (และเพื่อนสนิทชั่วนิรันดร์ของคุณ) จะสามารถเก็บค่าอาหารคืนมาได้


\InputFile

\textbf{บรรทัดแรก} รับจำนวนเต็ม $N$ แทนจำนวนลูกค้า $( 1 \leq N \leq 1\,000 )$

\textbf{บรรทัดที่ $2$ ถึง $N+1$ }แต่ละบรรทัดประกอบด้วย จำนวนเต็ม $P_i$ $K_i$ $V_i$ แทนเวลาการออกจากร้าน เวลาที่คุณกลับมาหากไล่จับลูกค้า และจำนวนเงินที่เขาโกงไป $( 1 \leq P_i \leq K_i \leq 1\,000 ; 1 \leq V_i \leq 10\,000 )$


\OutputFile

\textbf{มีบรรทัดเดียว} ระบุจำนวนเงินที่คุณสามารถทวงคืนมาได้มากที่สุด

\Examples

\begin{example}
\exmp{5
2 5 6520
2 3 7573
3 4 7127
3 4 6662
4 5 8976}{30338}%
\exmp{5
1 3 4782
1 2 783
2 4 3645
2 4 777
1 4 7665}{12447}%
\end{example}


\Source

สรวิทย์  สุริยกาญจน์ ( PS.int )

ศูนย์ สอวน. โรงเรียนมหิดลวิทยานุสรณ์

\end{problem}

\end{document}