\documentclass[11pt,a4paper]{article}

\usepackage{../../templates/style}

\begin{document}

\begin{problem}{หยิบหนังสือ}{standard input}{standard output}{1 second}{16 megabytes}

เนื่องจากคุณมาเข้าแถวสาย คุณจึงโดนทำโทษให้จัดหนังสือในห้องสมุด หนังสือแต่ละเล่มที่คุณจัดล้วนแต่เป็นหนังสือเรียนหนา ๆ น่าเบื่อ  ทันทีที่ถึงเวลาพักเที่ยงและอาจารย์ห้องสมุดของคุณไม่อยู่ คุณจึงคิดเกมหนึ่งขึ้นมาเล่นกับตัวเอง

เกมนี้คุณคิดเองเล่นเองคนเดียว เริ่มต้นจากมีหนังสืออยู่ $n$ เล่มวางเรียงอยู่บนกองเดียวกันกองหนึ่ง หนังสือแต่ละเล่มถูกติดหมายเลขไว้บนสันปก ซึ่งเป็นจำนวนเต็มไม่ติดลบซึ่งมีค่าไม่เกิน $1\,000\,000$

คุณสามารถหยิบหนังสือออกจากได้\textbf{ทีละสามเล่มที่อยู่ติดกัน}  แต้มที่ได้จากการหยิบหนังสือหนึ่งครั้งเกิดจากการนำหมายเลขของหนังสือสองเล่มบนรวมกันแล้วหักด้วยหมายเลขของหนังสือเล่มล่าง  คุณสามารถหยิบหนังสือออกมาได้หลาย ๆ ครั้ง เพื่อทำให้ผลรวมแต้มของการหยิบหนังสือออกมามีค่ามากที่สุดเท่าที่ทำได้  อนึ่งคุณไม่จำเป็นต้องหยิบหนังสือออกจนหมดทั้งกอง
    

\bigskip
\underline{\textbf{โจทย์}} จงเขียนโปรแกรมรับหมายเลขของหนังสือ $n$ เล่มในกองจากบนลงล่าง แล้วให้ตอบผลรวมแต้มที่มากที่สุดที่ทำได้จากการหยิบหนังสือนี้  


\InputFile

 \textbf{บรรทัดที่หนึ่ง} มีจำนวนเต็มบวก $N$ $(1 \leq N \leq 2\,000)$ แทนจำนวนหนังสือในกอง
 
\textbf{บรรทัดที่ $2$ ถึง $N+1$} จะบอกหมายเลขหนังสือในกอง จากบนลงล่าง บรรทัดละหนึ่งจำนวน ซึ่งจำนวนเต็มไม่ติดลบดังกล่าวมีค่าไม่เกิน $1\,000\,000$



\OutputFile

\textbf{มีบรรทัดเดียว} มีจำนวนเต็มจำนวนเดียวแต้มรวมที่มากที่สุดที่เป็นไปได้จากการหยิบหนังสือตามเงื่อนไข

\Examples

\begin{example}
\exmp{7
1
2
3
4
5
6
7}{5}%
\exmp{9
5
5
5
5
5
5
5
5
100}{10}%
\exmp{15
28
40
39
88
37
46
18
25
39
50
67
40
19
74
23}{385}%
\end{example}

  
\Source

วรภัทร จรางกูล

ดัดแปลงจากข้อสอบพี่ช่วยน้อง โรงเรียนมหิดลวิทยานุสรณ์ พ.ศ. 2552


\end{problem}

\end{document}