\documentclass[11pt,a4paper]{article}

\usepackage{../../templates/style}

\begin{document}

\begin{problem}{มด (mravi)}{standard input}{standard output}{1 second}{16 megabytes}

มดซึ่งมีขนาดเล็กมากๆ เดินด้วยอัตราเร็วคงที่ $1$ มม.ต่อวินาที อยู่บนเส้นเชือกตึงยาว แต่เมื่อมดเดินไปเจอกับมดตัวอื่น หรือที่สุดสาย มดตัวนั้นจะหันหน้ากลับไปด้านตรงข้ามและเริ่มเดินต่อไปทันทีด้วยอัตราเร็วคงที่

    เรามีข้อมูลว่า มดแต่ละตัวจะอยู่ ณ ตำแหน่งใดและหันหน้าไปทางใดในตอนเริ่มต้น โดยมดแต่ละตัวจะถูกทำเครื่องหมายไว้ด้วยตัวเลข $1, 2, 3, …, N$ (รวมทั้งสิ้น $N$ ตัว) ไม่มีมดสองตัวใดที่อยู่ที่ตำแหน่งเดียวกันที่เวลาเริ่มต้น

\bigskip
\underline{\textbf{โจทย์}}  จงเขียนโปรแกรมที่คำนวณหาตำแหน่งของมดแต่ละตัว ณ เวลาที่กำหนดให้ค่าหนึ่ง

\InputFile

\textbf{บรรทัดแรก} มีจำนวนเต็มสองจำนวนคือ $L$ (ความยาวของเชือกหน่วยเป็นมม.) และ $T$ (เวลาในหน่วยวินาที) โดยที่ $2 \leq L \leq 200\,000$ และ $1 \leq T \leq 1\,000\,000$ ซึ่งจะคั่นด้วยช่องว่างหนึ่งช่อง

\textbf{บรรทัดที่สอง} มีจำนวนเต็ม $N$ (จำนวนของมด) โดยที่ $1 \leq N \leq 70\,000$ และ $N < L$

\textbf{บรรทัดที่ $3$ ถึง $N+2$} ระบุตำแหน่งเริ่มต้น และทิศทางของมดแต่ละตัว ด้วยจำนวนเต็มหนึ่งจำนวน เป็นระยะทางจากปลายซ้ายสุดของเชือก (มม.) และอักษร ‘L’ หรือ ’D’ แทนมดที่เริ่มต้นหันไปทางซ้ายและขวา ตามลำดับ โดยตำแหน่งของมดจะถูกเรียงลำดับจากซ้ายไปขวาตามหมายเลขของมด


\OutputFile

\textbf{บรรทัดเดียว} ระบุตำแหน่ง(ระยะทางจากปลายซ้ายสุด)ของมดแต่ละตัว จากตัวที่ $1$ ถึงตัวที่ $N$ แต่ละตัวคั้นด้วยช่องว่าง $1$ ช่อง

\Examples

\begin{example}
\exmp{3 5
1
1 D}{0}%
\exmp{5 5
2
2 D
4 L}{1 3}%
\exmp{8 10
5
1 L
3 L
4 D
6 L
7 D}{1 2 4 7 7}%
\end{example}


\Source

Croatian Olympiad in Informatics 2004

\end{problem}

\end{document} 