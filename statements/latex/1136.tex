\documentclass[11pt,a4paper]{article}

\usepackage{../../templates/style}

\begin{document}

\begin{problem}{ขนม (snack)}{standard input}{standard output}{1 second}{32 megabytes}

หลังกลับจากต่างจังหวัด เพื่อนของคุณได้ซื้อขนมมาฝากคุณ $2$ ถุงใหญ่ แต่เนื่องจากเพื่อนของคุณเป็นนักคณิตศาสตร์ จะซื้อขนมมาให้ใครฟรีๆ ก็กระไรอยู่ เขาจึงท้าให้คุณเล่นเกมย้ายขนมให้ชนะเสียก่อน จึงจะได้รับขนมทั้งหมดเป็นของฝาก

เกมนี้มีกติกาว่า ในตอนเริ่มต้นจะมีถุงอยู่ $2$ ใบ แต่ละใบบรรจุขนมที่มีราคาต่างๆ อยู่จำนวนมาก คุณสามารถย้ายขนม $1$ ชิ้นจากถุงใบหนึ่งไปยังอีกใบหนึ่ง ให้สอดคล้องเงื่อนไขต่อไปนี้

\begin{itemize}

\item หลังจากย้ายแล้ว ถุงแต่ละใบต้องจะมีขนมเหลืออยู่อย่างน้อย $1$ ชิ้น
\item หลังจากย้ายแล้ว ค่าเฉลี่ยของราคาขนมในถุงแต่ละใบจะต้องมีค่าเพิ่มขึ้นจากเดิม
\end{itemize}

การย้ายขนมแต่ละครั้งสามารถย้ายจากถุงใบไหนไปใบไหนก็ได้ แต่สามารถย้ายได้ครั้งละ $1$ ชิ้นเท่านั้น คุณต้องย้ายขนมให้ได้มากครั้งที่สุดเท่าที่จะทำได้ โดยที่การย้ายขนมทุกครั้งจะต้องสอดคล้องตามเงื่อนไขเสมอ

\textbf{ตัวอย่างเช่น} หากในตอนเริ่มต้นถุงใบแรกมีขนมราคา $4, 6, 10, 15$ และ $20$ บาท ซึ่งมีค่าเฉลี่ยเป็น $11$ บาท ส่วนถุงใบที่สองมีขนมราคา $3$ และ $5$ บาท ซึ่งมีค่าเฉลี่ยเป็น $4$ บาท คุณสามารถย้ายขนมราคา $10$ บาท จากถุงใบแรกไปยังใบที่สองได้ ซึ่งจะทำให้ค่าเฉลี่ยของราคาขนมในถุงใบแรกเพิ่มเป็น $11.25$ บาท และใบที่สองเพิ่มเป็น $6$ บาท แต่หลังจากนั้น คุณจะไม่สามารถย้ายขนมชิ้นใดๆ ให้สอดคล้องตามเงื่อนไขได้อีกแล้ว ดังนั้น คุณจึงย้ายขนมได้เพียงครั้งเดียว

แต่วิธีดังกล่าวยังไม่ใช่วิธีที่ดีที่สุด หากคุณเริ่มต้นด้วยการย้ายขนมราคา $6$ บาท จากถุงใบแรกไปยังใบที่สอง ค่าเฉลี่ยของราคาขนมในถุงใบแรกจะเพิ่มเป็น $12.25$ บาท และใบที่สองจะเพิ่มเป็น $4.67$ บาท จากนั้นคุณจึงย้ายขนมราคา $10$ บาท จากถุงใบแรกไปยังใบที่สอง ทำให้ค่าเฉลี่ยของราคาขนมในถุงใบแรกเพิ่มเป็น $13$ บาท และใบที่สองเพิ่มเป็น $6$ บาท คุณจึงย้ายขนมได้ทั้งหมด $2$ ครั้ง ซึ่งวิธีนี้เป็นวิธีที่ทำให้ย้ายขนมตามเงื่อนไขได้มากครั้งที่สุดแล้ว

\bigskip
\underline{\textbf{โจทย์}}  จงเขียนโปรแกรมเพื่อรับราคาของขนมในแต่ละถุงตอนเริ่มต้น แล้วคำนวณหาจำนวนครั้งในการย้ายขนมตามเงื่อนไขที่มากที่สุดที่เป็นไปได้


\InputFile

\textbf{บรรทัดแรก} ระบุจำนวนเต็ม $A$ และ $B$ $(1 \leq  A,B \leq 1\,000\,000)$ แทนจำนวนขนมตอนเริ่มต้นในถุงใบแรกและใบที่สอง

\textbf{บรรทัดที่สอง} ระบุจำนวนเต็มบวก $A$ ตัว แทนราคาของขนมแต่ละชิ้นในถุงใบแรก โดยเรียงจากน้อยไปหามาก

\textbf{บรรทัดที่สาม} ระบุจำนวนเต็มบวก $B$ ตัว แทนราคาของขนมแต่ละชิ้นในถุงใบที่สอง โดยเรียงจากน้อยไปหามาก

รับประกันว่าขนมแต่ละชิ้นจะมีราคาไม่เกิน $1\,000\,000$ บาท และขนมแต่ละชิ้นอาจมีราคาเท่ากันได้



\OutputFile

\textbf{มีบรรทัดเดียว} แทนจำนวนครั้งในการย้ายขนมตามเงื่อนไขที่มากที่สุดที่เป็นไปได้

\Examples

\begin{example}
\exmp{5 2
4 6 10 15 20
3 5}{2}%
\exmp{4 3
4 6 15 20
3 5 10}{0}%
\end{example}

\Scoring 

\textbf{$20$\% ของข้อมูลทดสอบ:} $A,B \leq 10$

\textbf{$40$\% ของข้อมูลทดสอบ:} $A,B \leq 1\,000$

\textbf{$60$\% ของข้อมูลทดสอบ:} $A,B \leq 100\,000$


\Source

สุธี เรืองวิเศษ

\end{problem}

\end{document}