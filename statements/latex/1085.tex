\documentclass[11pt,a4paper]{article}

\usepackage{../../templates/style}

\begin{document}

\begin{problem}{ล่าสมบัติทั่วทุกทิศ (explore)}{standard input}{standard output}{1 second}{32 megabytes}

ในที่สุด หลังจากลุ้นตัวโก่ง คุณก็สามารถผ่านการตอบปัญหาไปได้ ขั้นต่อมาก็คือ ตามหาอัญมณีนี้ในโบราณสถานอันกว้างใหญ่
         
          คุณรู้มาว่า อัญมณีชิ้นนี้ถูกซ่อนอยู่ในโบราณสถานซึ่งมีลักษณะเป็นห้องๆซึ่งมีทั้งหมด $n$ ห้อง คือห้องที่ $1,2,3,…,n$ ซึ่งห้องที่อยู่ติดกันจะมีประตูหากันได้ กล่าวคือ ห้องที่ $1$ จะมีประตูเชื่อมกับห้องที่ $2$ , ห้องที่ $2$ จะมีประตูเชื่อมกับห้องที่ $3$ , … , ห้องที่ $n-1$ จะมีประตูเชื่อมกับห้องที่ $n$

          \textbf{โดยประตูนั้นจะเป็นประตูทางเดียว} กล่าวคือ จะไม่สามารถใช้เดินทางจากห้อง $i+1$ ไปยัง ห้อง $i$ ได้ แต่จะสามารถใช้เดินทางได้เฉพาะการเดินทางจากห้อง $i$ ไปยัง ห้อง $i+1$ (เมื่อ $1 \leq i < n$)

          เนื่องจากอัญมณีดังกล่าวนั้นมีมูลค่าสูงจนประเมินไม่ได้ มันจึงถูกซ่อนอยู่ในห้องที่ $n$ ยิ่งไปกว่านั้นแล้ว เทพเจ้าที่รักษาสถานที่นี้ไม่ต้องการให้คุณได้อัญมณีล้ำค่าไปง่ายๆจึงสร้างหินมากั้นห้องบางห้องไว้ทำให้คุณไม่สามารถเดินไปต่อยังห้องต่อไปได้

          แต่ยังโชคดีที่คุณค้นพบระบบขนย้ายมวลสารในโบราณสถานนี้ กล่าวคือคุณสามารถข้ามจากห้องที่มีเครื่องขนย้ายมวลสารไปยังห้องปลายทางที่เครื่องกำหนดได้ ซึ่งเครื่องนี้มีลักษณะพิเศษคือ จะเคลื่อนย้ายคุณไปยังห้องที่มีเลขสูงกว่าเท่านั้น คุณจะได้รับข้อมูลของโบราณสถานทั้งหมดแล้วจงหาว่า คุณสามารถไปถึงห้องที่มีอัญมณีได้หรือไม่ ถ้าไม่ได้ คุณสามารถไปยังห้องที่มีเลขห้องมากสุดเท่าใด โดยเริ่มต้นคุณอยู่ที่ห้องหมายเลข $1$

\bigskip
\underline{\textbf{โจทย์}}  จงเขียนโปรแกรมรับข้อมูลของโบราณสถานนี้ และตอบว่า คุณสามารถไปยังห้องที่มีอัญมณีได้หรือไม่ ถ้าไม่ได้ คุณสามารถไปยังห้องที่มีเลขห้องมากสุดเท่าใด

\InputFile

\textbf{บรรทัดแรก} มีจำนวนเต็มบวก $n, m, k$ แทนจำนวนห้องทั้งหมด จำนวนเครื่องขนย้ายมวลสาร และ จำนวนหินกั้นทางตามลำดับ โดยที่ $1 \leq n \leq 500\,000; 0 \leq m \leq 1\,000\,000; 0 \leq k \leq n-1$

\textbf{บรรทัดที่ $2$ ถึง $m+1$} แต่ละบรรทัดประกอบด้วยจำนวนเต็ม $2$ จำนวน $a, b$ หมายความว่า ในห้อง $a$  มีเครื่องขนย้ายมวลสาร ที่สามารถใช้เดินทางไปยังห้อง $b$ ได้

\textbf{บรรทัดที่ $m+2$ ถึง $m+k+1$} แต่ละบรรทัดมีจำนวนเต็ม $x$ ซึ่งบอกว่ามีหินกั้น ระหว่างห้อง $x$ กับ $x + 1$


\OutputFile

\textbf{มีบรรทัดเดียว} ถ้าสามารถเก็บอัญมณีได้ให้พิมพ์ $1$ ถ้าเก็บอัญมณีไม่ได้ให้พิมพ์ $0$ แล้วตามด้วยหมายเลขห้องที่มีค่าสูงสุดที่ไปถึงได้

\Examples

\begin{example}
\exmp{5 1 1 
2 5
2}{1}%
\exmp{5 1 1
1 2
2}{0 2}%
\end{example}


\Source

วีระกานต์ สินทวีเลิศมงคล

\underline{\href{http://www.thailandoi.org/toi.c/03-2009}{TOI.CPP:03-2009}}


\end{problem}

\end{document}