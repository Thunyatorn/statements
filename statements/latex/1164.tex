\documentclass[11pt,a4paper]{article}

\usepackage{../../templates/style}

\begin{document}

\begin{problem}{ระเบิดมหาประลัย (Bomb)}{standard input}{standard output}{2 second}{32 megabytes}

ทหารนาวิกโยธินกำลังต้องการที่จะบุกเข้าไปชิงตัวประกันออกมาจากสถานที่ลับแห่ง หนึ่ง ในการที่จะบุกเข้าไปในที่แห่งนี้ ทหารนาวิกโยธินจะต้องผ่านเหมืองระเบิด โดยในเหมืองระเบิดนี้จะมีทั้งระเบิดจริงและระเบิดปลอมอยู่ทั้งหมดจำนวน $n$ ตำแหน่งที่ไม่ซ้ำกัน คือ $\{p_1,p_2,p_3,…,p_n\}$ โดยที่ $p_i = (x_i,y_i)$ เป็นพิกัดของระเบิด หน่วยข่าวกรองของทหารทราบมาว่า ระเบิดจริงจะอยู่ในตำแหน่งที่มีลักษณะพิเศษที่เรียกว่าตำแหน่งมหันตภัย ซึ่งลักษณะพิเศษดังกล่าวถูกระบุตามเงื่อนไขดังต่อไปนี้

\begin{enumerate}

\item ศัพท์ทางการทหารกล่าวว่าตำแหน่ง $p_1$ บดบังตำแหน่ง $p_2$ ก็ต่อเมื่อ $x_1 > x_2$ และ $y_1 > y_2$
\item ตำแหน่งมหันตภัยคือ ตำแหน่งที่ไม่มีตำแหน่งอื่นๆ บดบัง
\end{enumerate}


\bigskip
\underline{\textbf{โจทย์}}  จงเขียนโปรแกรมที่มีประสิทธิภาพในการระบุตำแหน่งมหันตภัยที่มีระเบิดจริงทั้งหมด


\InputFile

\textbf{บรรทัดแรก} ระบุค่าของตัวแปร $n$ โดยที่ $1 \leq n \leq 1\,000
\,000$

\textbf{บรรทัดที่ $2$ ถึง $n+1$} ระบุตำแหน่งของระเบิดทั้งหมด แต่ละบรรทัดระบุค่าของตำแหน่งเป็นจำนวนเต็มบวกสองตัว $x$ และ $y$ โดยมีช่องว่างคั่นอยู่ระหว่างตัวเลขทั้งสอง โดยที่ $1 \leq x, y \leq 10\,000\,000$


\OutputFile

\textbf{มีหลายบรรทัด} ให้ระบุตำแหน่งมหันตภัยทั้งหมด โดยให้แต่ละบรรทัดระบุค่าของตำแหน่งเป็นจำนวนเต็มบวกสองตัว $x$ และ $y$ โดยมีช่องว่างคั่นอยู่ การเรียงก่อนหลังของตำแหน่งให้จากค่าพิกัด $x$ จากน้อยไปมาก หากพิกัดคู่ใดมีค่าพิกัด $x$ เท่ากัน ให้เรียงตามพิกัด $y$ จากมากไปน้อย

\textbf{หมายเหตุ:} แนะนำให้ใช้ scanf ในการรับค่าและ printf ในการแสดงผล

\Examples

\begin{example}
\exmp{5
9 1
8 2
7 3
6 4
5 5}{5 5
6 4
7 3
8 2
9 1}%
\exmp{7
1 2
2 4
4 1
7 3
5 5
6 6
3 7}{3 7
6 6
7 3}%
\end{example}


\Source

การแข่งขันคอมพิวเตอร์โอลิมปิกระดับชาติครั้งที่ 7 (NUTOI7) :: ดัดแปลงเล็กน้อย

\end{problem}

\end{document}