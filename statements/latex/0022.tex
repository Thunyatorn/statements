\documentclass[11pt,a4paper]{article}

\usepackage{../../templates/style}

\begin{document}

\begin{problem}{ดาว (star)}{standard input}{standard output}{1 second}{32 megabytes}

 ในยุคของการแข่งขันนี้ คุณถูกกดดันจาก เสียงคีย์บอร์ดรอบตัว Ranking อันดับสูง ๆ ในเว็บโปรแกรมมิ่ง ทำให้คุณเหลือเพียงทางเลือกเดียวเท่านั้น...

            เวลาเหลืออยู่ไม่มากแล้ว คุณได้พยายามเจาะเข้าสู่ระบบฐานข้อมูลของ TOI.C โดยทันที เป้าหมายคือการค้นหาไฟล์ข้อสอบ TOI.C ที่จะต้องเผชิญในวันพรุ่งนี้ แต่แล้วคุณก็ต้องพบเข้ากับปราการด่านแรก !

            ระบบรักษาความปลอดภัยของ TOI.C ทำงานเข้าเสียแล้ว สิ่งเดียวที่จะยับยั้งมันได้ ก่อนที่จะมีคนพบว่าคุณกำลังเจาะเข้าสู่ระบบนั้น คือการกรอกรหัสรูปดาวของทีมงาน ที่ใช้หยุดสถานการณ์ฉุกเฉินของระบบได้

            คุณตัดสินใจเขียนโปรแกรมที่จะแสดงผลดาว (ดอกจัน) ที่เรียงกันเป็นรูปสี่เหลี่ยมขนมเปียกปูนในคอมพิวเตอร์ เพื่อที่จะหยุดการทำงานของระบบรักษาความปลอดภัย แต่เนื่องจากจอมอนิเตอร์ของคุณเป็นจอขนาดยาวพิเศษ ที่แสดงตัวอักษรได้ถึง $N$ บรรทัด โปรแกรมของคุณจึงต้องแสดงผลดาวออกมาได้ $N$ บรรทัดพอดี (โปรดดูตัวอย่างข้อมูลนำเข้า)

\underline{\textbf{โจทย์}} จงเขียนโปรแกรมที่รับจำนวนบรรทัดของจอมอนิเตอร์ และแสดงผลลัพธ์เป็นรูปดาวตามเงื่อนไข

\InputFile

\textbf{มีบรรทัดเดียว} มีจำนวนเต็ม $N$ $(1 \leq N \leq 1\,000)$ แสดงจำนวนบรรทัดของจอมอนิเตอร์

รับประกันว่า ในข้อมูลทดสอบทั้งหมดจะมีผู้ชนะเพียงคนเดียวเสมอ

\OutputFile

\textbf{มี $N$ บรรทัด} แสดงเป็นรูปดาวลักษณะขนมเปียกปูนตามที่โจทย์ต้องการ (ดาวจะใช้เครื่องหมายดอกจัน \textbf{“ * ”} ส่วนที่เหลือเป็นอักขระขีดกลาง hyphen \textbf{“ - ”})

\Examples

\begin{example}
\exmp{4}{	-*-
*-*
*-*
-*-}%
\exmp{5}{--*--
-*-*-
*---*
-*-*-
--*--}%
\exmp{6}{	--*--
-*-*-
*---*
*---*
-*-*-
--*--}%
\end{example}


\Source

แต่งโดย: ทักษพร กิตติอัครเสถียร 

\href{http://thailandoi.org/toi.c/01-2009}{TOI.C:01-2009} 

\end{problem}

\end{document}