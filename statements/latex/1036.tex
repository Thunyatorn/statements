\documentclass[11pt,a4paper]{article}

\usepackage{../../templates/style}

\begin{document}

\begin{problem}{Equipped}{standard input}{standard output}{1 second}{64 megabytes}

แดงเตรียมตัวไปตั้งแค้มป์ในป่าเขาดงดิบกับเพื่อน ๆ เขาไปเดินเลือกซื้ออุปกรณ์ที่ห้างสรรพสินค้าโชว์ห่วย ในร้านมีอุปกรณ์ตั้งแค้มป์ $n$ ชิ้น ผลิตภัณฑ์ชิ้นที่ $i$ มีราคา $w_i$ บาท

แดงต้องการอุปกรณ์เหล่านี้ เพื่อใช้งานหลายอย่าง เช่น เหลาไม้ ขุดดิน ฟังเพลง เลื่อยไม้ กรองน้ำ ถลุงเหล็ก โม่แป้ง เป็นต้น รวมการใช้งานทั้งหมดมีได้ $k$ แบบ

แดงมีข้อมูลว่าอุปกรณ์แต่ละชิ้นทำอะไรได้บ้าง โดยสำหรับอุปกรณ์ที่ $i$ และการใช้งานที่ $j$ ค่า $p(i,j)$ จะระบุว่า อุปกรณ์ดังกล่าวมีความสามารถใชงานสำหรับงานที่ j หรือไม่ กล่าวคือ $p(i,j) = 1$ เมื่ออุปกรณ์ที่ $i$ สามารถทำงาน $j$ ได้ และ $p(i,j) = 0$ เมื่ออุปกรณ์ชิ้นที่ $i$ ทำไม่ได้

ช่วยแดงเลือกเซตของอุปกรณ์ที่จะซื้อเพื่อให้สามารถใช้งานทำงานทุกงานได้ครบ กล่าวคือ สำหรับการใช้งาน $j$ ใดๆ จะต้องมีอุปกรณ์ที่เลือกไปอย่างน้อย $1$ อย่างที่สามารถใช้ทำงาน $j$ ได้ นอกจากนี้ให้เลือกโดยใช้เงินน้อยที่สุดด้วย

\bigskip
\underline{\textbf{โจทย์}}  จงเขียนโปรแกรมรับข้อมูลสินค้าแต่ละชิ้น แล้วแสดงผลเป็นจำนวนเงินที่น้อยที่สุดที่สามารถซื้ออุปกรณ์จำนวนหนึ่งที่สำหรับงานใดๆจะต้องมีหนึ่งในอุปกรณ์ที่ซื้อมาที่สามารถทำงานนั้นได้

\InputFile

\textbf{บรรทัดแรก} รับจำนวนเต็ม $n$  $k$ $(1 \leq n <= 10\,000; 1 \leq k \leq 8$) 

\textbf{บรรทัดที่ $2$ ถึง $n+1$} ในบรรทัดที่ $i+1$ จะรับข้อมูลของอุปกร์ชิ้นที่ $i$ โดยจะรับจำนวนเต็ม $k+1$ จำนวนเรียงตามลำดับดังนี้ $w_i$ $p(i,1)$ $p(i,2)$ $. . .$ $p(i,k)$ แต่ละค่าคั่นด้วยช่องว่าง $1$ ช่อง

\OutputFile
\textbf{มีบรรทัดเดียว} เป็น
จำนวนเงินที่น้อยที่สุดที่สามารถซื้อของที่ทำงานได้ครบทุกอย่าง

\Examples

\begin{example}
\exmp{5 3
10 1 0 1
30 0 1 1
5 1 0 0
4 0 0 1
150 1 1 1}{35}%
\end{example}


\Source

อ.ดร.จิตร์ทัศน์ ฝักเจริญผล

\end{problem}

\end{document}