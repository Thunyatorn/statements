\documentclass[11pt,a4paper]{article}

\usepackage{../../templates/style}

\begin{document}

\begin{problem}{TREZOR}{standard input}{standard output}{3 second}{32 megabytes}

Mirko ตัดสินใจที่จะเปิดตัวธุรกิจใหม่ คือ ห้องนิรภัยในธนาคาร   สาขาของธนาคารสามารถมองเห็นได้ในแนวระนาบและห้องนิรภัยก็สามารถกำหนดจุดในแนวระนาบได้เช่นกัน   สาขาของ Mirko ประกอบด้วย ห้องนิรภัยจำนวน $L\times (A+B+1)$ ห้องเท่านั้น   ดังนั้นในแต่ละจุดซึ่งเป็นพิสัยของเลขจำนวนเต็มภายในสี่เหลี่ยมผืนผ้าที่มี $(1, −A)$ และ $(L, B)$ เป็นด้านมุมจะมีห้องนิรภัยเพียง $1$ ห้องเท่านั้น

พนักงานรักษาความปลอดภัย $2$ คนสามารถมองเห็นห้องนิรภัยเหล่านั้นได้ โดยพนักงานคนหนึ่งยืนอยู่ที่ตำแหน่ง $(0, -A)$ และอีกคนหนึ่งยืนอยู่ที่ $(0, B)$   พนักงานรักษาความปลอดภัย $1$ คนสามารถมองเห็นห้องนิรภัย $1$ ห้องได้ ถ้าไม่มีห้องนิรภัยห้องอื่นบนแนวเดียวกันเชื่อมต่อกัน

ถ้าไม่มีพนักงานรักษาความปลอดภัยคนใดสามารถมองเห็นห้องนิรภัยได้ จะถือว่าห้องนิรภัยนี้ไม่มีความปลอดภัย แต่ถ้ามีพนักงาน $1$ คนสามารถมองเห็นได้ จะถือว่าห้องนิรภัยนี้มีความปลอดภัย และถ้าพนักงานรักษาความปลอดภัยทั้งสองคนสามารถมองเห็นห้องนิรภัยนี้ได้ จะถือว่าห้องนิรภัยนี้มีความปลอดภัยสูงมาก
    

\bigskip
\underline{\textbf{โจทย์}}  จงเขียนโปรแกรมเพื่อรับค่า $A, B$ และ $L$ แล้วให้แสดงผลจำนวนของห้องนิรภัยที่ไม่มีความปลอดภัย   ห้องนิรภัยที่มีความปลอดภัย   และห้องนิรภัยที่มีความปลอดภัยสูงมาก ตามลำดับ


\InputFile

\textbf{บรรทัดแรก}  ประกอบด้วยเลขจำนวนเต็ม $A$ และ $B$ ซึ่งแยกกันด้วยช่องว่าง โดย $A$ และ $B$ มีค่าดังนี้  $1 \leq A \leq 2\,000$ และ $1 \leq B \leq 2\,000$

\textbf{บรรทัดที่สอง}   ประกอบด้วยเลขจำนวนเต็มหนึ่งจำนวน $L$ ซึ่ง $1 \leq L \leq 1\,000\,000\,000$


\OutputFile

\textbf{มีสามบรรทัด} แต่ละบรรทัดให้แสดง จำนวนของห้องนิรภัยที่ไม่มีความปลอดภัย   ห้องนิรภัยที่มีความปลอดภัย   และห้องนิรภัยที่มีความปลอดภัยสูงมาก ตามลำดับ

\Examples

\begin{example}
\exmp{1 1
3}{2
2
5}%
\exmp{2 3
4}{0
16
8}%
\exmp{7 11
1000000}{6723409
2301730
9974861}%
\end{example}

\Scoring 

\textbf{50\%ชองชุดทดสอบทั้งหมด:} $L \leq 1\,000$

\textbf{75\%ชองชุดทดสอบทั้งหมด:} $A,B \leq 100$  (แต่ $L$ สามารถมีค่าได้มากถึง $1\,000\,000\,000$)
  
\Source


\end{problem}

\end{document}