\documentclass[11pt,a4paper]{article}

\usepackage{../../templates/style}

\begin{document}

\begin{problem}{รวมอนุภาค (atom)}{standard input}{standard output}{1 second}{16 megabytes}

อนุภาคแบบสั่งทำพิเศษจำนวน $N$ อนุภาควางเรียงกัน เราจะเรียกอนุภาคดังกล่าวว่าอนุภาคที่ $1, 2, ...,$ ถึงอนุภาคที่ $N$ ตามลำดับ อนุภาคแต่ละอนุภาคจะมีค่าพลังงานสะสมอยู่ กล่าวคืออนุภาคที่ $i$ จะมีพลังงานสะสมเท่ากับ $X_i$ หน่วย

                อนุภาคสองอนุภาคใด ๆ เมื่อนำมาชนกัน จะสลายตัวและปล่อยพลังงานออกมา โดยพลังงานที่ปล่อยออกมานั้น มีค่าเท่ากับผลต่างของพลังงานสะสมของอนุภาคทั้งสอง

                หัวหน้าห้องปฏิบัติการวานให้คุณทดลองนำาอนุภาคทั้ง $N$ อันมาชนกัน โดยหัวหน้าได้ระบุคำสั่งไว้ดังนี้ 
\begin{enumerate}
                
\item ให้เลือกอนุภาคสองอนุภาคที่ติดกัน ที่ชนกันแล้วสลายตัวให้พลังงานมากที่สุด ถ้ามีหลายทางเลือกให้เลือกคู่ของอนุภาคที่ประกอบด้วยอนุภาคที่มีหมายเลขน้อยที่สุด 
\item นำอนุภาคทั้งสองมาชนกัน ทำไปเรื่อยๆ จนกระทั่งอนุภาคหมด หรือเหลือแค่ $1$ อนุภาค (ไม่สามารถชนกับใครได้อีก)
\end{enumerate}

                สังเกตว่าเมื่ออนุภาคชนกันแล้วจะสลายไปทั้งคู่ ทำให้อนุภาคคู่อื่น ๆ ที่เมื่อเริ่มต้นไม่ได้มีตำแหน่งติดกัน มีลำดับอยู่ติดกันได้
\bigskip

\textbf{ตัวอย่าง}

สมมติมีอนุภาค $7$ อนุภาคที่มีพลังงานสะสมดังนี้: $1$ \underline{$2$ $4$} $3$ $1$ $2$ $3$

คุณเลือกชนอนุภาคที่ $2$ กับ $3$ (สังเกตว่า คู่ของอนุภาค $3$ กับ $1$ ก็มีผลต่างเท่ากับ $2$ เหมือนกัน แต่เราไม่เลือกเนื่องจากอนุภาคที่ $2$ มีหมายเลขน้อยกว่า) ได้พลังงาน $2$ หน่วย หลังจากนั้น เราจะเหลืออนุภาค $5$ อนุภาค: \underline{$1$ $3$} $1$ $2$ $3$

เลือกคู่อนุภาค $1$ กับอนุภาค $4$ ได้พลังงาน $2$ หน่วย จะเหลืออนุภาค $3$ อนุภาค: \underline{$1$ $2$} $3$

เลือกคู่อนุภาค $5$ กับอนุภาค $6$ ได้พลังงาน $1$ หน่วย จะเหลืออนุภาค $1$ อนุภาค: $3$

                เมื่อเหลืออนุภาคเดียวเราจะไม่สามารถชนได้อีก รวมแล้วได้พลังงานทั้งหมด 5 หน่วย

\bigskip
\underline{\textbf{โจทย์}}  รับข้อมูลพลังงานสะสมของอนุภาค จากนั้นคำนวณพลังงานทั้งหมดที่ได้รับจากการชนอนุภาคด้วยวิธีการตามที่หัวหน้าห้องปฏิบัติการระบุ

\newpage
\InputFile

\textbf{บรรทัดแรก} ระบุจำนวนเต็ม $N$ $(1 \leq N \leq 1\,000)$ แทนจำนวนอนุภาค

\textbf{บรรทัดที่ $2$ ถึง $N+1$} ระบุพลังงานสะสมของแต่ละอนุภาค กล่าวคือ บรรทัดที่ $i+1$ จะระบุจำนวนเต็ม $X_i$ $(1 \leq X_i \leq 1\,000\,000)$ แทนพลังงานสะสมของอนุภาคที่ $i$

\OutputFile

\textbf{มีบรรทัดเดียว} แสดงพลังงานรวมทั้งหมดที่ได้รับ

\Examples

\begin{example}
\exmp{7
1
2
4
3
1
2
3}{5}%
\end{example}


\Source

การแข่งขัน YTOPC กุมภาพันธ์ 2552

\end{problem}

\end{document}