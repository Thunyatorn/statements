\documentclass[11pt,a4paper]{article}

\usepackage{../../templates/style}

\begin{document}

\begin{problem}{MJEHURIC}{standard input}{standard output}{1 second}{32 megabytes}

Goran มีท่อนไม้ $5$ ชิ้นเรียงกันตามลำดับ ท่อนไม้แต่ละชิ้นจะมีตัวเลขที่มีค่าจาก $1-5$ สลักอยู่ด้านบน ดังนั้น ค่าของตัวเลขที่กำกับอยู่ด้านบนของท่อนไม้ในแต่ละชิ้นจะปรากฏแค่เพียงครั้งเดียวเท่านั้น

Goran ต้องการที่จะเรียงลำดับท่อนไม้แต่ละชิ้นให้มีค่า $1$ $2$ $3$ $4$ $5$ ตามลำดับ โดยกระทำด้วยวิธีการดังนี้
\begin{enumerate}  
\item ถ้าตัวเลขบนไม้ท่อนแรกมีค่ามากกว่าตัวเลขบนไม้ท่อนที่สอง   ให้สลับที่กัน
\item ถ้าตัวเลขบนไม้ท่อนที่สองมีค่ามากกว่าตัวเลขบนไม้ท่อนที่สาม   ให้สลับที่กัน
\item ถ้าตัวเลขบนไม้ท่อนที่สามมีค่ามากกว่าตัวเลขบนไม้ท่อนที่สี่   ให้สลับที่กัน
\item ถ้าตัวเลขบนไม้ท่อนที่สี่มีค่ามากกว่าตัวเลขบนไม้ท่อนที่ห้า   ให้สลับที่กัน
\item ถ้าท่อนไม้ทั้งหมดยังไม่เรียงกันตามลำดับ 1 2 3 4 5   ให้กลับไปทำขั้นที่ 1 ใหม่

\end{enumerate}

\bigskip
\underline{\textbf{โจทย์}}  จงเขียนโปรแกรมเพื่อรับค่าของตัวเลขบนท่อนไม้ตามลำดับในตอนเริ่มต้น และแสดงผลค่าของตัวเลขบนท่อนไม้ตามลำดับที่ได้หลังจากการสลับที่ในแต่ละครั้ง


\InputFile

\textbf{มีบรรทัดเดียว}   ประกอบด้วยค่าของตัวเลขบนท่อนไม้ทั้ง $5$ ท่อน ตามลำดับ ซึ่งแยกกันโดยใช้ช่องว่าง $1$ ช่อง ตัวเลขทั้งหมดจะต้องมีค่าจาก $1$ ถึง $5$ และไม่มีตัวเลขใดมีค่าซ้ำกัน

ค่าของตัวเลขบนท่อนไม้ในตอนเริ่มต้น จะต้องไม่ใช่ $1$ $2$ $3$ $4$ และ $5$ ตามลำดับ


\OutputFile

\textbf{มีหลายบรรทัด} หลังจากท่อนไม้ $2$ ท่อนใด ๆถูกสลับที่กัน   ให้แสดงผลค่าของตัวเลขบนท่อนไม้เหล่านั้นซึ่งเรียงกันตามลำดับลงบนบรรทัดเดียวกัน โดยใช้ช่องว่างแยกตัวเลขแต่ละตัวออกจากกัน

\Examples

\begin{example}
\exmp{2 1 5 3 4}{1 2 5 3 4
1 2 3 5 4
1 2 3 4 5}%
\exmp{2 3 4 5 1}{2 3 4 1 5
2 3 1 4 5
2 1 3 4 5
1 2 3 4 5}%
\end{example}

  
\Source

COCI 2008/2009, Contest \#4 – January 17, 2009

\end{problem}

\end{document}