\documentclass[11pt,a4paper]{article}

\usepackage{../../templates/style}

\begin{document}

\begin{problem}{Plate}{standard input}{standard output}{1 second}{64 megabytes}

โรงเรียนประจำแห่งหนึ่ง นักเรียนจะต้องเข้าแถวเพื่อรับถาดอาหารกลางวันตั้งแต่เวลา 12:00 น. ของทุกวัน คุณครูจะให้นักเรียนทุกชั้นเข้าแถวเดียวกัน โดยมีระเบียบในการเข้าแถวอยู่ดังนี้

\textbf{เมื่อเริ่มต้น} นักเรียนคนใดมาก่อนคุณครูจะให้ยืนที่หัวแถว และนักเรียนคนที่จะมาเข้าแถวคนต่อไปคุณครูจะสำรวจตำแหน่งของนักเรียนโดยเริ่มต้นจากหัวแถว และจะแทรกนักเรียนคนนั้นเข้าไปในตำแหน่งต่อจากคนสุดท้ายของนักเรียนในชั้นเรียนเดียวกัน แต่ถ้าไม่มีนักเรียนในชั้นเดียวกันอยู่ในแถวคุณครูจะให้นักเรียนคนนั้นไปต่อที่ท้ายแถว

\textbf{การออกจากแถวเพื่อไปรับถาดอาหาร} นักเรียนที่อยู่หัวแถวที่ได้ออกจากแถวก่อนและให้แสดงเลขประจำตัว นักเรียนที่ได้รับถาดอาหารแล้วไม่สามารถกลับเข้ามาในแถวได้อีก

ทั้งนี้คุณครูจะทราบเลขประจำตัวและชั้นเรียนของนักเรียนทุกคนอยู่แล้ว และนักเรียนทุกคนจะมีเลขประจำตัวไม่ซ้ำกัน นักเรียนบางคนอาจไม่ได้ถูกเรียกมาเข้าแถว และนักเรียนบางคนอาจจะไม่ได้ออกจากแถว


\underline{\textbf{โจทย์}} จงเขียนโปรแกรมเพื่อจัดแถวเข้ารับถาดอาหารตามระเบียบของโรงเรียนแห่งนี้ แล้วแสดงลำดับการรับถาดอาหารของนักเรียน \textbf{กรณีที่ไม่มีนักเรียนในแถวให้เขียนข้อความว่า “empty”} ถ้ามีให้แสดงเลขประจำตัวนักเรียนที่ได้ออกจากแถว


\InputFile
\textbf{บรรทัดแรก} มีจำนวนเต็มสองตัว $N_c$ และ $N_s$ แทนจำนวนชั้นเรียนและจำนวนนักเรียนตามลำดับ โดยที่ $1 \leq N_c \leq 10$ และ $1 \leq N_s \leq 1,000$ ตัวเลขทั้งสองถูกคั่นด้วยช่องว่างหนึ่งช่อง

\textbf{บรรทัดที่ $2$ ถึง $N_s + 1$} เก็บรายละเอียดของนักเรียนแต่ละคนด้วยจำนวนเต็มสองค่าคั่นด้วยช่องว่างหนึ่งช่อง คือ $c$ และ $s$ ซึ่งแทน หมายเลขชั้นเรียนและเลขประจำตัวของนักเรียนตามลำดับ โดยที่ $1 \leq c \leq N_c$ และ $1 \leq s \leq 10,000$

การมาเข้าแถวและการนำนักเรียนออกจากหัวแถวเพื่อไปรับถาดอาหาร แต่ละบรรทัดจะมีรูปแบบคำสั่ง ดังต่อไปนี้:
\begin{itemize}

\item \textbf{$E$ $id$} เป็นการนำนักเรียนที่มีเลขประจำตัว \textbf{$id$} มาเข้าแถว

\item \textbf{$D$} เป็นการนำนักเรียนที่อยู่ที่หัวแถวออกจากแถว

\item \textbf{$X$} เป็นการระบุว่าเป็นคำสั่งสุดท้าย
\end{itemize}
\OutputFile

\textbf{มีหลายบรรทัด} แต่ละบรรทัดแสดงเลขประจำตัวนักเรียนที่ถูกนำออกจากแถวเพื่อรับถาดอาหารตามลำดับ โดยบรรทัดสุดท้ายให้ใส่จำนวนเต็มศูนย์

\Examples

\begin{example}
\exmp{2 6
1 41
1 42
1 43
2 201
2 202
2 203
E 41
E 201
D
E 202
E 42
E 43
D
E 203
D
D
D
X}{41
201
202
203
42
0}%
\end{example}

\Source

การแข่งขันคอมพิวเตอร์โอลิมปิก สอวน. ครั้งที่ 1 มหาวิทยาลัยเกษตรศาสตร์

\end{problem}

\end{document}