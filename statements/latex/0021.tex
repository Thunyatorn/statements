\documentclass[11pt,a4paper]{article}

\usepackage{../../templates/style}

\begin{document}

\begin{problem}{Kemija}{standard input}{standard output}{1 second}{32 megabytes}

ลูก้าเริ่มงี่เง่ากับคาบเคมีอีกแล้ว.. แทนที่เขาจะดุลสมการเคมี เขากลับมานั่งเข้ารหัสข้อความบนกระดาษของเขา

โดยเขาจะเปลี่ยนทุกตัวสระ (a, e, i, o, u) โดยเขียนด้วยสระตัวนั้นก่อน ตามด้วยตัวอักษร ‘p’ แล้วตามด้วยสระตัวเดิมนั้นอีกครั้ง

ตัวอย่างเช่น คำว่า “kemija” (แปลว่าเคมี) จะกลายเป็น “kepemipijapa”
และคำว่า “paprika” จะกลายเป็น “papapripikapa”

แต่คุณครูก็มาหาเขาและยึดกระดาษที่เขียนข้อความที่เข้ารหัสแล้ว และต้องการที่จะถอดหาข้อความเดิมออกมา

\underline{\textbf{โจทย์}} จงเขียนโปรแกรมที่รับข้อความที่เข้ารหัสของลูก้า และพิมพ์ข้อความก่อนเข้ารหัสออกมา

\InputFile

\textbf{มีบรรทัดเดียว} เป็นข้อความที่เข้ารหัสแล้ว ข้อความจะประกอบด้วย ตัวอักษร a$-$z ตัวเล็ก และ ช่องว่าง โดยจะไม่มีช่องว่างที่หัวหรือท้ายข้อความ จำนวนตัวอักษรทั้งหมดจะไม่เกิน $100$ ตัวอักษร


\OutputFile

\textbf{มีบรรทัดเดียว} แสดงเป็นข้อความก่อนที่ลูก้าจะเข้ารหัส

\Examples

\begin{example}
\exmp{zepelepenapa papapripikapa}{zelena paprika}%
\exmp{bapas jepe doposapadnapa opovapa kepemipijapa}{bas je dosadna ova kemija}%
\end{example}


\Source

COCI 2008/2009, Contest \#3 – December 13, 2008



\end{problem}

\end{document}