\documentclass[11pt,a4paper]{article}

\usepackage{../../templates/style}

\begin{document}

\begin{problem}{Seven Dwarves}{standard input}{standard output}{1 second}{32 megabytes}

สโนไวท์มีหน้าที่จัดเตรียมอาหารค่ำให้กับคนแคระทั้งเจ็ด ซึ่งเหน็ดเหนื่อยมาจากการทำงานในเหมืองทั้งวัน ทุก ๆ วันสโนไวท์จะจัดที่นั่งไว้เจ็ดที่ เตรียมจาน ซ่อม และมีดไว้เจ็ดชุด สำหรับคนแคระทั้งเจ็ดที่หิวโซกลับมาในตอนเย็น

ปัญหาเกิดขึ้นเมื่อค่ำวันหนึ่ง \textbf{มีคนแคระกลับมาในตอนเย็นเก้าคนแทนที่จะเป็นเจ็ดคน} อาจจะเนื่องมาจากอยากแอบมาร่วมทานอาหารเย็นรสเลิศของสโนไวท์ก็เป็นได้ สโนไวท์ซึ่งเคยได้รับตำแหน่งนางงามคณิตศาสตร์จึงต้องหาวิธีค้นหาคนแคระทั้งเจ็ดตัวจริงให้ได้

โชคดีที่คนแคระทั้งเจ็ดสวมหมวกที่สโนไวท์ได้ปักหมายเลขไว้บนหมวก \textbf{ซึ่งหมายเลขของคนแคระตัวจริงทั้งเจ็ดคนจะรวมกันได้เท่ากับจำนวนเต็ม $100$ พอดี } หน้าที่ของคุณคือให้ช่วยสโนไวท์เขียนโปรแกรมเพื่อคนหาว่าคนแคระคนใดบ้างจำนวนเจ็ดคน จากทั้งหมดเก้าคน ที่มีเลขบนหมวกรวมกันได้ $100$ พอดี


\underline{\textbf{โจทย์}} จงเขียนโปรแกรมรับตัวเลขจำนวนเต็มบวกเก้าจำนวนแทนเลขบนหมวกของคนแคระแต่ละคน แล้วแสดงผล ออกมาเป็นเลขบนหมวกคนแคระทั้งเจ็ดตัวจริง

\InputFile

\textbf{มีเก้าบรรทัด} แต่ละบรรทัดมีจำนวนเต็มหนึ่งจำนวนที่มีค่าอยู่ในช่วง $1$ และ $99$ เลขทั้งเก้าจำนวนไม่ซ้ำกันเลย และจะมีเลขเพียงชุดเดียวเท่านั้นที่จะรวมกันได้ $100$ พอดี

\OutputFile

\textbf{มีเจ็ดบรรทัด} แต่ละบรรทัดเป็นตัวเลขบนหมวกของคนแคระตัวจริงทั้งเจ็ดของสโนไวท์ โดยที่ตัวเลขทั้งเจ็ดจะเรียงลำดับตามลำดับในข้อมูลนำเข้า

\Examples

\begin{example}
\exmp{7
8
10
13
15
19
20
23
25}{7
8
10
13
19
20
23}%
\exmp{8
6
5
1
37
30
28
22
36}{8
6
5
1
30
28
22}%
\end{example}

\Source

Croatian Open Competition in Informatics

Contest 3 – December 16, 2006

\end{problem}

\end{document}