\documentclass[11pt,a4paper]{article}

\usepackage{../../templates/style}

\begin{document}

\begin{problem}{ขับรถหลบสิ่งกีดขวาง (Car)}{standard input}{standard output}{2 second}{16 megabytes}

ในการแสดงขับรถผาดโผนบนถนนที่มีเลนทั้งหมด $m$ เลน โดยให้หมายเลขประจำเลนจากซ้ายไปขวามีค่าตั้งแต่ $1$ จนถึง $m$ ตามลำดับ นักแสดงขับรถผาดโผนต้องบังคับรถให้แล่นไปบนถนนดังกล่าวให้ปลอดภัยตลอดระยะเวลา $t$ หน่วย การแสดงเริ่มต้น ณ เวลา $t = 0$ นักแสดงขับรถผาดโผนอยู่ในเลนที่ $n$

ในแต่ละ $1$ หน่วยเวลา อาจมีสิ่งกีดขวางตกลงมายังถนนบางเลน ทำให้เขาต้องบังคับรถเพื่อหลีกเลี่ยงสิ่งกีดขวาง ซึ่งมีทางเลือกในการบังคับรถอยู่ $3$ แบบ ได้แก่ 
\begin{itemize}

\item $1$ หมายถึง การเปลี่ยนเลนไปทางซ้าย 1 เลนในเวลาถัดไป (ไปยังเลนที่มีหมายเลขประจำเลนน้อยกว่า)
\item $2$ หมายถึงการเปลี่ยนเลนไปทางขวา 1 เลนในเวลาถัดไป (ไปยังเลนที่มีหมายเลขประจำเลนมากกว่า)
\item $3$ หมายถึง การขับอยู่ในเลนเดิม กำหนดให้ถนนเป็นเส้นตรงตลอดทาง
\end{itemize}



\bigskip
\underline{\textbf{โจทย์}}  จงเขียนโปรแกรมเพื่อบังคับให้รถแล่นไปตามเส้นทางนี้โดยปลอดภัย \textbf{โดยชุดข้อมูลทดสอบจะมีคำตอบที่ถูกต้องเพียง 1 คำตอบเสมอ}


\InputFile

 \textbf{บรรทัดแรก} ระบุจำนวนเลน $m$ โดยที่ $2 \leq m \leq 40$

\textbf{บรรทัดที่สอง} ระบุหมายเลขเลนเริ่มต้น $n$ โดยที่ $1 \leq n \leq m$

\textbf{บรรทัดที่สาม} ระบุระยะเวลา $t$ โดยที่ $1 \leq t \leq 100$

\textbf{บรรทัดที่ $4$ ถึง $t + 3$} ระบุสถานะของถนน ณ เวลา $1,2,...,t$ ตามลำดับ แต่ละบรรทัดระบุตัวเลข $m$ ตัว เลขแต่ละตัวแสดงสถานะของถนน ตั้งแต่เลนที่ $1$ ถึงเลนที่ $m$ โดยเลข $0$ หมายถึงเลนนั้นไม่มีสิ่งกีดขวาง และเลข $1$ หมายถึงมีสิ่งกีดขวางอยู่



\OutputFile

\textbf{มี $t$ บรรทัด }แต่ละบรรทัดมีตัวเลข $1$ ตัวเพื่อแสดงถึงทางเลือกในการบังคับรถของนักแสดงขับรถผาดโผนในแต่ละช่วงเวลา บรรทัดที่ $i$ หมายถึงการเปลี่ยนเลนจากเวลาที่ $i −1$ ไปยังเวลาที่ $i$ เมื่อ $i =1,2,…,t$ โดยที่เลข $1$ จะหมายถึงขับไปทางซ้าย 1 เลน, เลข $2$ หมายถึงขับไปทางขวา 1 เลน, และเลข $3$ หมายถึงขับอยู่ในเลนเดิม

\Examples

\begin{example}
\exmp{7
5
5
0 0 0 0 0 0 0
0 0 0 0 0 0 0
0 0 0 0 0 0 0
0 1 1 0 0 0 0
1 0 1 1 1 1 1}{1
1
1
1
2}%
\exmp{5
2
3
0 0 0 1 0
0 1 1 0 0
1 1 1 0 1}{2
2
3}%
\end{example}


\Source

การแข่งขันคอมพิวเตอร์โอลิมปิกระดับชาติครั้งที่ 7 (NUTOI7)

\end{problem}

\end{document}