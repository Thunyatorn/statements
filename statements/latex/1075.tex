\documentclass[11pt,a4paper]{article}

\usepackage{../../templates/style}

\begin{document}

\begin{problem}{หนอน (worm)}{standard input}{standard output}{1 second}{128 megabytes}

 หลังจากคุณหยุดระบบรักษาความปลอดภัยที่ทำงานกะทันหันจากความผิดพลาดของตัวคุณเองได้สำเร็จ ถึงเวลาแล้วที่จะต้องหาแผนการใหม่ หลังจากครุ่นคิดอยู่ชั่วครู่ แผนการอันแยบยลก็ผุดขึ้นมาในสมองคุณ นั่นคือการถล่มด้วยหนอน !

            แต่แล้วปัญหาก็เกิดขึ้นอีกแล้ว เมื่อคุณพบว่าในการยิงหนอนแต่ละตัวนั้น ต้องใช้ค่าไฟมากยิ่งขึ้นไปอีก เหล่สายตาไปมองตัวเลขบนบิลค่าไฟที่อยู่ข้าง ๆตัว นั่นทำให้คุณตกที่นั่งลำบากอีกเสียแล้ว

            คุณมีหนอนอยู่ทั้งหมด $N$ ตัว แต่ละตัวมีค่าไฟในการยิงและจำนวนข้อมูลที่สามารถทำลายได้แตกต่างกันไป การคิดค่าไฟในการยิงหนึ่งครั้งจะคิดโดยคิดตามค่าไฟของหนอนตัวที่มีมูลค่ามากที่สุด ตัวอย่างเช่น ถ้ามีหนอน $5$ ตัว มีจำนวนข้อมูลที่ทำลายได้และค่าไฟ ดังนี้
            
\begin{center}
\begin{tabular}{|c|c|c|}
\hline
หนอนตัวที่& จำนวนข้อมูลที่ทำลายได้& ค่าไฟ\\
\hline
1&3&30\\
\hline
2&6&10\\
\hline
3&10&20\\
\hline
4&7&50\\
\hline
5&18&70\\
\hline
\end{tabular}
\end{center}

ถ้าเลือกยิงหนอนตัวที่ $1, 3, 5$ ซึ่งใช้ค่าไฟ $30, 20, 70$ ตามลำดับ จะต้องเสียค่าไฟในการยิงทั้งหมด $70$ หน่วย แต่ถ้าเลือกยิงหนอนตัวที่ $3, 4$ ซึ่งใช้ค่าไฟ $20, 50$ ตามลำดับ จะต้องเสียค่าไฟในการยิงทั้งหมด $50$ หน่วย

คุณสามารถนิยามอัตราส่วนความคุ้มค่าของการยิงหนอนให้มีค่าเท่ากับ\textbf{ จำนวนข้อมูลที่ทำลายได้ทั้งหมด หารด้วย ค่าไฟที่ใช้ในการยิง} ซึ่งแน่นอนว่าคุณไม่ต้องการจะเสียค่าไฟให้เยอะกว่าเดิมโดยเปล่าประโยชน์

\bigskip
\underline{\textbf{โจทย์}}  เขียนโปรแกรมที่รับข้อมูลของหนอนทั้งหมด $N$ ตัว และคำนวณหาอัตราส่วนความคุ้มค่าที่มากที่สุดที่เป็นไปได้

\InputFile

\textbf{บรรทัดแรก} รับจำนวนเต็ม $N$ $(1 \leq N \leq 100\,000)$ แทนจำนวนของหนอน

\textbf{บรรทัดที่ $2$ ถึง $N + 1$} ในบรรทัดที่ $i+1$ จะรับข้อมูลที่ประกอบด้วยจำนวนเต็ม $D_i$ และ $C_i$ $(0 \leq D_i \leq 50\,000; 1 \leq C_i \leq 800\,000\,000)$ แทนจำนวนข้อมูลที่ทำลายได้ และค่าไฟที่ใช้ในการยิงของหนอนตัวที่ $i$ ตามลำดับ


\OutputFile

\textbf{มีบรรทัดเดียว} แสดงจำนวนข้อมูลที่คุณสามารถทำลายได้ทั้งหมด และค่าไฟที่ใช้ในการยิง คั่นด้วยช่องว่าง $1$ ช่อง ในวิธีที่มีอัตราส่วนความคุ้มค่าที่มากที่สุด

\textbf{หมายเหตุ} หากมีวิธีค่าส่งหลายวิธีให้ตอบวิธีที่ใช้ค่าไฟน้อยที่สุด

\Examples

\begin{example}
\exmp{5
3 30
6 10
10 20
7 50
18 70}{16 20}%
\end{example}

\Note 

\textbf{อธิบายตัวอย่างข้อมูลนำเข้าและส่งออกที่ 1}

ถ้าเลือกยิงหนอนตัวที่ $2$ และ $3$ จะสามารถทำลายข้อมูลได้รวมเท่ากับ $16$ และเสียค่าไฟ $20$ หน่วย ซึ่งมีอัตราส่วนความคุ้มค่า $= 0.80$ ซึ่งเป็นค่าที่มากที่สุดในการยิงหนอนครั้งนี้

\Scoring

\textbf{30\% ของชุดข้อมูลทดสอบ:} $ N \leq 20\,000$ 
\textbf{100\%ของข้อมูลทดสอบ:} $N \leq 100\,000$
\Source

ศรัณย์ ไพศาลศรีสมสุข 

\underline{\href{http://thailandoi.org/toi.c/01-2009}{TOI.C:01-2009}}


\end{problem}

\end{document}