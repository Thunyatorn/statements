\documentclass[11pt,a4paper]{article}

\usepackage{../../templates/style}

\begin{document}

\begin{problem}{Expression}{standard input}{standard output}{1 second}{64 megabytes}

ในการแทนนิพจน์ \textit{(expression}) ใด ๆด้วยฟังก์ชัน นิพจน์หลักจะถูกแบ่งเป็นนิพจน์ย่อยๆ ด้วยตัวดำเนินการ \textit{(operator)} ต่าง ๆดังนี้ การบวก \textbf{“+”}, วงเล็บ \textbf{“( )”}, การคูณ \textbf{“ * ”} และการยกกำลัง \textbf{“\^{}”} โดยสามารถเขียนแทนด้วยฟังก์ชันได้ดังนี้ $op(i ,e)$ โดยที่ $e$ หมายถึงนิพจน์ทางคณิตศาสตร์ใด ๆ ซึ่งสามารถถูกแบ่งเป็นนิพจน์ย่อย ๆ ได้โดยใช้ตัวดำเนินการที่มีลำดับความสำคัญในการทำงาน \textit{(priority)} ต่ำสุดในนิพจน์นั้น และ $i$ คือลำดับของนิพจน์ย่อยนั้นๆ ตัวอย่างเช่น นิพจน์\textbf{ a*b+b*c+c*d }สามารถแบ่งเป็นสามนิพจน์ย่อย โดยมีนิพจน์ย่อยที่ $1$ คือ\textbf{ a*b}, นิพจน์ย่อยที่ $2$ คือ \textbf{b*c} และนิพจน์ย่อยที่ $3$ คือ \textbf{c*d} เนื่องจากตัวดำเนินการ \textit{“+”} มีความสำคัญต่ำสุดในการทำงานในนิพจน์นี้ 

กำหนดให้ลำดับความสำคัญในการทำงานของตัวดำเนินการจากมากสุดไปน้อยสุดมีดังนี้ \textit{“( )”}, \textit{“\^{}”}, \textit{“ * ”} และ \textit{“+”} ตามลำดับ

วัตถุประสงค์ของฟังก์ชันแทนนิพจน์คือ ต้องการแทนนิพจน์ย่อยด้วยฟังก์ชันเพื่อใช้ในการคำนวณ เช่น $op(2,e)$ แทนนิพจน์ย่อยลำดับที่สองของ $e$ ที่กำหนดให้ข้างบน \textbf{(a*b+b*c+c*d)} ซึ่งจะได้ $op(2,e)=$ \textbf{b*c}

\bigskip

\textbf{ตัวอย่าง}

กำหนดให้นิพจน์ $p$ มีค่าดังนี้: \textbf{a\^{}b*c+(d*c)\^{}f*z+b} ,สามารถแทนนิพจน์ย่อยใดๆ ของ $p$ ด้วยฟังก์ชันได้ดังนี้
\begin{itemize}

\item $op(3,p) =$ \textbf{b}
\item $op(1,op(3,p)) =$ \textbf{b}
\item $op(2,p) =$ \textbf{(d*c)\^{}f*z}
\item $op(1,op(2,p)) =$ \textbf{(d*c)\^{}f}
\item $op(1,op(1,op(2,p))) =$ \textbf{(d*c)}
\item $op(1,op(1,op(1,op(2,p)))) =$ \textbf{d*c}
\item $op(2,op(1,op(1,op(2,p)))) =$ \textbf{null} (ไม่มีคำตอบ)
\item $op(2,op(2,p)) =$ \textbf{z}
\end{itemize}

\underline{\textbf{โจทย์}}  จงเขียนโปรแกรมเพื่อรับข้อมูลนิพจน์ $p$ ใด ๆ และฟังก์ชันคำถาม จากนั้นคำนวณหานิพจน์ย่อยของ $p$ ที่สอดคล้องกับฟังก์ชันที่กำหนด



\InputFile

\textbf{บรรทัดแรก} รับนิพจน์หลัก $p$ ที่ประกอบด้วยตัวอักษรภาษาอังกฤษพิมพ์เล็ก $a$ ถึง $z$ และตัวดำเนินการเขียนติดกันโดยไม่มีช่องว่าง รับประกันว่าความยาวตัวอักษรและตัวดำเนินการรวมกันไม่เกิน $64$ ตัว

\textbf{บรรทัดที่สอง} รับเลขจำนวนเต็มบวก $n$ $(1 \leq n \leq 10)$ แสดงจำนวนฟังก์ชันคำถาม $n$ ฟังก์ชัน

\textbf{บรรทัดที่ 3 ถึง $n+2$} บรรทัดที่ $i+2$ ให้รับฟังก์ชันคำถามที่ $i$ โดยแต่ละบรรทัดประกอบด้วยเลขจำนวนเต็มบวกอยู่ระหว่าง $1$ ถึง $9$ คั่นด้วยช่องว่าง $1$ ช่อง และปิดท้ายด้วย $0$

\bigskip
\textbf{ตัวอย่างข้อมูลนำเข้าในบรรทัดที่ 3 ถึง $n+2$} \\
ข้อมูลนำเข้า $3$ $0$ หมายถึงฟังก์ชัน $op(3,p)$\\
ข้อมูลนำเข้า $2$ $1$ $1$ $0$ หมายถึงฟังก์ชัน $op(1,op(1,op(2,p)))$\\
ข้อมูลนำเข้า $1$ $2$ $2$ $0$ หมายถึงฟังก์ชัน $op(2,op(2,op(1,p)))$

\OutputFile

\textbf{มี $n$ บรรทัด} บรรทัดที่ $i$ ให้แสดงฟังก์ชันและนิพจน์ย่อยที่สอดคล้องกับฟังก์ชันคำถามที่ $i$ โดยในแต่ละบรรทัดของข้อมูลส่งออกจะต้องไม่มีการเว้นวรรคใดๆ กรณีที่ไม่มีคำตอบให้แสดง “null”

\Examples

\begin{example}
\exmp{a*b\^{}c+d*e\^{}f
2
1 0
2 0}{op(1,p)=a*b\^{}c
op(2,p)=d*e\^{}f}%
\exmp{a*b\^{}c+d*e\^{}f
3
1 1 0
1 2 0
1 2 2 0}{op(1,op(1,p))=a 
op(2,op(1,p))=b\^{}c 
op(2,op(2,op(1,p)))=c }%
\exmp{(x+y)+z
5
1 0
1 1 0
1 1 1 0
1 1 2 0
3 0}{op(1,p)=(x+y)
op(1,op(1,p))=x+y
op(1,op(1,op(1,p)))=x
op(2,op(1,op(1,p)))=y
op(3,p)=null}%
\end{example}

\Scoring

ในข้อมูลทดสอบ $10$ ชุด จะมีนิพจน์ที่ใช้ตัวดำเนินการ “วงเล็บ” จำนวน $5$ ชุด $(50\%)$

\Source

การแข่งขันคอมพิวเตอร์โอลิมปิก สอวน. ครั้งที่ 3 มหาวิทยาลัยขอนแก่น

\end{problem}

\end{document}