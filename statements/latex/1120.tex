\documentclass[11pt,a4paper]{article}

\usepackage{../../templates/style}

\begin{document}

\begin{problem}{ตัวหาร (divisor)}{standard input}{standard output}{1 second}{32 megabytes}

พิจารณาจำนวนเต็มบวก $X$ ใดๆ เราจะเรียก $Y$ ว่าเป็นตัวหารบวกของ $X$ เมื่อ $Y$ เป็นจำนวนเต็มบวก และ $X$ หารด้วย $Y$ ลงตัว

\textbf{ตัวอย่างเช่น} $6$ มีตัวหารบวกทั้งหมด $4$ ตัว คือ $1, 2, 3$ และ $6$ ในขณะที่ $16$ มีตัวหารบวกทั้งหมด $5$ ตัว คือ $1, 2, 4, 8$ และ $16$

\bigskip
\underline{\textbf{โจทย์}}  จงเขียนโปรแกรมเพื่อตอบคำถามทั้งหมด $Q$ คำถามว่า ในบรรดาจำนวนเต็มบวกทั้งหมดตั้งแต่ $X$ ถึง $Y$ มีกี่จำนวนที่มีตัวหารบวกอยู่ $D$ ตัวพอดี


\InputFile

\textbf{บรรทัดแรก} ระบุจำนวนเต็ม $Q$ $(2 \leq Q \leq 100)$ แทนจำนวนคำถามทั้งหมด

\textbf{บรรทัดที่ $2$ ถึง $Q+1$} ในบรรทัดที่ $i+1$ $(1 \leq i \leq Q)$ จะระบุจำนวนเต็ม $X, Y$ และ $D$ $(1 \leq X \leq Y \leq 1\,000\,000; 1 \leq D \leq 500)$ แสดงถึงคำถามที่ $i$


\OutputFile

\textbf{มี $Q$ บรรทัด} โดยในบรรทัดที่ $i$ $(1 \leq i \leq Q)$ แสดงคำตอบของคำถามที่ $i$

\Examples

\begin{example}
\exmp{3
5 17 2
1 10 1
4 9 4}{5
1
2}%
\exmp{4
1 100 2
20 50 6
15 45 7
15 45 8}{25
6
0
4}%
\end{example}

\Scoring

\textbf{$25$\% ของข้อมูลทดสอบ:} $1 \leq X \leq Y \leq1\,000$ ในทุกคำถาม

\textbf{$50$\% ของข้อมูลทดสอบ:} $1 \leq X \leq Y \leq 10\,000$ ในทุกคำถาม

\textbf{$75$\% ของข้อมูลทดสอบ: }$1 \leq X \leq Y \leq 100\,000$ ในทุกคำถาม

\Source

สุธี เรืองวิเศษ

การแข่งขัน IOI Thailand League เดือนมิถุนายน 2553

\end{problem}

\end{document}