\documentclass[11pt,a4paper]{article}

\usepackage{../../templates/style}

\begin{document}

\begin{problem}{เหมืองแร่ (mine)}{standard input}{standard output}{2 second}{64 megabytes}

นอกจากประเทศแอฟริกาใต้จะเป็นเจ้าภาพจัดการแข่งขันฟุตบอลโลก 2010 แล้ว ยังเป็นประเทศที่อุดมสมบูรณ์ไปด้วยทรัพยากรธรรมชาติ เช่น ทองแดง เงิน และทองคำอีกด้วย คุณจึงวางแผนจะไปทำธุรกิจเปิดเหมืองทองแดง เหมืองเงิน และเหมืองทองคำที่ประเทศแอฟริกาใต้

พื้นที่ของประเทศแอฟริกาใต้มีลักษณะเป็นตารางสี่เหลี่ยมจัตุรัสขนาด $N \times N$ ช่อง แต่ละช่องจะมีทองแดง เงิน และทองคำอยู่ในปริมาณต่างๆ คุณต้องการสร้างเหมืองทองแดง เหมืองเงิน และเหมืองทองคำ ซึ่งแต่ละเหมืองจะเป็นสี่เหลี่ยมจัตุรัสขนาด $K \times K$ ช่อง ปริมาณแร่แต่ละชนิดที่คุณขุดได้จะมีค่าเท่ากับผลรวมของปริมาณแร่ชนิดนั้นในพื้นที่เหมืองของคุณ เหมืองแร่แต่ละเหมืองสามารถซ้อนทับกันได้โดยไม่มีผลกระทบต่อการขุดแร่

อย่างไรก็ตาม หากเหมืองแร่แต่ละเหมืองอยู่ห่างกันมากเกินไป จะทำให้การบริหารงานไม่สะดวก คุณจึงกำหนดเงื่อนไขว่า เหมืองแร่สองเหมืองใดๆ จะต้องมีพื้นที่ร่วมกันอย่างน้อยหนึ่งช่อง คุณต้องการเลือกตำแหน่งของเหมืองแร่แต่ละเหมืองให้เหมาะสม เพื่อให้ปริมาณแร่ที่ขุดได้ทั้งสามชนิดรวมกันมากที่สุดเท่าที่จะทำได้

\bigskip
\underline{\textbf{โจทย์}}  จงเขียนโปรแกรมเพื่อรับปริมาณทองแดง เงิน และทองคำ ในพื้นที่แต่ละช่อง และคำนวณปริมาณรวมของแร่ทั้งสามชนิดที่มากที่สุดที่คุณสามารถขุดได้


\InputFile

\textbf{บรรทัดแรก} ระบุจำนวนเต็ม $N$ และ $K$ $(1 \leq K \leq N \leq 1\,000)$ แทนขนาดของพื้นที่ทั้งหมด และขนาดของเหมืองแร่แต่ละเหมือง

\textbf{บรรทัดที่ $2$ ถึง $N+1$} ระบุจำนวนเต็มบวกบรรทัดละ $N$ ตัว แทนปริมาณทองแดงในพื้นที่แต่ละช่อง

\textbf{บรรทัดที่ $N+2$ ถึง $2N+1$} ระบุจำนวนเต็มบวกบรรทัดละ $N$ ตัว แทนปริมาณเงินในพื้นที่แต่ละช่อง

\textbf{บรรทัดที่ $2N+2$ ถึง $3N+1$} ระบุจำนวนเต็มบวกบรรทัดละ $N$ ตัว แทนปริมาณทองคำในพื้นที่แต่ละช่อง

ปริมาณของแร่แต่ละชนิดในพื้นที่แต่ละช่องจะมีค่าไม่เกิน $500$


\OutputFile

\textbf{มีบรรทัดเดียว} แสดงปริมาณรวมของแร่ทั้งสามชนิดที่มากที่สุดที่คุณสามารถขุดได้


\Examples

\begin{example}
\exmp{3 2
1 2 3
4 5 6
7 8 9
1 2 3
4 5 6
7 8 9
9 8 7
6 5 4
3 2 1}{84}%
\exmp{4 2
7 7 1 1
7 7 1 1
1 1 1 1
1 1 1 1
1 8 8 1
1 8 8 1
1 1 1 1
1 1 1 1
1 1 1 1
9 9 1 1
9 9 1 1
1 1 1 1}{96}%
\end{example}

\Scoring 

\textbf{$30$\% ของข้อมูลทดสอบ:} $N \leq 15$

\textbf{$50$\% ของข้อมูลทดสอบ:} $N \leq 50$

\Source

สุธี เรืองวิเศษ

การแข่งขัน IOI Thailand League เดือนมิถุนายน 2553

\end{problem}

\end{document}