\documentclass[11pt,a4paper]{article}

\usepackage{../../templates/style}

\begin{document}

\begin{problem}{ชั่งสินค้า (balance)}{standard input}{standard output}{1 second}{32 megabytes}

คุณเป็นพ่อค้าที่ตระเวนขายสินค้าไปตามเมืองต่างๆ ในการชั่งสินค้าแต่ละชิ้นนั้น คุณจะใช้ตาชั่งสองแขนและตุ้มน้ำหนักเป็นอุปกรณ์ในการชั่ง เนื่องจากคุณทราบว่าสินค้าที่ขายจะมีน้ำหนักเป็นจำนวนเต็มเสมอ และจะมีน้ำหนักไม่เกิน $1\,000\,000\,000$ หน่วย คุณจะเตรียมตุ้มน้ำหนักขนาดตั้งแต่ $1$ หน่วย, $2$ หน่วย, $3$ หน่วย ไปเรื่อยๆ จนถึง $1\,000\,000\,000$ หน่วย ซึ่งมีจำนวนมากและสร้างความลำบากในการพกพาระหว่างเดินทางเป็นอย่างมาก

วันหนึ่ง คุณได้เดินทางไปพบหมาป่าสาวตัวหนึ่ง เธอได้ขอร่วมเดินทางไปกับคุณ และเนื่องจากเธอเป็นหมาป่าที่ฉลาดมาก เมื่อเธอเห็นตุ้มน้ำหนักจำนวนมากมายมหาศาลของคุณ จึงได้บอกว่า คุณไม่จำเป็นจะต้องใช้ตุ้มน้ำหนักมากขนาดนี้ ใช้เพียงแค่ $20$ อันก็เพียงพอแล้ว คือใช้ตุ้มน้ำหนักที่ขนาดเป็นกำลังของ $3$ ทั้งหมด นั่นคือขนาด $1$ หน่วย $3$ หน่วย $9$ หน่วย ไปเรื่อยๆ จนถึง $3^{19} = 1\,162\,261\,467$ หน่วย

ตัวอย่างเช่น ถ้าคุณมีสินค้าน้ำหนัก $10$ หน่วย คุณสามารถชั่งสินค้าได้โดยการนำสินค้าวางไว้บนตาชั่งข้างซ้าย แล้วนำตุ้มน้ำหนักขนาด $9$ หน่วยและ $1$ หน่วย วางไว้บนตาชั่งข้างขวา ก็จะทำให้ตาชั่งสมดุล ดังนั้น คุณจึงใช้ตุ้มน้ำหนักทั้งหมด $2$ อัน และสิ่งของบนตาชั่งแต่ละข้างจะมีน้ำหนักรวมข้างละ $10$ หน่วย

หรือถ้าคุณมีสินค้าน้ำหนัก $20$ หน่วย คุณก็จะสามารถชั่งได้โดยการนำสินค้าและตุ้มน้ำหนักขนาด $9$ หน่วยและ $1$ หน่วย วางไว้บนตาชั่งข้างซ้าย แล้วนำตุ้มน้ำหนักขนาด $27$ หน่วยและ $3$ หน่วย วางไว้บนตาชั่งข้างขวา ก็จะทำให้ตาชั่งสมดุล ดังนั้นคุณจึงใช้ตุ้มน้ำหนักทั้งหมด $4$ อัน และสิ่งของบนตาชั่งแต่ละข้างจะมีน้ำหนักรวมข้างละ $30$ หน่วย

หมาป่าสามารถพิสูจน์ได้ว่า ไม่ว่าสินค้าจะมีน้ำหนักเท่าใดก็จะสามารถนำไปชั่งบนตาชั่งสองแขนให้สมดุลได้เสมอ และจะมีวิธีชั่งได้เพียงวิธีเดียวเท่านั้น (การสลับตาชั่งข้างซ้ายกับข้างขวาถือว่าเป็นวิธีเดียวกัน) เมื่ออธิบายจนคุณเชื่อแล้วหมาป่าก็จากคุณไป

แม้ว่าคุณจะพอเข้าใจ แต่คุณคิดเลขได้ไม่เร็วเท่าหมาป่า ในการขายสินค้าชิ้นหนึ่งคุณต้องการทราบว่า ในการชั่งสินค้าชิ้นนั้น จะต้องใช้ตุ้มน้ำหนักทั้งหมดกี่อัน และเมื่อชั่งแล้วตาชั่งแต่ละข้างจะมีน้ำหนักรวมข้างละเท่าไร

\bigskip
\underline{\textbf{โจทย์}}  จงเขียนโปรแกรมเพื่อรับน้ำหนักของสินค้า และคำนวณหาจำนวนตุ้มน้ำหนักที่ต้องใช้ในการชั่งสินค้าด้วยตาชั่งสองแขน และน้ำหนักรวมของสิ่งของบนตาชั่งข้างหนึ่ง


\InputFile

\textbf{มีบรรทัดเดียว} ระบุจำนวนเต็ม $N$ $(1 \leq N \leq 1\,000\,000\,000)$ แทนน้ำหนักของสินค้า

\OutputFile

\textbf{มีบรรทัดเดียว} ระบุจำนวนตุ้มน้ำหนักที่ต้องใช้ และน้ำหนักรวมของสิ่งของบนตาชั่งข้างหนึ่ง

\Examples

\begin{example}
\exmp{10}{2 10}%
\exmp{20}{4 30}%
\end{example}

\Scoring

\textbf{$30$\% ของข้อมูลทดสอบ:} $N \leq 100\,000$

\Source

สุธี เรืองวิเศษ

การแข่งขัน IOI Thailand League เดือนกรกฏาคม 2553


\end{problem}

\end{document}