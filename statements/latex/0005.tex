\documentclass[11pt,a4paper]{article}

\usepackage{../../templates/style}


\graphicspath{ {./img/0005/} }

\begin{document}

\begin{problem}{Pythagorus}{standard input}{standard output}{1 second}{64 megabytes}

รูปสามเหลี่ยมมุมฉาก (right, rectangled) มีมุมภายในมุมหนึ่งมีขนาด $90^{\circ}$ (มุมฉาก) ด้านที่อยู่ตรงข้ามกับมุมฉากเรียกว่า ด้านตรงข้ามมุมฉาก ซึ่งเป็นด้านที่ยาวที่สุดในรูปสามเหลี่ยม อีกสองด้านเรียกว่า ด้านประกอบมุมฉาก

มีทฤษฎีที่เกียวข้องกับสามเหลี่ยมมุมฉาก ทฤษฎีนั้นคือ ทฤษฎีบทพีทาโกรัส กล่าวไว้ว่า "ผลรวมของพื้นที่ของรูปสี่เหลี่ยมจัตุรัสบนด้านประชิดมุมฉากทั้งสอง จะเท่ากับ พื้นที่ของรูปสี่เหลี่ยมจัตุรัสบนด้านตรงข้ามมุมฉาก"


\underline{\textbf{โจทย์}} จงคำนวณความยาวของด้านตรงข้ามมุมฉาก เมื่อระบุความยาวของด้านประกอบมุมฉากทั้งสองด้านมาให้

\InputFile

\textbf{บรรทัดแรก} ประกอบไปด้วยจำนวนจริงบวก 2 จำนวน คั่นด้วยช่องว่าง 1 ช่อง แต่ละจำนวนจะบ่งบอกถึงความยาวของด้านประกอบมุมฉากของรูปสามเหลี่ยมรูปหนึ่ง

\OutputFile

\textbf{บรรทัดแรกเพียงบรรทัดเดียว} แสดงความยาวของด้านตรงข้ามมุมฉากของรูปสามเหลี่ยมมุมฉากที่มีด้านประกอบมุมฉากที่มีความยาวเท่ากับที่ระบุไว้ในข้อมูลนำเข้า ตอบเป็นทศนิยม 6 ตำแหน่ง

\Examples

\begin{example}
\exmp{3.000000 4.00000
}{5.000000
}%
\end{example}

\Source

Programming.in.th (Northern series)

\end{problem}

\end{document}