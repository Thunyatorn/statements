\documentclass[11pt,a4paper]{article}

\usepackage{../../templates/style}

\begin{document}

\begin{problem}{เกินความจำเป็น (overtree)}{standard input}{standard output}{1.5 second}{32 megabytes}

      การเตรียมงานทุกอย่างเป็นอันเรียบร้อย เหลือเพียงแต่แขกผู้ทรงเกียรติทั้งหลายที่ยังไม่มีใครมาถึงคฤหาสน์แห่งนี้เลยซักคน มีสายโทรศัพท์จากคนขับรถทัวร์มาแจ้งว่า รถทัวร์ที่รับแขกทุกคนมางานเลี้ยงกำลังหลงทางอยู่ในป่าพิศวง!

      เดิมที คนขับรถจะต้องมีแผนที่เส้นทางลัดในป่า ซึ่งมีลักษณะเป็นรูปกราฟต้นไม้ ที่ประกอบไปด้วยจุดพักนักท่องเที่ยว $N$ จุดและถนนเชื่อมระหว่างจุดพักนักท่องเที่ยว $N-1$ สายซึ่งมีความยาวต่าง ๆ และเป็นถนนแบบ Two-way และเชื่อมจุดพักนักท่องเที่ยวถึงกันได้หมด

      แต่ด้วยความสะเพร่า คนขับรถเจ้าปัญหาได้ลืมแผนที่เส้นทางลัดในป่าไว้ที่บ้าน นั่นทำให้เขาเหลือแต่แผนที่เส้นทางในป่าฉบับเจาะลึก ซึ่งระบุระยะทางระหว่างจุดพักนักท่องเที่ยวต่าง ๆ แทนที่จะระบุถนนที่เชื่อมระหว่างจุดพักนักท่องเที่ยว   ข้อมูลระยะทางที่ระบุในแผนที่ดังกล่าวมีทั้งสิ้น $M$ ข้อมูล แต่ละข้อมูล $i$ จะบอกว่าระหว่างจุดพัก $u_i$ กับ $v_i$ มีระยะทางห่างกัน $d_i$ โดยรับประกันว่า
\begin{itemize}

\item จะมีข้อมูลระยะทางระหว่างเมืองสองเมืองที่มีถนนเชื่อมกันโดยตรง ปรากฏเป็นหนึ่งข้อมูลหนึ่งใน $M$ ข้อมูลนี้เสมอ
\item ระยะทางของถนนระหว่างจุดพักนักท่องเที่ยวใดๆ จะไม่เกิน $10\,000$ เมตรเสมอ
\item ไม่มีข้อมูลของระยะทางระหว่างคู่จุดพักใด ๆ ที่ปรากฏในข้อมูลนำเข้ามากกว่าหนึ่งครั้ง
\end{itemize}      

คนขับรถได้วานขอให้คุณช่วยหาถนนในแผนที่เส้นทางลัด ทั้งหมด $N-1$ เส้น จากข้อมูลระยะทางระหว่างจุดพักนักท่องเที่ยวในแผนที่ฉบับเจาะลึกทั้งสิ้น $M$ ข้อมูลที่กำหนดให้

\bigskip
\underline{\textbf{โจทย์}}  จงเขียนโปรแกรมรับจำนวนจุดพักนักท่องเที่ยวและข้อมูลระยะทางระหว่างคู่ของจุดพักนักท่องเที่ยวหลายแห่ง แล้วหาว่ามีถนนเส้นใดอยู่ในแผนที่ทางลัดบ้าง


\InputFile

\textbf{บรรทัดแรก} รับจำนวนเต็มบวก $N$ $(2 \leq N \leq 100\,000)$ แทนจำนวนจุดพักนักท่องเที่ยวทั้งหมดและจำนวนเต็ม $M$ $(N-1 \leq M \leq 1\,000\,000)$ แทนข้อมูลความยาวของระยะทางระหว่างคู่ของจุดพักนักท่องเที่ยว

\textbf{บรรทัดที่ $2$ ถึง $N+1$} จะมีข้อมูลระยะทางระหว่างจุดพักนักท่องเที่ยวสองแห่งในแผนที่ฉบับเจาะลึก  กล่าวคือในบรรทัดที่ $i+1$ ของข้อมูลนำเข้า โดยที่ $1 \leq i \leq M$ จะระบุจำนวนเต็ม $u_i,v_i,d_i$ ซึ่งหมายความว่าเส้นทางระหว่างจุดพักนักท่องเที่ยว $u_i$ และ $v_i$  บนกราฟต้นไม้นี้มีระยะทางเป็น $d_i$ $(1 \leq u_i,v_i \leq N$ และ $u_i \neq v_i)$

\OutputFile

\textbf{มี $N-1$ บรรทัด} บอกคู่ของถนนที่ปรากฏในแผนที่เส้นทางลัด บรรทัดละหนึ่งคู่ โดยสามารถแสดงคำตอบในลำดับใดก็ได้

\Examples

\begin{example}
\exmp{5 6
1 2 2
2 3 3
1 4 6
2 4 4
3 5 9
4 5 2}{1 2
2 3
2 4
4 5}%
\exmp{6 5
1 2 1
2 3 2
3 4 5
4 5 11
5 6 4}{1 2
2 3
3 4
4 5
5 6}%
\end{example}


\Source

อาภาพงศ์ จันทร์ทอง

\underline{\href{http://www.thailandoi.org/toi.c/02-2009}{TOI.CPP:02-2009}}

\end{problem}

\end{document}