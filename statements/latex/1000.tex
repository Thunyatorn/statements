\documentclass[11pt,a4paper]{article}

\usepackage{../../templates/style}

\begin{document}

\begin{problem}{Word Chain}{standard input}{standard output}{1 second}{64 megabytes}

\textbf{โซ่คำ} คือลำดับของคำที่มีจำนวนอักขระเท่ากันและแต่ละคำที่มีลำดับติดกันจะต้องมีตำแหน่งที่มีตัวอักขระต่างกันไม่เกินสองตำแหน่ง เช่น HEAD และ HEAP จะต่างกันตำแหน่งเดียวคือ D และ P ในตำแหน่งตัวอักขระที่ 4 ของคำ ในขณะที่ REAR กับ BAER จะมีตำแหน่งต่างกัน 3 ตำแหน่ง คือ ตำแหน่งที่ 1 (R กับ B) ตำแหน่งที่ 2 (E และ A) และ ตำแหน่งที่ 3 (A และ E)

\textbf{ตัวอย่างของโซ่คำที่ต่อเนื่อง}ได้แก่ HEAD HEAP LEAP TEAR REAR และ EGG EAG GAE GAP TAP TIN

\textbf{ตัวอย่างของโซ่คำที่ขาด}ได้แก่ LEAP TEAR REAR BAER BAET BEEP ซึ่งจะขาดที่คำว่า BAER

\underline{\textbf{โจทย์}} จงเขียนโปรแกรมรับชุดของโซ่คำมาชุดหนึ่ง แล้วแสดงผลเป็น\textbf{คำสุดท้ายในโซ่คำ }ก่อนที่โซ่คำจะขาด

\InputFile
\textbf{บรรทัดแรก} เก็บจำนวนเต็ม $L$ แทนจำนวนตัวอักษรของแต่ละคำ โดยที่ $3 \leq L \leq 1\,000$

\textbf{บรรทัดที่สอง} เก็บจำนวนเต็ม $N$ แทนจำนวนคำทั้งหมดในแฟ้มข้อมูล โดยที่ $1 \leq N \leq 30\,000$

\textbf{บรรทัดที่ $3$ ถึง $N +2$} เก็บลำดับของคำที่มีจำนวนตัวอักขระ $L$ ตัว แต่ละบรรทัดเก็บคำที่เขียนด้วยตัวอักษร \\ (‘A’ ถึง ‘Z’ ) ที่เป็นตัวพิมพ์ใหญ่

\OutputFile

\textbf{มีบรรทัดเดียว} แสดงผลเป็นคำสุดท้ายของโซ่คำ

\Examples

\begin{example}
\exmp{4
12
HEAD 
HEAP 
LEAP 
TEAR 
REAR
BAER
BAET
BEEP
JEEP
JOIP
JEIP
AEIO}{REAR}%
\end{example}

\Source

การแข่งขันคอมพิวเตอร์โอลิมปิก สอวน. ครั้งที่ 1 มหาวิทยาลัยเกษตรศาสตร์

\end{problem}

\end{document}