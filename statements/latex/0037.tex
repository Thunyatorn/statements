\documentclass[11pt,a4paper]{article}

\usepackage{../../templates/style}

\begin{document}

\begin{problem}{บินดูไฟ (skylight)}{standard input}{standard output}{1 second}{32 megabytes}

ในวันปีใหม่ สนามกีฬาแห่งหนึ่งได้ประดับไฟที่พื้นสนามฟุตบอลเพื่อความสวยงาม ในการประดับไฟนั้นทำโดยแบ่งสนามสี่เหลี่ยมเป็นช่องย่อยๆ จำนวน $N$ แถว แถวละ $M$ คอลัมน์ รวม $N\times M$ ช่อง เจ้าของสนามได้เปิดสนามให้ประชาชนทั่วไปเข้าชมเพื่อความสวยงาม

อย่างไรก็ตาม โลกนี้ไม่มีอะไรฟรี เจ้าของสนามจะต้องจ่ายค่าไฟให้กับไฟประดับเหล่านี้ เนื่องจากมีการประดับไฟเป็นลวดลายต่างๆ ค่าไฟของไฟแต่ละช่องไม่จำเป็นต้องเท่ากัน

เพื่อไม่ให้เป็นการขาดทุน เจ้าของสนามจึงได้จัดเครื่องไอพ่นเจ็ตส่วนบุคคลให้กับประชาชนเช่าเพื่อบินดูไฟประดับ เครื่องพ่นเจ็ตแต่ละเครื่องเมื่อเช่าไปแล้วจะผู้ใช้จะสามารถบินได้ทั้งสิ้น $K$ ครั้ง ในการบินแต่ละครั้งจะใช้เชื้อเพลิงมูลค่าเท่ากับ $L$ บาท ดังนั้น ค่าใช้จ่ายรวมทั้งหมดของเจ้าของสนามคือค่าไฟรวมของไฟประดับ และค่าเชื้อเพลิงรวมของการบินเครื่องไอพ่นเจ็ตในการบินทั้งหมด

เจ้าของสนามทราบว่าจะมีคนมาชมและเช่าเครื่องไอพ่นเจ็ตจำนวน $C$ คน เขาต้องการคำนวณค่าเช่าเครื่องไอพ่นเจ็ตต่อคนที่น้อยที่สุด ที่จะทำให้เขาไม่ขาดทุน เพื่อให้การเช่าเป็นไปได้สะดวก ค่าเช่าจะต้องเป็นจำนวนเต็มเสมอด้วย

\underline{\textbf{โจทย์}} จงเขียนโปรแกรมรับราคาค่าไฟของสนามแต่ละช่อง รวมทั้งข้อมูลของการใช้เครื่องไอพ่นเจ็ต จากนั้นคำนวณหาค่าเช่าเครื่องไอพ่นเจ็ตต่อคนที่เป็นจำนวนเต็มที่น้อยที่สุด ที่จะทำให้เจ้าของสนามไม่ขาดทุน

\InputFile

\textbf{บรรทัดแรก} ระบุจำนวนเต็มบวก $N$ และ $M$ คั่นด้วยช่องว่าง แทนขนาดความกว้างและความยาวของสนาม $(1 \leq N \leq 100; 1 \leq M \leq 100)$

\textbf{บรรทัดที่สอง} ระบุจำนวนเต็มบวก $L$ และ $K$ คั่นด้วยช่องว่าง โดยที่ $L$ แทนราคาเชื้อเพลิงต่อการบินหนึ่งครั้งและ $K$ แทนจำนวนครั้งที่เครื่องไอพ่นใช้บินได้ต่อคนเช่าหนึ่งคน $(1 \leq L \leq 100; 1 \leq K \leq 100)$

\textbf{บรรทัดที่สาม} ระบุจำนวนเต็มบวก $C$ แทนจำนวนผู้เล่นทั้งหมดที่เข้ามาเล่น $(1 \leq C < 1\,000)$

\textbf{บรรทัดที่ $4$ ถึง $3+N$} แต่ละบรรทัดรับจำนวนเต็มบวก $M$ ตัว แต่ละตัวถูกคั่นด้วยช่องว่าง แทนค่าไฟในแต่ละช่องที่ประดับไฟ ซึ่งจะเป็นจำนวนเต็มบวกที่มีค่าไม่เกิน $3\,000$

\OutputFile

\textbf{มีบรรทัดเดียว} มีบรรทัดเดียวเป็นจำนวนเต็มบวกหนึ่งจำนวน แทนค่าเช่าเครื่องไอพ่นเจ็ตต่อคนที่เป็นจำนวนเต็มที่น้อยที่สุด  ที่จะทำให้เจ้าของสนามไม่ขาดทุน

\Examples

\begin{example}
\exmp{3 3
2 1
1
1 1 1
1 1 1
1 1 1}{11}%
\exmp{3 4
3 2
7
1 2 3 4
4 3 2 1
1 1 1 1}{10}%
\end{example}

\Source

โจทย์โดย: ธงชัย วิโรจน์ศักดิ์เสรี

การแข่งขัน IOI Thailand League เดือนสิงหาคม 2553

\end{problem}

\end{document}