\documentclass[11pt,a4paper]{article}

\usepackage{../../templates/style}

\begin{document}

\begin{problem}{สามเหลี่ยมอีกครั้ง (triagain)}{standard input}{standard output}{1 second}{16 megabytes}

     บรรยากาศรอบตัวเย็นขึ้นอย่างน่าประหลาด ทันใดนั้น แม่ครัวก็ได้เตือนคุณว่าคุณลืมเตรียมท่อนไม้มาจุดไฟเตาผิงนั่นเอง ลูกน้องแต่ละคนต่างเคร่งเครียดอยู่กับหน้าที่ที่ได้รับ ดูเหมือนว่าคุณจะต้องลงมือออกไปเก็บท่อนไม้ด้วยตัวเองแล้ว   

      เนื่องจากรอบคฤหาสน์นี้รายล้อมไปด้วยป่าทั้งสิ้น คุณจึงพบท่อนไม้กระจัดกระจายอยู่มากมาย เยอะมากพอสำหรับที่จะนำไปใช้เป็นเชื้อเพลิง แต่ทุกอย่างไม่ได้ง่ายอย่างที่มีคิด ป้ายคำเตือนเสียบติดไว้กับพื้นดิน มีข้อความบางอย่างเขียนอยู่

\begin{center}
    \textit{  “เทพเจ้าตัวเลขทรงโปรดรูปสามเหลี่ยมมาก ท่านต้องการอนุโมทนาบุญด้วยการมอบท่อนไม้ที่มีอยู่ให้กับคนที่จำเป็นต้องใช้มัน แต่ท่านเทพจะทรงพระกริ้วถึงขีดสุดเมื่อพบว่า ในจำนวนทั้งหมดของท่อนไม้ที่หยิบไปนั้น มีท่อนไม้สามท่อนที่ไม่สามารถต่อกันเป็นรูปสามเหลี่ยมได้ ระวังอย่าให้ท่านเทพโกรธนะจ๊ะ \^{} - \^{}\””}
\end{center}

      ข้อความนั้นดูคุ้นเคยอย่างบอกไม่ถูก แต่นี่ไม่ใช่เวลาที่จะสงสัย หน้าที่ของคุณตอนนี้คือนำท่อนไม้กลับไปให้มากที่สุดเท่าที่จะเป็นไปได้

      เราจะกล่าวว่าเซตของท่อนไม้ทั้งหมดที่คุณหยิบไปเป็นเซตสามเหลี่ยม ก็ต่อเมื่อ ถ้ามีท่อนไม้สามท่อนใด ๆในเซต ต้องสามารถนำไปประกอบเป็นด้านของรูปสามเหลี่ยมที่มีพื้นที่มากกว่าศูนย์ได้เสมอ  เช่น $\{4,4,5,7\}$ ถือว่าเป็นเซตสามเหลี่ยม ส่วน $\{3,5,5,9\}$ ไม่ถือว่าเป็นเซตสามเหลี่ยม เป็นต้น

      จะกำหนดเซตของท่อนไม้ทั้งหมดมาให้ แล้วให้หาเซตสามเหลี่ยมของต้นไม้ที่มีจำนวนสมาชิกมากที่สุดที่คุณสามารถนำกลับไปได้

\bigskip
\underline{\textbf{โจทย์}}  เขียนโปรแกรมที่รับข้อมูลความยาวของท่อนไม้ทั้งหมด แล้วหาจำนวนท่อนไม้ที่มากที่สุดของเซตสามเหลี่ยม

\InputFile

\textbf{บรรทัดแรก} มีจำนวนเต็มบวก $N$ $(1 \leq N \leq 30\,000)$ ซึ่งระบุจำนวนท่อนไม้ทั้งหมด

\textbf{บรรทัดที่ $2$ ถึง $N+1$} ในบรรทัดที่ $i+1$ จะรับุจำนวนเต็มบวกหนึ่งตัว $a_i$ ระบุความยาวของท่อนไม้ท่อนที่ $i$ ซึ่งทุกท่อนมีความยาวเป็นจำนวนเต็มที่อยู่ในช่วง $1 \leq a_i \leq 500$ 


\OutputFile

\textbf{มีบรรทัดเดียว} มีจำนวนเต็มหนึ่งจำนวน ระบุจำนวนท่อนไม้มากที่สุดในเซตสามเหลี่ยมที่เป็นไปได้

\Examples

\begin{example}
\exmp{10
7
1
2
8
10
6
1
7
9
9}{7}%
\end{example}


\Source

ศิระ ทรงพลโรจนกุล

\underline{\href{http://www.thailandoi.org/toi.c/02-2009}{TOI.C:02-2009}}

\end{problem}

\end{document}