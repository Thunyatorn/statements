\documentclass[11pt,a4paper]{article}

\usepackage{../../templates/style}

\begin{document}

\begin{problem}{Brick}{standard input}{standard output}{1 second}{64 megabytes}

ในตารางเกมขนาด $N$ แถว $M$ คอลัมน์ โดยในตารางมีสิ่งกีดขวางวางเอาไว้ ด้านบนมีก้อนอิฐหลายๆ ก้อนที่กำลังจะหล่นลงมา ตัวอย่างของเกมดูได้ในตัวอย่างข้อมูลนำเข้า เป็นตารางขนาด $(N=8)\times (M=5)$ ซึ่งสถานะเริ่มต้นของตารางเกมแสดงในตัวอย่างข้อมูลน้ำเข้า และเมื่อเกมได้ประมวลผลแล้ว ซึ่งก็คืออิฐตกจากด้านบนลงสู่ด้านล่าง จะเห็นว่าอิฐจะมีการตกค้างที่สิ่งกีดขวาง และผลลัพธ์หลังจากประมวลเสร็จสิ้นดังแสดงในตัวอย่างข้อมูลส่งออก


\underline{\textbf{โจทย์}} จงเขียนโปรแกรมเพื่อรับตารางเกมเริ่มต้นและจำนวนอิฐที่จะตกลงมาในแต่ละคอลัมน์ ให้ประมวลผลก้อนอิฐทุกก้อน โดยมีเงื่อนไขดังนี้
\begin{enumerate}
\item  ถ้าก้อนอิฐตกลงมาแล้วพบสิ่งกีดขวางที่อยู่ในตารางเกม ก็จะค้างอยู่ ณ ตำแหน่งที่พบสิ่งกีดขวาง
\item ถ้าก้อนอิฐไม่พบสิ่งกีดขวางจะตกลงมาอยู่แถวล่างสุด เมื่อประมวลผลครบทุกก้อนอิฐให้แสดงผลสถานะของตารางเกม 
\end{enumerate}

\InputFile
\textbf{บรรทัดแรก} จะระบุจำนวนเต็มสองจำนวน $N$ และ $M$ โดยที่ $1 < N < 20$ และ $1 < M < 20$

\textbf{บรรทัดที่ $2$ ถึง $N+1$} จะเป็นการระบุตารางเกม โดยในบรรทัดที่ $1 + i$ จะเป็นข้อมูลของตารางเกมแถวที่ $i$ ซึ่งจะระบุเป็นสายอักขระความยาว $M$ ตัวอักขระ ที่มีรูปแบบดังนี้:
\begin{enumerate}

\item เครื่องหมายจุด \textbf{‘.’} แทนช่องที่ว่างในตารางเกม 
\item ตัวอักษร \textbf{‘O’} (ตัวพิมพ์ใหญ่โอ) 
\end{enumerate}

\textbf{บรรทัดที่ $N+2$} ประกอบด้วยตัวเลข $M$ ตัวคือ $a_1,a_2,a_3,…,a_M$ แต่ละตัวคั่นด้วยช่องว่างหนึ่งช่อง โดยที่ $a_j$ คือจำนวนก้อนอิฐที่จะตกลงมาในคอลัมน์ที่ $j$ และ $0\leq a_j\leq20$

\OutputFile

\textbf{มี $N$ บรรทัด} ให้ระบุตารางเกมผลลัพธ์ในรูปแบบเดียวกับในแฟ้มข้อมูลนำเข้า ให้ใช้เครื่องหมาย \textbf{‘#’} แทนก้อนอิฐอยู่ในตาราง

\Examples

\begin{example}
\exmp{8 5
.....
.....
.OO..
.....
.O...
...O.
.....
.....
1 1 3 2 0}{..\#..
.\#\#..
.OO..
...\#.
.O.\#.
...O.
.....
\#....}%
\end{example}

\Source

การแข่งขันคอมพิวเตอร์โอลิมปิก สอวน. ครั้งที่ 1 มหาวิทยาลัยเกษตรศาสตร์

\end{problem}

\end{document}