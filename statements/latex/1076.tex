\documentclass[11pt,a4paper]{article}

\usepackage{../../templates/style}

\begin{document}

\begin{problem}{โกหก (lieman)}{standard input}{standard output}{1 second}{128 megabytes}

 และแล้วคุณก็ต้องตกใจกับแผนการถล่มด้วยหนอนที่ไม่ได้ระแคะระคายระบบของ\textit{ TOI.C }เอาเลยแม้แต่น้อย นั่นทำให้คุณรู้สึกจนปัญญาเอามาก

รู้สึกตัวอีกที คุณก็กำลังอยู่ในเว็บบอร์ดใต้ดินที่รวบรวมนักเลงคอมพิวเตอร์มือฉมังเอาไว้เสียมากมาย เป้าหมายของคุณคือการเสาะหาวิธีการบุกเจาะเข้าไปยังฐานข้อมูลของ \textit{TOI.C} หลังจากปฏิบัติการล้มเหลวมาแล้วถึงสองหน ถึงเวลาที่จะต้องพึ่งคนอื่นแล้ว

คุณได้ตัดสินใจตั้งกระทู้ถามคำถามใช่-ไม่ใช่ เกี่ยวกับระบบการรักษาความปลอดภัยของ\textit{ TOI.C} เป็นจำนวนมากถึง $M$ คำถาม เหล่านักเลงคอมพิวเตอร์ผู้สิงสถิตในบอร์ดต่างตื่นตาตื่นใจกับคำถาม และได้ตอบคำถามของคุณอย่างจริงจัง เป็นจำนวน $N$ คน (ไม่นับคนที่ป่วนกระทู้ ปั่นกระทู้ ดันกระทู้ กวนกระทู้ และดองกระทู้) โดยคนที่มาตอบคำถามอาจไม่ตอบทุกคำถามของคุณก็ได้ แต่แล้วคุณก็ต้องตกใจ เพราะคำตอบที่ได้มานั้น มีทั้งคำตอบที่ถูกต้องและคำโกหก คุณลืมไปเสียสนิทว่านักเลงคอมพิวเตอร์เหล่านี้มักจะลองเชิงสมาชิกหน้าใหม่ด้วยการโกหกคำตอบในบางข้อ

แต่ถึงอย่างนั้น คุณได้ตัดสินใจวิเคราะห์คำตอบของแต่ละคน แต่ถึงจะวิเคราะห์ยังไง คุณก็ไม่สามารถวิเคราะห์ได้ด้วยมือซักที เพราะคุณไม่ใช่นักเศรษฐศาสตร์ หรือนักจิตวิทยา แต่คุณคือโปรแกรมเมอร์ ฉะนั้นหน้าที่ของคุณคือ เขียนโปรแกรมเพื่อวิเคราะห์ว่า ถ้าคนจำนวนมากที่สุดที่เป็นไปได้ตอบคำถามเป็นความจริงแล้วจะมีคนโกหกทั้งหมดกี่คน

\bigskip
\underline{\textbf{โจทย์}}  เขียนโปรแกรมรับคำตอบของผู้ตอบคำถามแต่ละคน จากนั้นวิเคราะห์ว่าถ้าคนจำนวนมากที่สุดที่เป็นไปได้ตอบคำถามเป็นความจริงแล้วจะมีคนโกหกทั้งหมดกี่คน

\InputFile

\textbf{บรรทัดแรก} รับจำนวนเต็ม $N$ $(1 \leq N \leq 20)$, $M$ $(1 \leq M \leq 20)$ แทนจำนวนคนตอบกระทู้ และจำนวนคำถาม ตามลำดับ

\textbf{บรรทัดที่ $2$ ถึง $N+1$} จะมีจำนวนเต็มบรรทัดละ $M$ ตัว แทนข้อมูลการตอบคำถามของนักเลงคอมพิวเตอร์แต่ละคนในแต่ละคำถาม โดยจำนวนเต็มตัวที่ $j$ ของบรรทัดที่ $i + 1$ จะระบุคำตอบของนักเลงคอมพิวเตอร์หมายเลข $i$ ต่อคำถามที่ $j$ ซึ่งจำนวนเต็มดังกล่าวแปลความหมายดังนี้

\begin{itemize}

\item \textbf{1} แทนคำตอบว่า \textbf{ใช่}
\item \textbf{-1} แทนคำตอบว่า \textbf{ไม่ใช่}
\item \textbf{0 }แทน \textbf{ไม่ตอบกระทู้}
\end{itemize}

\OutputFile

\textbf{มีบรรทัดเดียว} พิมพ์จำนวนคนที่โกหก เมื่อคนจำนวนมากที่สุดที่เป็นไปได้ไม่ได้โกหก


\Examples

\begin{example}
\exmp{3 3
0 1 0
-1 -1 1
1 1 1}{1}%
\end{example}

\Note 

\textbf{อธิบายตัวอย่างข้อมูลนำเข้าและส่งออก}

คนที่ $2$ เป็นคนเดียวที่โกหกในกระทู้ที่ $1$ และ $2$ ส่วนคนที่ $1$ กับ $3$ พูดความจริง นี่เป็นรูปแบบที่คนไม่โกหกมีจำนวนมากที่สุดที่เป็นไปได้ (สังเกตได้ว่า คนที่ $2$ อาจไม่ได้โกหกในทุกกระทู้ที่เขาตอบ)

\Source

ภัทร สุขประเสริฐ 

\underline{\href{http://thailandoi.org/toi.c/01-2009}{TOI.C:01-2009}}


\end{problem}

\end{document}