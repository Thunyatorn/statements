\documentclass[11pt,a4paper]{article}

\usepackage{../../templates/style}

\begin{document}

\begin{problem}{$X_2$}{standard input}{standard output}{1 second}{32 megabytes}

$S$ คือ ค่าเฉลี่ยของตัวเลข $2$ จำนวน $X_1$ และ $X_2$ โดย $S$ มีค่าเท่ากับ $(X_1+X_2)/2$
สมชายให้ของขวัญปรีดา $2$ ชิ้น ซึ่งมีมูลค่า $X_1$ และ $X_2$ โดยที่ $X_1$ และ $X_2$ เป็นจำนวนเต็ม ปรีดาได้คำนวณมูลค่าเฉลี่ยของของขวัญ $2$ ชิ้นนั้นซึ่งมีค่าเป็นจำนวนเต็มเช่นกัน แต่ปรีดาได้ทำมูลค่าของของขวัญชิ้นที่ $2$ $(X_2)$ หายไปกรมเมอร์ชั้นเซียนต่อไป

\underline{\textbf{โจทย์}} จงเขียนโปรแกรมหาค่า $X_2$

\InputFile

\textbf{มีบรรทัดเดียว} ประกอบด้วย เลขจำนวนเต็ม $2$ จำนวน คือ $X_1$ และ $S$ ตามลำดับ โดยทั้งนี้ค่า $X_1$ และ $S$ จะมีค่าอยู่ในช่วง $-1\,000$ ถึง $1\,000$


\OutputFile

\textbf{มีบรรทัดเดียว} เป็นค่าผลลัพธ์

\Examples

\begin{example}
\exmp{11 15
}{19
}%
\exmp{4 3
}{2
}%
\end{example}

\Source

Croatian Open Competition in Informatics

Contest 2 – November 25, 2006

\end{problem}

\end{document}