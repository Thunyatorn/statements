\documentclass[11pt,a4paper]{article}

\usepackage{../../templates/style}

\begin{document}

\begin{problem}{SLIKAR}{standard input}{standard output}{1 second}{32 megabytes}

Josip เป็นจิตรกรที่มีนิสัยแปลก ๆ   เขาต้องการที่จะระบายสีลงบนรูปภาพที่มีขนาด $N \times N$ พิกเซล โดยที่ $N$ สามารถเขียนให้อยู่ในรูปของสองยกกำลังตัวเลขใดๆ ($1, 2, 4, 8, 16$ และอื่น ๆ)    ในแต่ละพิกเซลจะต้องเป็นสีขาวหรือดำเท่านั้นและ Josip ก็มีแนวทางในการระบายสีลงในแต่ละพิกเซลแล้วด้วย
การระบายสีนี้ของ Josip ไม่น่าที่จะมีปัญหาอะไร ถ้าเขาไม่ระบายสีด้วยวิธีการแปลก ๆ โดยเขาได้ใช้วิธีการระบายสีแบบเรียกซ้ำ ดังนี้ 

\begin{itemize}

\item ถ้ารูปภาพมีขนาด \textit{pixel} เดียว  เขาจะระบายสีลงไปบนภาพนั้นตามแนวทางที่เขาตั้งใจ

\item ถ้าไม่เช่นนั้น   เขาจะแบ่งรูปภาพออกเป็นรูปสี่เหลี่ยมขนาดเล็ก $4$ รูป แล้วทำดังนี้
\begin{enumerate}
\item เลือกรูปเหลี่ยมขนาดเล็กจาก $1$ ใน $4$ รูปแล้วระบายสีขาวลงไป

\item เลือกรูปสี่เหลี่ยมขนาดเล็กจาก $1$ ใน $3$ ของรูปที่เหลือ แล้วระบายสีดำลงไป

\item จากนั้น เขาจะพิจารณารูปสี่เหลี่ยมขนาดเล็ก $2$ รูปที่เหลือเสมือนว่าเป็นการระบายสีครั้งใหม่ และใช้วิธีการ 3 ขั้นตอนนี้กับรูปเหล่านั้น
\end{enumerate} \end{itemize} 

เมื่อเร็ว ๆ นี้เขาสังเกตพบว่า  มันเป็นไปไม่ได้ที่จะเปลี่ยนการมองเห็นภาพของเขามาเป็นการระบายสีด้วยวิธีการนี้ได้

\bigskip
\underline{\textbf{โจทย์}}  จงเขียนโปรแกรมที่สามารถระบายสีลงบนรูปภาพ ให้เกิดความแตกต่างจากภาพที่ต้องการให้น้อยที่สุดเท่าที่จะเป็นไปได้ ความแตกต่างระหว่างรูปทั้งสองนี้จะถูกคำนวณจากจำนวนของสีที่แตกต่างกันในแต่ละคู่ของพิกเซลที่ตำแหน่งตรงกัน


\InputFile

\textbf{บรรทัดแรก} ประกอบด้วยเลขจำนวนเต็ม $N$ $(1 \leq N \leq 512)$ ซึ่งเป็นขนาดของรูปที่ Josip ต้องการจะระบายสีลงไป และ $N$ สามารถเขียนให้อยู่ในรูปของสองยกกำลังตัวเลขใดๆ

\textbf{บรรทัดที่ $2$ ถึง $N+1$} แต่ละบรรทัดจะประกอบด้วยเลขจำนวนเต็ม $0$ หรือ $1$ จำนวน $N$ ตัวซึ่งหมายถึงสี่เหลี่ยมสีขาวและดำในรูปเป้าหมาย


\OutputFile

\textbf{มีบรรทัดเดียว}   ให้แสดงผลข้อมูลส่งออกของค่าความแตกต่างที่น้อยที่สุดที่สามารถทำได้ เมื่อคุณระบายสีตามรูปแบบ

\Examples

\begin{example}
\exmp{4
0001
0001
0011
1110}{1}%
\exmp{4
1111
1111
1111
1111}{6}%
\exmp{8
01010001
10100011
01010111
10101111
01010111
10100011
01010001
10100000}{16}%
\end{example}

  
\Source

COCI 2008/2009, Contest \#4 – January 17, 2009 :: ดัดแปลงเล็กน้อย (:


\end{problem}

\end{document}